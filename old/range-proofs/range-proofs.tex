In a Range Proof one wants to prove that a GS commitment $[\vecb{c}]_1$ opens to some integer $y$ in the range $[0,2^n-1]$. That is, construct a NIZK proof system for the language
$$
\Lang_{[0,2^n-1],\vecb{u}_1,\vecb{u}_2} := \{[\vecb{c}]_1\in\GG_1^2: \exists y,r\in\Z_q\text{ s.t. }[\vecb{c}]_1=\GS.\Com(y;r)\wedge y\in[0,2^n-1]\}
$$
The most efficient Range Proof under falsifiable assumptions, specifically the $q$-Hidden Strong Diffie-Hellman Assumption, is of size $\Theta(\frac{n}{\log n-\log\log n})$ \cite{PAIRING:RiaKohPre09}.
In general, the approach from \cite{PAIRING:RiaKohPre09}, which closely follows \cite{AC:CamChaShe08}, can be summarized as follows:

\begin{enumerate}[label=\alph*)]
\item Commit to $y_1,\ldots y_\ell$.
\item Show that $y_i\in[0,d-1]$, for each $i\in[\ell]$. \label{rp:b}
\item Show that $y=\sum_{i\in[\ell]}y_id^{i-1}$.
\end{enumerate}

Given that the maximum $y$ expressible as $\sum_{i\in[\ell]}x_id^{i-1}$ is $\sum_{i\in[\ell]}(d-1)d^{i-1}=d^\ell-1$, it must hold that $\ell=n/\log d$. The total size of the proof is $\mathsf{S}_{[0,d-1]}(\ell)+\Theta(\ell)$, where $\mathsf{S}_{[0,d-1]}(\ell)$ is the size of $\ell$ Range Proofs in the interval $[0,d-1]$.

\subsection{Our Construction}
Note that \ref{rp:b} is equivalent to show that $(\GS.\Com(y_1)||\cdots||\GS.\Com(y_\ell))\in\Lang_{d,\vecb{u}_1,\vecb{u}_2}^\ell$. Thus, using the proof system from Sect. \ref{sec:bin-lang} we are able to aggregate $\ell$ Range Proofs in the interval $[0,d-1]$ into a single proof of size $\Theta(\log d)$. Choosing $d=n^k$ we get that $\mathsf{S}_{[0,d-1]}(l)=\Theta(k\log n)$ and $l=n/\log n^k=\frac{n}{k\log n}$, and thus the size of our Range Proof is $\Theta(\frac{n}{k\log n})$ for an arbitrarily chosen $k\in\mathbb{N}$. One would be tempted to choose $d=2^{\sqrt{n}}$ to obtain a proof of size $\Theta(\sqrt{n})$. However, the proof system from Sect. \ref{sec:bin-lang} requires $\Lang_{d,\vecb{u}_1,\vecb{u}_2}$ to be \emph{efficiently falsifiable}, which seems to be infeasible as when $d=2^{\sqrt{n}}$.

A detailed description of our proof system can be found in Fig. \ref{fig:rp}.

\begin{figure} 
$$
\begin{array}{ll}
\begin{array}{l}
\algK_1(\gk,[\vecb{u}]_1,[\vecb{u}_2]_1,n)\quad (\mathsf{S}_1(\gk,[\vecb{u}_1]_1,[\vecb{u}_2]_1,n))\\
\hline
d:=n^k, m:=\log d, \ell := n/m\\
\crs_\GS\gets\GS.\algK_1(\gk)\\
{\crs_\sfset\gets\Pi_\sfset.\algK_1(\gk,[\matr{M}]_1,[\matr{N}]_1,n)}\\
\text{Return } \ \crs:=([\vecb{u}_1]_1,[\vecb{u}_2]_1,\crs_\GS,\crs_\sfset).\\
(\tau_\GS\gets\GS.\algS_1(\gk)\\
\tau_\sfset\gets\Pi_\sfset.\algS_1(\gk,[\matr{M}]_1,[\matr{N}]_1,n).\\
\tau := (\matr{S},\tau_\GS,\tau_\sfset)).\\
\\
\end{array}
&
\begin{array}{l}
{\algP(\crs, [\matr{c}]_1, \langle y, r\rangle)}\\
\hline
\vecb{y}\in\Z_q^{\ell} \text{ is s.t. } y=\sum_{i\in[\ell]}{y}_id^{i-1}\\
\matr{B}\in\bits^{m\times\ell} \text{ is s.t. }\\
\qquad \vecb{y}^\top = (2^0,\ldots,2^{m-1})\matr{B}\\
\pi_\GS\gets\GS.\algP(\crs_\GS,\eq,\langle \vecb{y},r\rangle)\\
\pi_\sfset \gets \Pi_\sfset.\algP(\crs_\sfset, [\matr{X}]_1,\langle\matr{B},\matr{R}\rangle)\\
//{[\matr{X}]_1} = \GS.\Com_{\crs_\GS}(\vecb{y}^\top;\matr{R})\\
\text{Return } \  (\pi_\GS,\pi_\sfset). \\
\\
\end{array}\\
\begin{array}{l}
{\algV(\crs,[\vecb{c}]_1,(\pi_\GS,\pi_\sfset))}\\
\hline
\mathsf{ans}_1 \gets \GS.\algV(\crs_\GS,\eq,\pi_\GS)\\
\mathsf{ans}_2 \gets \Pi_\sfset.\algV(\crs_\sfset,[\matr{X}]_1,\pi_\sfset)\\
\text{Return } \ \mathsf{ans}_1\wedge\mathsf{ans}_2.
\\
\\
\end{array}
&
\begin{array}{l}
{\mathsf{S}_2(\crs,[\vecb{c}]_1,\tau)}\\
\hline
\pi_\GS\gets \GS.\algS_2(\crs_\GS,\eq,\tau_\GS)\\
\pi_\sfset\gets \Pi_\sfset.\algS_2(\crs_\sfset,[\matr{X}]_1,\tau_\sfset)\\
//{[\matr{X}]_1=\GS.\Com_{\crs_\GS}(\vvar{Y};\matr{R})}\\
\text{Return }  (\pi_\GS,\pi_\sfset).
\end{array}
\end{array}$$
\caption{The proof system $\Pi_\rp$ for the language $\Lang_{ck,[0,2^n-1]}$. $\Pi_\sfset$ is the proof system from Sect. \ref{sec:bin-lan-constr}. The matrices $\matr{M},\matr{N}$ are defined as $\matr{M}:=\vecb{u}'_1(2^0,2^1,\ldots,2^{m-1}),\matr{N}:=\vecb{u}'_2$, where $\vecb{u}'_1,\vecb{u}'_2$ are the GS commitment keys from $\crs_\GS$. The equation $\eq$ is defined as $[\vecb{c}]_1-\sum_{i\in[\ell]}{y}_id^{i-1}[\vecb{u}_1]_1=r[\vecb{u}_2]_1$.
\label{fig:rp}}
\end{figure}


