\subsubsection{Completeness.}
If $[\vecb{c}]_1=\GS.\Com(y;r)$ and $y\in[0,2^n-1]$, then there exists $y_1,\ldots,y_\ell\in[0,d-1]$  such that $y=\sum_{i\in[\ell]}y_id^{i-1}$. Therefore $\vvar{Y}=(y_1,\ldots,y_\ell)$ and $\vvar{z}=r$ are solutions to $\eq$ and $[\matr{X}]_1=\GS.\Com(\vvar{Y})\in\Lang_{d,\vecb{u}'_1,\vecb{u}'_2}^\ell$.

\subsubsection{Soundness.}
%\begin{theorem}
%Let $\adv(\advA)$ be the advantage of an adversary $\advA$ against the soundness of the proof system described in Fig. \ref{fig:rp}. There exists PPT an adversary $\advB$ against $\Pi_\sfbin$ such that
%$$
%\adv_{\mathcal{PS}}(\advA)\leq \adv_{\Pi_\sfbin}(\advB).
%$$
%\end{theorem}
Given an adversary $\advA$ against the soundness of the proof system from Fig. \ref{fig:rp}, we construct an adversary $\advB$ against the soundness of $\Pi_\sfbin$. If $y\notin[0,2^n-1]$, then the perfect soundness if GS proofs implies that there is some $\vvar{y}_i\notin[0,d-1]$. Therefore $[\matr{X}]_1=\GS.\Com(\vvar{Y})\notin\Lang_{d,vecb{u}'_1,\vecb{u}'_2}^\ell$ and $\adv_{\Pi_\mathsf{range-proof}}(\advA)\leq\adv_{\Pi_\sfbin}(\advB)$.

\subsubsection{Zero-Knowledge.} Follows directly from Zero-Knowledge of GS proofs, the fact that $\eq$ allows simulation, and the fact that $[\matr{X}]_1=\GS.\Com(\matr{0}_{1\times n})\in\Lang_{d,\vecb{u}'_1,\vecb{u}'_2}^\ell$.
