We give a NIZK proof of mebership in the following language

$$
\Lang_{\hmatr{U},\sfbits} :=
\left\{
\hvecb{c}\in\Gr^{\tilde{n}+m} :
\hvecb{c}:=\hvecb{U}_1\vecb{b}+\hvecb{U}_2\vecb{w},
(\vecb{b},\vecb{w})\in\bits^{n}\times\Z_q^m
\right\}
$$
Where $\hvecb{U}:=(\hvecb{U}_1,\hvecb{U}_2)\in\Gr^{(\tilde{n}+m)\times n}\times\Gr^{(\tilde{n}+m)\times m}$. We do not require $\hvecb{U}$ to define perfectly binding commitments, in contrast to XXX. However, this relaxation is not for free since our proofs are now of size $\Theta(\ell)$, where $\ell:=n/\tilde{n}$ ($\ell:=n/\rank(\hmatr{U})$?), and we require $\ell = O(\log \lambda)$.

\subsection{Simple proof of size $\Theta(\ell^2)$}

The next construction is a simple adaptation of XXX.

\begin{description}

\item[$\algK_0(1^\lambda)$:]  Return $\Gamma := (q,\Gr,\Hr,\T,e,\hat{g},\check{h}) \leftarrow \ggen_a(1^{\lambda})$.

\item[$\dist_\Gamma$:]
The distribution $\dist_\Gamma$ over $\Gr^{(\tilde{n}+1) \times (n+1)}$ is some witness samplable distribution which 
defines the relation $\R_\Gamma = \{\R_{\hmatr{U}}\} 
\subseteq \Gr^{\tilde{n}+1}\times(\bits^n\times\Z_q)$,
where $\hmatr{U}\gets\dist_\Gamma$,
such that $(\hvecb{c},\langle\vecb{b}, w\rangle)\in\R_{\hmatr{U}}$ iff
$\hvecb{c}=\hmatr{U}\binom{\vecb{b}}{w}$. The relation $\R_{par}$ consists of pairs $(\hmatr{U},\matr{U})$ where $\hmatr{U} \gets \dist_{\Gamma}$.
\item[$\algK_1(\Gamma, \hmatr{U})$:]
Let $\lrck_{n+1}\gets \Z_q^{\ell+1}$
and for all $i \in [n]$, $\lrck_{i}:=\epsilon_{i}\lrck_{n+1}$, where
$\epsilon_{i} \leftarrow \Z_q$. Define
$\Rck := (\rck_1|| \ldots ||\rck_{n+1})$.
Choose 
$\matr{\Delta} \leftarrow \Z_q^{(\ell+1) \times (n+1)}$,
define $\Lck := \matr{\Delta}\hmatr{U}$
and $\hat{\llck}_{i}:=\matr{\Delta} \hat{\vecb{u}}_i \in \Gr^{\ell+1}$, for all $i \in [n+1]$. 
Let $\vecb{a} \leftarrow \distlin_{\ell}$ and define $\cvecb{a}_{\Delta}:=\matr{\Delta}^\top\cvecb{a} \in \Hr^{n+1}$. 
For any pair $(i,j) \in \indexSet{n}{1}$, let 
$\matr{T}_{i,j}\gets\Z_q^{(\ell+1)\times(\ell+1)}$ and set:
$$\hmatr{C}_{i,j}:=\lck_i \lrck_j^{\top} - \hmatr{T}_{i,j}  \in \Gr^{(\ell+1) \times (\ell+1)},
\qquad \qquad 
\cmatr{D}_{i,j}:=\cmatr{T}_{i,j} \in \Hr^{(\ell+1) \times (\ell+1)}.$$ 
Note that $\hmatr{C}_{i,j}$ can be efficiently computed 
as $\lrck_j \in \Z_q^{\ell+1}$ is the vector of discrete logarithms of $\check{\lrck}_j$.

Let $\QANIZKsum$ be the proof system for Sum in Subspace 
(Sect. \ref{sec:QANIZKsum}) and $\QANIZKcomms$
%=(\algK_0,\algK_1,\algP,\algV,\algS_1,\algS_2)$
be an instance of our proof system for Equal Opening (Sect. \ref{sec:aggcomms}).

Let
$\crs_\QANIZKsum \gets \algK_1(\Gamma, \{\hmatr{C}_{i,j},\cmatr{D}_{i,j}\}_{(i,j)\in\indexSet{n}{1}})$.\footnote{We identify
matrices in $\Gr^{(\ell+1) \times (\ell+1)}$ (resp. in $\Hr^{(\ell+1) \times (\ell+1)}$) with vectors in $\Gr^{(\ell+1)^2}$ (resp. in $\Hr^{(\ell+1)^2}$).} Generate $\crs_\QANIZKcomms \gets \algK_1(\Gamma, \Lck,\Rck,n)$. 
The common reference string is given by:
\begin{eqnarray*}
\mathsf{crs}_P&:=&\left( \hat{\matr{U}},  \Lck,
    \check{\Lrck}, \{\hmatr{C}_{i,j},\cmatr{D}_{i,j} \}_{(i,j) \in \indexSet{n}{1}},\crs_\QANIZKsum,\crs_\QANIZKcomms \right), \\
\mathsf{crs}_V&:=&\left(\cvecb{a}, \cvecb{a}_\Delta, \crs_\QANIZKsum,\crs_\QANIZKcomms \right). 
 \end{eqnarray*}
\item[$\algP(\mathsf{crs}_P, \hvecb{c}, \langle \vecb{b}, w_g \rangle)$:]
Pick $w_h \gets \Z_q$,  $\matr{R} \gets \Z_q^{(\ell+1)\times (\ell+1)}$ and then: 
\begin{enumerate}
\item Define 
%$$\hat{\vecb{c}}:=\hat{\matr{U}}\begin{pmatrix} b_1\\ \vdots \\ b_n \\ w \end{pmatrix},  \qquad
$$\hvecb{c}_{\Delta} := \Lck \begin{pmatrix} \vecb{b} \\ w_g \end{pmatrix},
\qquad \cvecb{d} := \Rck \begin{pmatrix} \vecb{b} \\ w_h \end{pmatrix}.$$ 
\item Compute 
 $(\hmatr{\Theta}_{b(\overline{b}-1)}, \cmatr{\Pi}_{b(\overline{b}-1)})\, :=$
\begin{eqnarray} \label{eq:ThetaPi}
%(\hmatr{\Theta}_{b(\overline{b}-1)}, \cmatr{\Pi}_{b(\overline{b}-1)})
%& := &
%    \sum_{i \in [n]}\left(
%        b_i w_h (\hmatr{C}_{i,n+1},\cmatr{D}_{i,n+1})+
%        w_g(b_i-1) (\hmatr{C}_{n+1,i}, \cmatr{D}_{n+1,i})\right)
%        \nonumber\\ & &           +
%       \sum_{i \in [n]}  \sum_{\substack{j \in [n]\\ j\neq i}} b_i (b_j-1) (\hmatr{C}_{i,j}, \cmatr{D}_{i,j})\nonumber\\
%       & &
%     +
%    w_gw_h (\hmatr{C}_{n+1,n+1}, \cmatr{D}_{n+1,n+1}) +  (\hmatr{R},-\cmatr{R}).
& &
    \sum_{i \in [n]}\left(
        b_i w_h (\hmatr{C}_{i,n+1},\cmatr{D}_{i,n+1})+
        w_g(b_i-1) (\hmatr{C}_{n+1,i}, \cmatr{D}_{n+1,i})\right)
        \nonumber\\ & &           +
       \sum_{i \in [n]}  \sum_{\substack{j \in [n]\\ j\neq i}} b_i (b_j-1) (\hmatr{C}_{i,j}, \cmatr{D}_{i,j})\nonumber\\
       & &
     +
    w_gw_h (\hmatr{C}_{n+1,n+1}, \cmatr{D}_{n+1,n+1}) +  (\hmatr{R},-\cmatr{R}).
%\\
%(\hmatr{\Theta}_{b-\overline{b}}, 
%\cmatr{\Pi}_{b-\overline{b}})
%  & := &
%    \sum_{i\in[n]}\left(
%        (b_i-w_h)(\hmatr{C}_{i,n+1},\cmatr{D}_{i,n+1})
%         +
%        (w_g-b_i)(\hmatr{C}_{n+1,i}, \cmatr{D}_{n+1,i})\right)\\
%     & &  + \sum_{i\in[n]}\sum_{\substack{j\in[n]\\j\neq i}} (b_i-b_j)(\hmatr{C}_{i,j}, \cmatr{D}_{i,j}) 
%\\ & &+
%    (w_g-w_h)(\hmatr{C}_{n+1,n+1},\cmatr{D}_{n+1,n+1}) + (\hmatr{S},-\cmatr{S}).
 \end{eqnarray}

\item Compute a proof $(\hat{\boldsymbol \rho}_{b(\overline{b}-1)},\check{\boldsymbol \sigma}_{b(\overline{b}-1)})$
that $\matr{\Theta}_{b(\overline{b}-1)}+\cmatr{\Pi}_{b(\overline{b}-1)}$
belongs to the space spanned by $\{\matr{C}_{i,j}+\matr{D}_{i,j}\}_{(i,j)\in\indexSet{n}{1}}$,
 and a proof 
$(\hgrkb{\rho}_{b-\overline{b}}, \cgrkb{\sigma}_{b-\overline{b}})$
that
$(\hvecb{c}_\Delta,\cvecb{d})$ open to the same value,
using $\vecb{b},w_g$, and $w_h$. 
\end{enumerate}

\item[$\algV(
    \mathsf{crs}_V,
    \hvecb{c},
    \langle
        \hvecb{c}_{\Delta}, \cvecb{d},
        (\hmatr{\Theta}_{b(\overline{b}-1)}, \cmatr{\Pi}_{b(\overline{b}-1)}), 
        \{(\hat{\boldsymbol \rho}_{X}, \check{\boldsymbol \sigma}_{X})\}_{X \in \{b(\overline{b}-1), b-\overline{b}\}} \rangle)$:] ~
%   
\begin{enumerate}
\item  Check if $\hvecb{c}^\top\cvecb{a}_\Delta = \hvecb{c}_\Delta^\top\cvecb{a}$. 
\item Check if 
\begin{equation}\label{eq:ver1}\hvecb{c}_{\Delta}
\left(
    \check{\vecb{d}}-
    \sum_{j \in [n]} \rck_{j}
\right)^{\top} =
    \hmatr{\Theta}_{b(\overline{b}-1)} \cmatr{I}_{2 \times 2} +
    \hmatr{I}_{2 \times 2}\cmatr{\Pi}_{b(\overline{b}-1)}.
    \end{equation}  
  \item Verify that $(\hgrkb{\rho}_{b(\overline{b}-1)}, \cgrkb{\sigma}_{b(\overline{b}-1)}),(\hgrkb{\rho}_{b-\overline{b}},\cgrkb{\sigma}_{b-\overline{b}})$ are valid proofs for %\linebreak 
  $(\hmatr{\Theta}_{b(\overline{b}-1)},$ $\cmatr{\Pi}_{b(\overline{b}-1)})$
        and $(\hvecb{c}_\Delta,\cvecb{d})$ using $\crs_\QANIZKsum$ and $\crs_\QANIZKcomms$ respectively.
\end{enumerate}
If any of these checks fails, the verifier outputs $0$, else it outputs $1$.
%\item[$\mathsf{S}_1(\Gamma,\hat{\matr{U}})$:] The simulator receives as input a description of an asymmetric bilinear group $\Gamma$ and a matrix $\hat{\matr{U}} \in \Gr^{(n+1) \times (n+1)}$ sampled according to distribution $\dist_{\Gamma}$. It generates and outputs the CRS in the same way as $\algK_1$, but additionally it also  outputs the simulation trapdoor 
%$$\tau=\left(\Lrck, \matr{\Delta}, \tau_\QANIZKsum, \tau_\QANIZKcomms\right),$$
%where $\tau_\QANIZKsum$ and $\tau_\QANIZKcomms$ are, respectively, $\QANIZKsum$'s and $\QANIZKcomms$'s simulation trapdoors.
%\item[$\mathsf{S}_2(\crs_P,\vc,\tau)$:] Compute $\vc_{\Delta}:=\matr{\Delta} \vc$.
%      Then pick random $\overline{w}_h \gets \Z_q$, $\matr{R} \gets \Z_q^{2 \times 2}$ and define 
% $\vecb{d}:= \overline{w}_{h} \lrck_{n+1}.$
% Then set:
%\begin{align*} 
%\hmatr{\Theta}_{b(\overline{b}-1)} & :=  \hvecb{c}_\Delta \left(\vecb{d}-\sum_{i \in [n]} \lrck_i\right)^\top + \hmatr{R},
%    &
%    \cmatr{\Pi}_{b(\overline{b}-1)} & := - \cmatr{R}.
%%\\
%%\hmatr{\Theta}_{b-\overline{b}} & :=  \hvecb{c}_\Delta \left( \sum_{i\in[n]}\lrck_{i}^\top \right) -
%%                          \left( \sum_{i\in[n]}\hvecb{u}_{i} \right)\vecb{d}^\top + \hmatr{S},
%%    &
%%    \cmatr{\Pi}_{b-\overline{b}}& := - \cmatr{S}.
%\end{align*}
%Finally, simulate proofs $(\hat{\boldsymbol \rho}_{X},
%  \check{\boldsymbol \sigma}_{X})$
%%for $(\hmatr{\Theta}_{X}, \cmatr{\Pi}_{X})$
%for $X \in \{b(\overline{b}-1),  b-\overline{b}
%\}$  using $\tau_\QANIZKsum$ and $\tau_\QANIZKcomms$.
%
%\end{description} 
%The simulators $\algS_1$ and $\algS_2$ are defined as follows.
%\begin{description}
\item[$\mathsf{S}_1(\Gamma,\hat{\matr{U}})$:] The simulator receives as input a description of an asymmetric bilinear group $\Gamma$ and a matrix $\hat{\matr{U}} \in \Gr^{(n+1) \times (n+1)}$ sampled according to distribution $\dist_{\Gamma}$. It generates and outputs the CRS in the same way as $\algK_1$, but additionally it also  outputs the simulation trapdoor 
$$\tau=\left(\Lrck, \matr{\Delta}, \tau_\QANIZKsum, \tau_\QANIZKcomms\right),$$
where $\tau_\QANIZKsum$ and $\tau_\QANIZKcomms$ are, respectively, $\QANIZKsum$'s and $\QANIZKcomms$'s simulation trapdoors.
\item[$\mathsf{S}_2(\crs_P,\vc,\tau)$:] Compute $\vc_{\Delta}:=\matr{\Delta} \vc$.
      Then pick random $\overline{w}_h \gets \Z_q$, $\matr{R} \gets \Z_q^{2 \times 2}$ and define 
 $\vecb{d}:= \overline{w}_{h} \lrck_{n+1}.$
 Then set:
\begin{align*} 
\hmatr{\Theta}_{b(\overline{b}-1)} & :=  \hvecb{c}_\Delta \left(\vecb{d}-\sum_{i \in [n]} \lrck_i\right)^\top + \hmatr{R},
    &
    \cmatr{\Pi}_{b(\overline{b}-1)} & := - \cmatr{R}.
%\\
%\hmatr{\Theta}_{b-\overline{b}} & :=  \hvecb{c}_\Delta \left( \sum_{i\in[n]}\lrck_{i}^\top \right) -
%                          \left( \sum_{i\in[n]}\hvecb{u}_{i} \right)\vecb{d}^\top + \hmatr{S},
%    &
%    \cmatr{\Pi}_{b-\overline{b}}& := - \cmatr{S}.
\end{align*}
Finally, simulate proofs $(\hat{\boldsymbol \rho}_{X},
  \check{\boldsymbol \sigma}_{X})$
%for $(\hmatr{\Theta}_{X}, \cmatr{\Pi}_{X})$
for $X \in \{b(\overline{b}-1),  b-\overline{b}
\}$  using $\tau_\QANIZKsum$ and $\tau_\QANIZKcomms$.
\end{description}

\begin{definition}
Let $\algK_1,\algP,$ and $\algV$ the algorithms defined exactly as algorithms $\algK_1,\algP,$ and $\algV$, respectively, in the proof system from XXX but with the following differences:
\begin{description}
\item[$\algK_1$:] It does not receive $\hmatr{U}$ as input (and is not part of the CRS), instead receives a positive integer $m$. Also, $\cvecb{a}$ and $\cvecb{a}_\Delta$ are discarded, and $\llck_i := \gamma_i\llck_{m+1}$, where $\gamma_i\gets\Z_q$, for all $i\in[m]$.
\item[$\algP$:] It does not receive $\hvecb{c}$ as input and $\hvecb{c}_\Delta$ and $\cvecb{d}$ are renamed as $\cgrkb{\alpha}$ and $\cgrkb{\beta}$ respectively.
\item[$\algV$:] It does not receive $\hvecb{c}$ as input and the first check ($\hvecb{c}^\top\hvecb{a}=\cvecb{c}_\Delta^\top\cvecb{a}$) is omitted.
\end{description}

Additionally, define define algorithm $\widetilde{\algK}_1$ exactly as the algorithm $\algK_1$ defined above, with the following differences:
\begin{enumerate}[label=(\alph*)]
\item It receives as additional input an integer $i^*\in[m]$.
\item $\llck_{i^*}\gets\Z_q^2$ and $\lrck_{i^*}\gets\Z_q^2$.
\item It outputs $(\widetilde{\crs}_\sfbits,\tau)$, where $\widetilde{\crs}_\sfbits$ is equal to $\crs_\sfbits$ from $\algK_1$ (but with the change mentioned above), and the trapdoor $\tau:=(\Llck,\Lrck)$.
\end{enumerate}
\end{definition}

\begin{lemma}
For all $i^*\in[m]$, all PPT adversary $\advA$, and all PPT distinguisher $\advD$ there exists soundness adversaries $\algP_1^*,\algP^*_2$ for $\QANIZKsum$ and $\QANIZKcomms$, $\mathcal{U}_1\mbox{-}\mddh_\Gr$ adversary $\advB_1$, and $\mathcal{U}_1\mbox{-}\mddh_\Hr$ adversary $\advB_2$ such that
\begin{enumerate}[label=(\roman*)]
\item If $(\Lck,\Rck,\widetilde{\crs}'_{\sfbits},\tau)\gets\widetilde{\algK}_1(\Gamma,m,i^*)$, where $\Gamma\gets\algK_0(1^\lambda)$, then $\lck_{i^*}$ and $\rck_{i^*}$ are linearly independent from $\{\lck_{i}:i\neq i^*\}$ and $\{\rck_{i}:i\neq i^*\}$, respectively, with probability at least $1-4/q$.
\item $\adv_{\mathsf{dist}}(\advD,i^*)\leq \adv_{\mathcal{U}_1\mbox{-}\mddh_\Gr}(\advB_1)+\adv_{\mathcal{U}_1\mbox{-}\mddh_\Hr}(\advB_2)$, where $$\adv_{\mathsf{dist}}(\advD,i^*):=
\left|\begin{matrix*}[l]
    \Pr[\crs_\sfbits\gets\algK_1(\Gamma,m):\advD(\Gamma,\crs_\sfbits)=1]-\\
    \Pr[(\widetilde{\crs}_\sfbits,\tau)\gets\widetilde{\algK}_1(\Gamma,m,i^*):\advD(\Gamma,\widetilde{\crs}_\sfbits)=1]
\end{matrix*}\right|.$$
\item $\adv_{\mathsf{sound}}(\advA,i^*) \leq 4/q + \adv_\QANIZKsum(\algP_1^*)+\adv_\QANIZKcomms(\algP_2^*)$, where $$
\adv_{\mathsf{sound}}(\advA,i^*):=
\Pr\begin{bmatrix*}[l]
        (\widetilde{\crs}_\sfbits,\tau)\gets\widetilde{\algK}_1(\Gamma,m,i^*);
        (\hgrkb{\alpha},\cgrkb{\beta},\pi')\gets\advA(\Gamma,\widetilde{\crs}_\sfbits,\tau):\\
            \algV(\widetilde{\crs}_\sfbits,(\hgrkb{\alpha},\cgrkb{\beta},\pi'))=1\text{ and }\\
            \neg\begin{pmatrix*}[l]
                \exists b_{i^*}\in\bits,\overline{w}_g,\overline{w}_h\in\Z_q\text{ s.t. }\\
                \hgrkb{\alpha}=b_{i^*}\lck_{i^*}+\overline{w}_g\lck_{m+1},\cgrkb{\beta}=b_{i^*}\rck_{i^*}+\overline{w}_h\rck_{m+1}
            \end{pmatrix*}
      \end{bmatrix*}
       .$$
\end{enumerate}
\label{lemma:spc-sound}
\end{lemma}

\begin{proof}
\begin{enumerate}[label=(\roman*)]
\item Given that $\lck_i=\gamma_i\lck_{m+1}$ and $\rck_i=\epsilon_i\rck_{m+1}$ for all $i\neq i^*$, the probability that $\lck_{i^*}$ and $\rck_{i^*}$ are linearly dependent from $\{\lck_i:i\neq i^*\}$ and $\{\rck_i:i\neq i^*\}$, respectively, is
\begin{eqnarray*}
\Pr[\lck_{i^*}\text{ is l.d. from }\lck_{m+1}\vee\lck_{i^*}\text{ is l.d. from }\lck_{m+1}] & \leq & 2\Pr[\lck_{i^*}\text{ is l.d. from }\lck_{m+1}]\\
& = & 2\Pr[\det((\lck_{i^*}||\lck_{m+1}))=0]\\
& \leq & 4/q.
\end{eqnarray*}

\item The proof follows from the indistinguishability of three games
\begin{itemize}
\item[$\sfGame_1$:] In this game $\crs_\sfbits$ is sampled from $\algK_1(\Gamma,m)$.
\item[$\sfGame_2$:] This game is exactly as the previous but now $\lck_{i^*}\gets\Z_q^2$.
\item[$\sfGame_3$:] In this game $\crs_\sfbits$ is sampled from $\widetilde{\algK}_1(\Gamma,m,i^*)$.
\end{itemize}
Note that with this definitions $\adv_{\mathsf{dist}}(\advD,i^*)=|\Pr[\sfGame_1(\advD)=1]-\Pr[\sfGame_3(\advD)=1]|$

\begin{lemma} There exists a $\mathcal{U}_1$-$\mddh_\Gr$ adversary $\advB$ such that
$|\Pr\left[\mathsf{Game}_{1}(\advA)=1\right]$ $-\Pr\left[\mathsf{Game}_{2}(\advA)=1\right]|$ $\leq
    \mathsf{Adv}_{\mathcal{U}_1,\Gr}(\advB).$
\label{lemma:bits2}
\end{lemma}
\begin{proof}
%The simulator starts by picking a random $\matr{B} \leftarrow \distlin{2}-\mddh$,  $\matr{B} \in \Z_q^{3 \times 2}$. 
The adversary $\advB$ receives an instance of the $\mathcal{U}_1$-$\mddh_\Gr$ problem, which is a pair
$(\hvecb{s}, \hvecb{t})$, where $\hvecb{s}$ is a uniform vector 
of $\Gr^{2}$ and $\hvecb{t}$ is either a uniform vector in $\Gr^2$ or 
$\hvecb{t}=\gamma\hvecb{s}$, for random $\gamma\in\Z_q$.
 
Adversary $\advB$ defines
$\lck_{m+1}:= \hvecb{s}$ and the rest of the columns of $\Lck$ are honestly sampled
with the sole exception of $\lck_{i^*}$, which is set to $\hvecb{t}$.

Given that adversary $\advB$ can only compute $\lck_{i}\lrck_j^\top\in\Gr^{2\times2}$,
it defines $\hmatr{C}_{i,j} := \lck_{i}\lrck_j^\top - \hmatr{T}_{i,j}$ and
$\cmatr{D}_{i,j}:=\cmatr{T}_{i,j}$, for $\matr{T}_{i,j}\gets\Z_q^{2\times 2}$ and $(i,j)\in\indexSet{m}{1}$. Note 
that this does not change the distribution of $(\cmatr{D}_{i,j},\hmatr{C}_{i,j})$, which is the uniform one conditioned
on $\matr{C}_{i,j}+\matr{D}_{i,j}= \llck_i\lrck_j^\top.$ The rest of the parameters, $\crs_\QANIZKcomms$ and $\crs_\QANIZKsum$, are generated honestly.

It is immediate to see that adversary $\advB$ perfectly simulates $\sfGame_1$ when $\hvecb{t}=\gamma\cvecb{s}$ and $\sfGame_2$ when $\hvecb{t}$ is uniform.  
\end{proof}

\begin{lemma} There exists a $\mathcal{U}_1$-$\mddh_\Hr$ adversary $\advB$ such that
$|\Pr\left[\mathsf{Game}_{2}(\advA)=1\right]$ $-\Pr\left[\mathsf{Game}_{3}(\advA)=1\right]|$ $\leq
\mathsf{Adv}_{\mathcal{U}_1,\Hr}(\advB).$
\label{lemma:bits3}
\end{lemma}

\begin{proof}
%The simulator starts by picking a random $\matr{B} \leftarrow \distlin{2}-\mddh$,  $\matr{B} \in \Z_q^{3 \times 2}$. 
The adversary $\advB$ receives an instance of the $\mathcal{U}_1$-$\mddh_\Hr$ problem, which is a pair
$(\cvecb{s}, \cvecb{t})$, where $\cvecb{s}$ is a uniform vector 
of $\Hr^{2}$ and $\cvecb{t}$ is either a uniform vector in $\Hr^2$ or 
$\cvecb{t}=\gamma\cvecb{s}$, for random $\gamma\in\Z_q$.      
 
Adversary $\advB$ defines 
$\rck_{m+1}:= \cvecb{s}$ and the rest of the columns of $\Rck$ are honestly sampled
with the sole exception of $\rck_{i^*}$, which is set to $\cvecb{t}$.

Given that adversary $\advB$ can only compute $\llck_{i}\rck_j^\top\in\Hr^{2\times2}$,
it defines $\cmatr{D}_{i,j} := \llck_{i}\rck_j^\top - \cmatr{T}_{i,j}$ and
$\hmatr{C}_{i,j}:=\hmatr{T}_{i,j}$, for $\matr{T}_{i,j}\gets\Z_q^{2\times 2}$ and $(i,j)\in\indexSet{m}{1}$. Note 
that this does not change the distribution of $(\cmatr{D}_{i,j},\hmatr{C}_{i,j})$, which is the uniform one conditioned
on $\matr{C}_{i,j}+\matr{D}_{i,j}= \llck_i\lrck_j^\top.$ The rest of the parameters, $\crs_\QANIZKcomms$ and $\crs_\QANIZKsum$,
are generated honestly.

It is immediate to see that adversary $\advB$ perfectly simulates $\sfGame_2$ when $\cvecb{t}=\gamma\cvecb{s}$ and $\sfGame_3$ when $\cvecb{t}$ is uniform.  
\end{proof}

\item By (i), with probability at least $1-4/q$, $\llck_{i^*}\lrck_{i^*}^\top$ is linearly independent from
$\{\llck_{i}\lrck_j^\top:(i,j)\in[m+1]^2\setminus\{(i^*, i^*)\}\}$ which implies that $\llck_{i^*}\lrck_{i^*}^\top\notin\Span(\{\matr{C}_{i,j}+\matr{D}_{i,j}:(i,j)\in\indexSet{m}{1}\}\})$. Since $\{\lck_{i^*},\lck_{m+1}\}$ and $\{\rck_{i^*},\rck_{m+1}\}$ are basis of $\Gr^2$ and $\Hr^2$, respectively,
we can define $b_{i^*},\overline{w}_g,\overline{w}_h,\overline{b}_{i^*}$ as the unique coefficients in $\Z_q$ such that $\hgrkb{\alpha}=b_{i^*}\lck_{i^*} + \overline{w}_g \lck_{m+1}$ and $\cgrkb{\beta}= \bb_{i^*} \rck_{i^*} + \overline{w}_h \rck_{m+1}$.

Additionally  $\sfGame_3(\advA)=1$ implies either $b_{i^*}\neq\bb_{i^*}$ or $b_{i^*} \notin \{0,1\}$,
while the verifier accepts the proof
$ (
        \hgrkb{\alpha}, \cgrkb{\beta},
        (\hmatr{\Theta}_{b(\overline{b}-1)}, \cmatr{\Pi}_{b(\overline{b}-1)}), 
        \{(\hat{\boldsymbol \rho}_{X}, \check{\boldsymbol \sigma}_{X})\}_{X \in \{b(\overline{b}-1), b-\overline{b}\}} )$
  produced by $\advA$.
We distinguish two cases:
\begin{itemize}
\item[1)] If $b_{i^*} \neq \overline{b}_{i^*}$. Given that $(b_{i}\llck_{i^*},\bar{b}_{i^*}\lrck_{i^*})$ is linearly independent from $\{(\llck_{i^*},\lrck_{i^*}),(\llck_{m+1},\lrck_{m+1})\}$
whenever $b_{i^*}\neq\bar{b}_{i^*}$, an adversary
$\advSound_2$ against $\QANIZKcomms$
outputs the pair $(\hgrkb{\rho}_{b-\overline{b}},
\cgrkb{\sigma}_{b-\overline{b}})
$ which is a fake proof for 
$(\hgrkb{\alpha},\cgrkb{\beta})$.
\item[2)] If $b_{i^*} = \overline{b}_{i^*}$ but 
$b_{i^*}(\overline{b}_{i^*} -1) \neq 0$.
If we express $\matr{\Theta}_{b(\overline{b}-1)}+\matr{\Pi}_{b(\overline{b}-1)}$
as a linear combination of $\{\llck_{i}\lrck_{j}^{\top}:i,j\in[m+1]\}$, the coordinate of
$\llck_{i^*}\lrck_{i^*}^\top$ is $b_{i^*}(\overline{b}_{i^*}-1)\neq 0$ and thus $\matr{\Theta}_{b(\overline{b}-1)}+\matr{\Pi}_{b(\overline{b}-1)}\notin\Span(\{\matr{C}_{i,j}+\matr{D}_{i,j}:(i,j)\in\indexSet{m}{1}\})$.
The adversary $\advSound_1$ against $\QANIZKsum$  outputs the pair
$(\hgrkb{\rho}_{b(\overline{b}-1)},$ $\cgrkb{\sigma}_{b(\overline{b}-1)})$
which is a fake proof for $(\hmatr{\Theta}_{b(\overline{b}-1)}, \cmatr{\Pi}_{b(\overline{b}-1)})$.

{\color{red} HAY QUE DECIR QUE \cite[Theorem~2]{EC:KilWee15} A\'UN ES V\'ALIDO SI EL ADV CONOCE LOS LOGARITMOS DISCRETOS DE $\matr{M}$}
\end{itemize}

\end{enumerate}
\end{proof}

