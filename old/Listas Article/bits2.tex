\begin{description}


\item[{$\algK_1(\Gamma, \bmatr{G}_1,\matr{H})$}:]
Pick $\matr{T}\gets\Z_q^{2\times 2}$ andi for each $(i,j)\in\indexSet{n}{1}$ define matrices
$$([\matr{C}_{i,j}]_1,[\matr{D}_{i,j}]_2):=([\vecb{g}_i]_1\vecb{h}_j^\top+[\matr{T}]_1,-[\matr{T}]_2).$$
Let $\Phi$ be the proof system for Sum in Subspace 
(Sect. \ref{sec:XXX}) and $\Psi$
be an instance of the proof system for Equal Commitment Opening (Sect. \ref{sec:XXX}).

Let
$\crs_\Phi \gets \Phi.\algK_1(\Gamma, \{[\matr{C}_{i,j}]_1,[\matr{D}_{i,j}]_2\}_{(i,j)\in\indexSet{n}{1}})$.\footnote{We identify
matrices in $\Gr^{2 \times 2}$ (resp. in $\Hr^{2 \times 2}$) with vectors in $\Gr^{4}$ (resp. in $\Hr^{4}$).} and let $\crs_\Psi \gets \Psi.\algK_1(\Gamma, \bmatr{G}_1,\bmatr{H}_2,n)$. 

The common reference string is given by:
\begin{eqnarray*}
\mathsf{crs}&:=&\left(  \bmatr{G}_1,
    [\matr{H}]_2, \{[\matr{C}_{i,j}]_1, [\matr{D}_{i,j}]_2 \}_{(i,j) \in \indexSet{n}{1}},\crs_\Phi,\crs_\Psi \right).
 \end{eqnarray*}
\item[$\algP(\mathsf{crs}_, \bvecb{c}_1, \langle \vecb{b}, w_g \rangle)$:]
Pick $w_h \gets \Z_q$,  $\matr{R} \gets \Z_q^{2\times 2}$ and then: 
\begin{enumerate}
\item Define 
$$[\vecb{d}]_2:= \mathsf{MP}.\Com_{[\matr{H}]_2} (\vecb{b};w_h) =  [\matr{H}]_2\smallpmatrix{\vecb{b}\\w_h}.$$ 
\item Compute 
\begin{eqnarray*}
([\matr{\Theta}]_1, [\matr{\Pi}]_2) & := &
\sum_{i\in[n]}\sum_{j\in[n]}b_i(b_i-1)([\matr{C}_{i,j}]_1,[\matr{D}_{i,j}]_2)+ w_gw_h([\matr{C}_{n+1,n+1}]_1,[\matr{D}_{n+1,n+1}]_2) \\
& &
 \sum_{i \in [n]}
 b_i w_h  ([\matr{C}_{i,n+1}]_1,[\matr{D}_{i,n+1}]_2) +
 w_g(b_i-1) ([\matr{C}_{n+1,i}]_1,[\matr{D}_{n+1,i}]_2).
\end{eqnarray*}

\item Compute a proof $([\grkb{\rho}_1]_1,[\grkb{\sigma}_1]_2)$
that $\matr{\Theta}+\cmatr{\Pi}$ is in the span of 
$\{\matr{C}_{i,j}+\matr{D}_{i,j}\}_{(i,j)\in\indexSet{n}{1}}$.
and a proof $([\grkb{\rho}_{2}]_1, [\grkb{\sigma}_{2}]_2)$
that
$([\vecb{c}]_1,[\vecb{d}]_2)$ open to the same value,
using $\vecb{b},w_g$, and $w_h$. \\
\end{enumerate}

\item[{$\algV(
    \mathsf{crs},
    [\vecb{c}]_1,
    [\vecb{d}]_2,
        ([\matr{\Theta}]_1, [\matr{\Pi}]_2), 
        \{([\grkb{\rho}_{i}]_1, [\grkb{\sigma}_{i}]_2)\}_{i \in [2]} )$}:] ~
%   
\begin{enumerate}
\item Check if 
\begin{equation}\label{eq:ver1}
e([\vecb{c}]_1,[\vecb{d}]_2^\top-
    \sum_{j \in [n]} [\vecb{h}_j]_2^\top) =
    e([\matr{\Theta}]_1,[\matr{I}_{2 \times 2}]) e([\matr{I}_{2 \times 2}]_1,[\matr{\Pi}]_2).
    \end{equation}  
  \item Verify that $([\grkb{\rho}_1]_1, [\grkb{\sigma}_1]_2),([\grkb{\rho}_{2}]_1,[\grkb{\sigma}_2]_2)$ are valid proofs for %\linebreak 
  $([\matr{\Theta}]_1,$ $[\matr{\Pi}]_2)$
        and $([\vecb{c}]_1,[\vecb{d}]_2)$ using $\crs_\Phi$ and $\crs_\Psi$ respectively.
\end{enumerate}
If any of these checks fails, the verifier outputs $0$, else it outputs $1$. \\
\end{description}

The simulators $\algS_1$ and $\algS_2$ are defined as follows.
\begin{description}
\item[{$\mathsf{S}_1(\Gamma,[\matr{G}]_1,\matr{H})$}:] It generates and outputs the CRS in the same way as $\algK_1$, but additionally it also  outputs the simulation trapdoor 
$$\tau=\left(\matr{H},\tau_\Phi, \tau_\Psi\right),$$
where $\tau_\Phi$ and $\tau_\Psi$ are, respectively, $\Phi$'s and $\Psi$'s simulation trapdoors.
\item[{$\mathsf{S}_2(\crs,[\vecb{c}]_1,\left(\matr{H},\tau_\Phi, \tau_\Psi\right))$}:]  Given the matrix $\matr{H}$ of discrete logarithms of $[\matr{H}]_2$, $\tau_\Phi$ and $\tau_\Psi$ which are, respectively, $\Phi$'s and $\Psi$'s simulation trapdoors, this algorithm samples 
$\overline{w}_h \gets \Z_q$, $\matr{R} \gets \Z_q^{2 \times 2}$ and defines 
 $\vecb{d}:= \overline{w}_{h} \vecb{h}_{n+1}.$
 Then it sets:
\begin{align*} 
[\matr{\Theta}]_1 & :=  [\vecb{c}]_1\left(\vecb{d}-\sum_{i \in [n]} \lrck_i\right)^\top + [\matr{R}]_1,
    &
[\matr{\Pi}]_2 & := - [\matr{R}]_2.
\end{align*}
Finally, it simulates proofs $([\grkb{\rho}_{i}]_1, [\grkb{\sigma}_{i}]_2)$
for $i \in \{1,2\}$  using $\tau_\Phi$ and $\tau_\Psi$.
\end{description}

\begin{theorem} Algorithms $(\algK,\algP,\algV,\algS_2)$ satisfy that
\begin{enumerate}[label=\alph*)]
 \item Perfect Completeness: If $[\vecb{c}]_1 = [\matr{G}]_1\smallpmatrix{\vecb{b}\\w_g}$, $\vecb{b}\in\bits^n$, and $\mathsf{proof}\gets\algP(\crs,[\vecb{c}]_1,\langle\vecb{b},w_g\rangle)$, then $\algV(\crs,[\vecb{c}],\mathsf{proof})=1$.
\item $1$-coordinate Soundness: If $\vecb{g}_{n+1}\neq\vecb{0}$, $\vecb{h}_{n+1}\neq\vecb{0}$, and there exists an index $i^*$ such that $\vecb{g}_{i^*} \notin \mathsf{Span}(\{\vecb{g}_{i} : i \neq i^*\})$ and $\vecb{h}_{i^*} \notin \mathsf{Span}(\{\vecb{h}_{i} : i \neq i^*\})$, then:
\begin{multline*}
\Pr\left[
    \begin{array}{l}
        \crs\gets\algK(\Gamma,[\matr{G}]_1,[\matr{H}]_2);\\
        ([\vecb{c}]_1,\mathsf{proof}) \gets \advA(\mathsf{crs},\matr{H})
    \end{array}:
    \begin{array}{l}
    \exists b_{i^*}\in\Z_q, \vecb{w}\in\Z_q^n\text{ s.t. } b_{i^*} \notin \{0,1\} \wedge\\
        \hvecb{c}= b_{i^*} \hvecb{g}_{i^*} + \sum_{j \in [n+1], j\neq i^*} w_{j} \hvecb{g}_{j}  \wedge
        \algV(\mathsf{crs},[\vecb{c}]_1,\mathsf{proof})=1 
    \end{array}\right]\leq\negl(\lambda).$$
\end{multline*}
Note that soundeness is guaranteed even when $\advA$ receives the discrete lograrithms of $[\matr{H}]_2$.
\item Perfect Zero-Knowledge: If $\rank(\matr{H})=1$ and $\vecb{h}_{n+1}\neq\vecb{0}$, then for every PPT adversary $\advA$.
\begin{align*}
&\Pr[
    \crs \gets \algK_1(\Gamma,[\matr{G}]_1,\matr{H}):
        \advA^{\algP(\crs,\cdot, \cdot)}(\Gamma,\crs) = 1]
=\\
&\Pr[
    (\crs,\tau) \gets \algS_1(\Gamma, [\matr{G}]_1,\matr{H}):
        \advA^{\algS(\crs,\tau,\cdot,\cdot)}(\Gamma,\crs)=1]
\end{align*}
where
\begin{itemize}
\item $\algP(\crs, \cdot, \cdot)$ emulates the actual prover. It takes input $([\vecb{c}]_1,\langle\vecb{b},w_g\rangle)$ and outputs a 
proof $\mathsf{proof}\gets\algP(\crs,[\vecb{c}]_1,\langle\vecb{b},w_g\rangle)$, if $[\vecb{c}]_1=[\matr{G}]_1\smallpmatrix{\vecb{b}\\w_g}$ for some $\vecb{b}\in\bits^n$ and $w_g\in\Z_q$. Otherwise, it outputs $\perp$.
\item $\algS(\crs,\tau, \cdot, \cdot)$ is an oracle that takes input $([\vecb{c}]_1,\langle\vecb{b},w_g\rangle)$. It outputs a simulated proof
$\mathsf{proof}\gets\algS_2(\crs,\tau, [\vecb{c}]_1)$, if $[\vecb{c}]_1=[\matr{G}]_1\smallpmatrix{\vecb{b}\\w_g}$ for some $\vecb{b}\in\bits^n$ and $w_g\in\Z_q$. Otherwise, it outputs $\perp$.
\end{itemize}

\end{enumerate}
\end{theorem}	
\begin{proof}
\begin{description}
\item[Perfect Completeness:]
Note that,
by definition of $\matr{C}_{i,j}$ and $\matr{D}_{i,j}$, 
$e([\matr{C}_{i,j}]_1,[\matr{I}_{2\times2}]_2)\cdot e([\matr{I}_{2\times2}]_1,[\matr{D}_{i,j}]_2)$
$= e([\vecb{g}_{i}]_1,[\vecb{h}_j]_2^\top)$.  Since $b_i(b_i-1) = 0$ for each $i\in[n]$,
\begin{eqnarray*}
\lefteqn{
e\left([\vecb{c}]_1, [\vecb{d}]_2^\top - \sum_{i\in[n]} [\vecb{h}_{i}]_2^\top \right)}\\
& = & 
    \prod_{i \in [n]}\left(
        e([\vecb{g}_{i}]_1,[\vecb{h}_{n+1}]_2^{\top})^{b_i w_h}\cdot
        e([\vecb{g}_{n+1}]_1,[\vecb{h}_i]_2^{\top})^{w_g(b_i-1)}\cdot
        \prod_{j \in [n]} e([\vecb{g}_{i}]_1,[\vecb{h}_{j}]_2^{\top})^{b_i (b_j-1)}
    \right) 
\\ & & \mbox{ }
    \cdot e([\vecb{g}_{n+1}]_1,[\vecb{h}_{n+1}]_2^{\top})^{w_gw_h}\\
& = & 
    e\left(\sum_{i\in[n]}\left(b_iw_h[\vecb{g}_{i}]_1\vecb{h}_{n+1}^{\top}+w_g(b_i-1)[\vecb{g}_{n+1}]_1\vecb{h}_i^{\top}+
        \sum_{\substack{j \in [n]\\j\neq i}}b_i(b_i-1)[\vecb{g}_{i}]_1\vecb{h}_{j}^{\top}\right),[\matr{I}_{2\times 2}]_2
    \right)
\\ & & \mbox{ }
    \cdot e(w_gw_h[\vecb{g}_{n+1}]_1\vecb{h}_{n+1}^{\top},[\matr{I}_{2\times 2}]_2)
    \cdot e([\matr{R}]_1,[\matr{I}_{2\times2}]_2)/e([\matr{I}_{2\times2}]_1,[\matr{R}]_2)
\\ & = &
    e([\matr{\Theta}]_1,[\matr{I}_{2\times2}]_2)\cdot
    e([\matr{I}_{2\times2}]_1,[\matr{\Pi}]_2).
\end{eqnarray*}

Finally, the rest of the proof follows from completeness of $\Phi$ and $\Psi$. 

\item[1-coordinate Soundness:]
Since $\{\vecb{g}_{i^*},\vecb{g}_{n+1}\}$ and $\{\vecb{h}_{i^*},\vecb{h}_{n+1}\}$ are both basis of $\Z_q^2$,
we can define $b_{i^*},\overline{w}_g,\overline{w}_h,\overline{b}_{i^*}$ as the unique coefficients in $\Z_q$ such that $\vecb{c}=b_{i^*}\vecb{g}_{i^*} + \overline{w}_g \vecb{g}_{n+1}$ and $\vecb{d}= \bb_{i^*} \vecb{h}_{i^*} + \overline{w}_h \vecb{h}_{n+1}$.

Additionally, If $\advA$ breaks $1$-coordinate soudness implies that $b_{i^*} \notin \{0,1\}$,
while the verifier accepts the proof
$ (
        [\vecb{d}]_2,
        ([\matr{\Theta}]_1, [\matr{\Pi}]_2), 
        \{([\grkb{\rho}_{i}]_1, [\grkb{\sigma}_{i}]_2)\}_{i \in [2]} )$
  produced by $\advA$.
We distinguish two cases:
\begin{itemize}
\item[1)] If $b_{i^*} \neq \overline{b}_{i^*}$. Given that $(b_{i}\vecb{g}_{i^*},\bar{b}_{i^*}\vecb{h}_{i^*})$ is linearly independent from $\{(\vecb{g}_{i^*},\vecb{h}_{i^*}),(\vecb{g}_{n+1},\vecb{h}_{n+1})\}$
whenever $b_{i^*}\neq\bb_{i^*}$, an adversary
$\advSound_2$ against $\Phi$
outputs the pair $([\grkb{\rho}_{2}]_1,[\grkb{\sigma}_{2}]_2)
$ which is a fake proof for 
$([\vecb{c}]_1,[\vecb{d}]_2)$.
{\color{red} HAY QUE DECIR QUE NUESTRA PRUEBA DE MEMBERSHIP EN $\GG_1^m\times\GG_2^n$ A\'UN ES V\'ALIDA SI EL ADV CONOCE LOS LOGARITMOS DISCRETOS DE $\matr{N}$ (QUE ES LO QUE CORRESPONDE A QUE AC\'A CONOZCA $\matr{H}$). ESTO DE ALGUNA FORMA TAMBI\'EN PASA EN LA PRUEBA DE LOS BITS, PUES EN LA REDUCCI\'ON HAY QUE CALCULAR, DADO $[\vecb{a}]_2$, $[\vecb{a}_\Delta]_2=\matr{\Delta}^\top[\vecb{a}]_2$ $\Rightarrow$ CONOCER $\matr{G}=\matr{\Delta}\matr{U}$.}
\item[2)] If $b_{i^*} = \overline{b}_{i^*}$ but 
$b_{i^*}(\overline{b}_{i^*} -1) \neq 0$.
If we express $\matr{\Theta}+\matr{\Pi}$
as a linear combination of $\{\vecb{g}_{i}\vecb{h}_{j}^{\top}:i,j\in[n+1]\}$, the coordinate of
$\vecb{g}_{i^*}\vecb{h}_{i^*}^\top$ is $b_{i^*}(\bb_{i^*}-1)\neq 0$ and thus $\matr{\Theta}+\matr{\Pi}\notin\Span(\{\matr{C}_{i,j}+\matr{D}_{i,j}:(i,j)\in\indexSet{n}{1}\})$.
The adversary $\advSound_1$ against $\Psi$  outputs the pair
$([\grkb{\rho}_1]_1,$ $[\grkb{\sigma}_1]_2)$
which is a fake proof for $([\matr{\Theta}]_1, [\matr{\Pi}]_2)$.
{\color{red} NO SE SI AC\'A PASA LO ANTERIOR.}
\end{itemize}
\item[Perfect Zero-Knowledge:] First, note that the vector $[\vecb{d}]_2 \in \GG_2^2$ output by the prover and the vector output by $\algS_2$ follow exactly the same distribution. This is because the rank of $\matr{H}$ is $1$ and $\vecb{h}_{n+1}\neq \vecb{0}$. In particular, although the simulator $\algS_2$ does not know $\vecb{b} \in \{0,1\}^{n}$ such that $[\vecb{c}]_1=[\matr{G}]_1\smallpmatrix{\vecb{b}\\w_g}$, for some $w_g\in\Z_q$, 
there exists $w_h \in \Z_q$ such that $[\vecb{d}]_2=[\matr{H}]_2\smallpmatrix{\vecb{b}\\ w_h}$. 

Since $\matr{R}$ is chosen uniformly at random in $\Z_q^{2 \times 2}$, the proof $([\matr{\Theta}]_1, [\matr{\Pi}]_2)$ is uniformly distributed conditioned on satisfying check 1) of algorithm $\algV$.
 Finally, the rest of the proof follows from Zero-Knowledge of $\Phi$ and $\Psi$.
\end{description}
\end{proof}
