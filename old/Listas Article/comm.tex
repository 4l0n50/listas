\section{Commitment schemes}

%We will be using two kinds of commitment schemes: Groth-Sahai commitments \cite{EC:GroSah08} and 

\begin{definition} A commitment scheme is a tuple of three algorithms 
	$(\algK,\Com,\algVrfy)$ such that:
	\begin{itemize} 
		\item  $\algK$ is a randomized algorithm, which on input the security parameter 
		$1^{\lambda}$ outputs a commitment key $ck$,
		\item $\Com$ is a randomized algorithm which, on input the commitment key $ck$ and a message 
		$m$ in the message space $\mathcal{M}_{ck}$ outputs a commitment $c$ in the commitment space $\mathcal{C}_{ck}$ and an opening $Op$, 
	\item $\algVrfy$ is a deterministic algorithm which, on input the commitment key $ck$,  a message 
	$m$ in the message space $\mathcal{M}_{ck}$ and an opening $Op$, outputs $1$ if $Op$ is a valid opening of $c$ to the message $m$ and $0$ otherwise. 
	\end{itemize}
\end{definition}

Correctness requires that for any $m \in \mathcal{M}_{ck}$
$$\Pr\left[ck \gets \algK(1^{\lambda}); m \gets \mathcal{M}_{ck}; 
(c, Op) \gets \Com(ck,m): \algVrfy(ck,c,m,Op)=1 \right]=1.$$


\begin{definition}  A commitment scheme is binding if, for any polynomial-time adversary $\advA$, 
	$$\Pr\left[ck \gets \algK(1^{\lambda}); (c,m,Op,m',Op') \gets \advA(ck): \algVrfy(ck,c,m,Op)=1 \cap \algVrfy(ck,c,m',Op')=1\right] $$
	is negligible.  It is hiding if, for any polynomial-time adversary $\advA$,
	$$\left|\Pr\left[ \begin{array}{l} ck \gets \algK(1^{\lambda}); (m_0,m_1,st) \gets \advA(ck); b \gets \{0,1\};\\ (c,Op) \gets \Com(ck,m_b); b' \gets \advA(st,c)
                      \end{array} : b'=b\right] -\dfrac{1}{2} \right|$$
	is negligible.
\end{definition}



%\begin{definition} A commitment scheme is trapdoor if  $\mathsf{K}$ additionally outputs a trapdoor key $tk$ and there is an efficient algorithm $\mathsf{TrapdoorEquiv}$ which, on input $(ck,tk,c,m,Op,m')$ outputs $Op'$ such that $1 \gets \mathsf{Vrfy}(ck,c,m',Op')$. Further, for any pair of valid messages $m$, $m'$ and legitimately generated 
%	$ck,tk$, it holds that the distributions $(ck,c,Op')$ when 
%	$(c,Op) \gets \mathsf{Comm}(ck,m)$, $Op' \gets \mathsf{TrapdoorEquiv}(ck,tk,c,m,Op,m')$  or when  
%	$(c,Op') \gets \mathsf{Comm}(ck,m')$ are indistinguishable.
%\end{definition}

In this paper we will be using two definitions of commitments, one is the GS commitment scheme, and the other is a generalization of the Multi-Pedersen commitment, which commits vectors of scalars as a single group element to a vector of two group elements. The advantage of considering such a commitment is that more information about the committed value can be extracted. 



\begin{definition} The 2-dimensional Multi-Pedersen commitment scheme in the group $\GG_1$ is specified by the following three algorithms 
	$\mathsf{MP}=(\mathsf{MP}.\algK,\mathsf{MP}.\Com,\mathsf{MP}.\algVrfy)$ such that:
	\begin{itemize} 
		\item  $\mathsf{MP}.\algK$ is a randomized algorithm, which on input the security parameter 
		$1^{\lambda}$ and a natural number $n \in \N$, outputs an asymmetric bilinear group, a group key $gk$, and a commitment key $ck=\bmatr{G}_1=[(\vecb{g}_1||\ldots||\vecb{g}_{n+1})]_1 \in \GG_1^{2 \times (n+1)}$, where $\matr{G}\gets\Z_q^{2\times(n+1)}$.
		\item $\mathsf{MP}.\Com$ is a randomized algorithm which, on input a group key $gk$ and a commitment key $ck=\bmatr{G}_1$ and a message 
		$\vecb{m} \in \Z_q^{n}$ in the message space $\mathcal{M}_{ck}=\Z_q^{n}$, it samples $r \gets \Z_q$ and outputs a commitment $\bvecb{c}_1:=\bmatr{G}_1\smallpmatrix{\vecb{m} \\ r}$ in the commitment space $\mathcal{C}_{ck}=\GG^2$ and an opening $Op=r$, 
		\item $\mathsf{MP}.\algVrfy$ is a deterministic algorithm which, on input the commitment key $ck=\bmatr{G}_1$, a commitment $\bvecb{c}_1$,  a message 
		$\vecb{m} \in \Z_q^{n}$ and an opening $Op=r$, outputs $1$ if $\bvecb{c}_1=\bmatr{G}_1\smallpmatrix{\vecb{m} \\ r }$
		and $0$ otherwise. 
	\end{itemize}
\end{definition}

%The following result ... 

\begin{theorem} $\mathsf{MP}$ is computationally hiding if $\ddh$ holds in $\GG_1$ and computationally binding if the Discrete Logarithm Assumption holds in $\GG_1$.
\end{theorem}


\begin{definition} The Groth-Sahai commitment scheme in the group $\GG_1$ is  specified by the following three algorithms 
	$(\mathsf{GS}.\algK,\mathsf{GS}.\Com,\mathsf{GS}.\algVrfy)$ such that:
	\begin{itemize} 
		\item  $\mathsf{GS}.\algK$ is a randomized algorithm, which on input the security parameter 
		$1^{\lambda}$, outputs an asymmetric bilinear group, a group key $gk$, and a commitment key $ck=\bmatr{U}_1 \in \GG^{2\times 2}$, where $\matr{U}\gets\Z_q^{2\times 2}$.
		\item $\mathsf{GS}.\Com$ is a randomized algorithm which, on input a group key $gk$, a commitment key $ck=\bmatr{U}_1$, and a message 
		$m \in \Z_q$ in the message space $\mathcal{M}_{ck}=\Z_q$, it samples $r \gets \Z_q$ and outputs a commitment $\bvecb{c}_1 := \bmatr{U}_1\smallpmatrix{m\\r}$ in the commitment space $\mathcal{C}_{ck}=\GG^2$ and an opening $Op=r$, 
		\item $\mathsf{MP}.\algVrfy$ is a deterministic algorithm which, on input the commitment key $ck=\bmatr{U}_1$, a commitment $\bvecb{c}_1$,  a message 
		$m \in \Z_q$ and an opening $Op=r$, outputs $1$ if $\bvecb{c}_1=\bmatr{U}_1\smallpmatrix{m \\ r}$
		and $0$ otherwise.
	\end{itemize}
\end{definition}

\begin{theorem} $\mathsf{GS}.\mathsf{Com}$ is perfectly binding and computationally hiding if the $\ddh$ Assumption holds in $\GG_1$.
\end{theorem}
