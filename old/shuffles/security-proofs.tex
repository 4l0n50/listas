\subsubsection{Completeness.} If $\matr{P}$ is a permutation matrix then $\vvar{X}=[\matr{S}]_1\matr{P}^\top=([s_{\pi(1)}]_1,\ldots,[s_{\pi_{n}}]_1)$, for some permutation $\pi$, which implies that $\GS.\Com(\vvar{X})\in\Lang_{S,\vecb{u}_1,\vecb{u}_2}^n$ and $\sum_{i\in[n]}[s_i]_1-\sum_{j\in[n]}\vvar{x}_j=[0]_1$. If $\vary=[\matr{S}]_1\grkb{\delta}$ then
\begin{eqnarray*}
\dfrac
    {e(\vvar{X},[\matr{C}]_2^\top)}
    {e([\matr{S}]_1,[\matr{D}]_2^\top)}
 = 
    e([\matr{S}]_1,([\matr{C}]_2\matr{P}-[\matr{D}]_2)^\top)
 = 
    e([\matr{S}]_1,([\vecb{v}]_2\grkb{\delta}^\top)^\top)
 =
    e(\vary,[\vecb{v}]_2^\top).
\end{eqnarray*}

\subsubsection{Soundness.}
\begin{theorem}
For any adversary $\advA$ against the soundness of the proof system from Figure \ref{fig:shuffles}, there exist an adversary $\advB_1$ against soundness of $\Pi_{\sfbin}$ and an adversary $\advB_2$ against the $\dist_{1,n}\mbox{-}\kermdh$ Assumption such that
$$\adv(\advA) \leq \adv_{\Pi_{\sfbin}}(\advB_1) + \adv_{\dist_{1,n}\mbox{-}\kermdh}(\advB_2).
$$
\end{theorem}

\begin{proof}
The adversary $\advB_1$ receives as input $\crs_\sfbin$ and honestly samples the rest of the CRS. Then $\advB_1$ runs $\advA$ until it halts and outputs $[\matr{F}]_1$ with the proof $\pi_\sfbin$.

The adversary $\advB_2$ receives as input $[\matr{S}]_1\in\GG_1^{1\times n}$, samples $\vecb{u}_1,\vecb{u}_2,\vecb{v}\gets \Z_q^2$, honestly simulates the rest of the CRS, and runs $\advA$ until it halts. It extracts $[\matr{X}]_1$, the opening of $[\matr{F}]_1$, using $\vecb{u}_1,\vecb{u}_2$, and aborts if $[\matr{F}]_1\notin\Lang_{S,\vecb{u}_1,\vecb{u}_2}^n$. Else $[\matr{X}]_1=[\matr{S}]_1\matr{B}$, where $\matr{B}\in\bits^{n\times n}$ and $\matr{1}_{1\times n}\matr{B}=\matr{1}_{1\times n}$. If $\sum_{i\in[n]}[s_i]_1-\sum_{j\in[n]}[x_j]_1\neq[0]_1$, $\advB$ outputs $\matr{1}_{n\times 1}-\matr{B}\matr{1}_{n\times 1}$. Else, using $\vecb{v}$, $\advB_2$ decrypts $[\matr{C}]_2$ and $[\matr{D}]_2$ obtaining $[\vecb{M}]_2\in\GG_2^{1\times n}$ and $[\vecb{M}']_2\in\GG_2^{1\times n}$, respectively, and returns $[\vecb{M}]_2\matr{B}^\top-[\vecb{M}']_2$.

Let $E_1$ the event where $([\matr{C}]_2,[\matr{D}]_2)\in\Lang_{[\vecb{v}]_2,n,\mathsf{shuffle}}$, $E_2$ the event where $[\matr{F}]_1\in\Lang_{S,\vecb{u}_1,\vecb{u}_2}^n$, and $E_3$ the event where $\matr{1}_{n\times 1}-\matr{B}\matr{1}_{n\times 1}=\vecb{0}_{n\times 1}\wedge[\vecb{M}]_2\matr{B}-[\vecb{M}']_2=0$. Note that $E_2\wedge E_3\Longrightarrow E_1$, because $E_2$ implies that $[\matr{X}]_1=[\matr{S}]_1\matr{B}$, where $\matr{B}\in\bits^{n\times n}$ and $\matr{1}_{1\times n}\matr{B}=\matr{1}_{1\times n}$. Together with $E_3$ implies that $\matr{B}$ is a permutation and thus $[\matr{C}]_2\matr{B}^\top-[\matr{D}]_2=\Enc([\matr{0}_{1\times n}]_2)\Longleftrightarrow E_1$. Note also that $\eq_1\wedge\eq_2\wedge\neg E_3$ implies that $\matr{1}_{n\times 1}-\matr{B}\matr{1}_{n\times 1}$ or $[\vecb{M}]_2\matr{B}-[\vecb{M}']_2$ are solutions to the $\dist_{1,n}\mbox{-}\kermdh$. Then it holds that
\begin{eqnarray*}
\adv(\advA) & = &
    \Pr[\neg E_1 \wedge \algV(\crs,([\matr{C}]_2,[\matr{D}]_2),\pi)=1]\\
& = &
    \Pr[\neg E_1 \wedge \algV(\crs,([\matr{C}]_2,[\matr{D}]_2),\pi)=1 \wedge \neg E_2]+\\
& &
    \Pr[\neg E_1 \wedge \algV(\crs,([\matr{C}]_2,[\matr{D}]_2),\pi)=1 \wedge E_2]\\
& \leq &
    \Pr[\neg E_2 \wedge \Pi_\sfbin.\algV(\crs_\sfbin,[\matr{F}]_2,\pi_\sfbin)=1]+
    \Pr[\neg E_1 \wedge \eq_1 \wedge \eq_2 \wedge E_2]\\
& \leq &
    \adv_{\Pi_\sfbin}(\advB_1)+\Pr[\neg E_1 \wedge \eq_1 \wedge \eq_2 \wedge E_2\wedge E_3] + \Pr[\neg E_1 \wedge \eq_1 \wedge \eq_2 \wedge E_2\wedge \neg E_3]\\
& \leq &
    \adv_{\Pi_\sfbin}(\advB_1) + \Pr[\neg E_1\wedge E_2 \wedge E_3]+
    \Pr[\eq_1\wedge\eq_2\wedge\neg E_3]\\
& = &
    \adv_{\Pi_\sfbin}(\advB_1)+0+\adv_{\dist_{1,n}\mbox{-}\kermdh}(\advB_2).
\end{eqnarray*} 
\end{proof}

\subsubsection{Zero-Knowledge}. We need to check that the inputs to the simulators are true statements and, for the GS simulator, that the equations allow simulation. This is certainly true for $\mathsf{eq}_1$ and, if $([\matr{C}]_1,[\matr{D}]_1)\in\Lang_{[\vecb{u}]_1,n,\mathsf{shuffle}}$, then is also true for $\mathsf{eq}'_2$. Furthermore, it is guaranteed that $\pi_\GS$ is computationally indistinguishable from a real proof for $\{\eq_1,\eq'_2\}$, which is identically distributed to a real proof for $\{\eq_1,\eq_2\}$ since $\eq'_2$ and $\eq_2$ accepts the same set of solutions.
Finally, since the perfectly hiding $\crs_\GS$ is such that $\rank(\vecb{u}_1||\vecb{u}_2)=2$, then $\Lang_{S,\vecb{u}_1,\vecb{u}_2}^n=\GG_1^{2\times n}$ and thus $[\matr{F}]_1\in\Lang_{S,\vecb{u}_1,\vecb{u}_2}^n$ is always true.
