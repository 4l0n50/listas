We motivate our construction with the following fact about permutation matrices

\begin{fact} The matrix $\matr{P}\in\Z_q^{n\times n}$ is a permutation matrix if and only if
\begin{enumerate}
    \item $\matr{P}\in\bits^{n\times n}$.
    \item $\matr{1}_{1\times n}\matr{P}=\matr{1}_{1\times n}$.
    \item $\matr{P}\matr{1}_{n\times 1}= \matr{1}_{n\times 1}$.
\end{enumerate}
\end{fact}

This fact motivates the following trivial solution. Compute coordinate-wise GS commitment to $\matr{P}$, satisfiability of equations implied by 1,2 and 3, imply that $\matr{P}$ is permutation matrix. However, only commitments to $\matr{P}$ requires $\Theta(n^2)$ group elements. Below we describe an alternative way to commit to $\matr{P}$ using only $\Theta(n)$ group elements.

Given GS commitment key $[\matr{V}]_2\in\GG_2^{2\times 2}$ and some list $L$ of size $n$ to be defined later, the proof system from Sect. \ref{sec:list-memb} allows to prove that a matrix $[\Lcom]_2\in\GG_2^{2\times n}\in\Lang_{[\matr{V}]_2,L,n}$. That is, there exists sombre matrix $\matr{R}\in\Z_q^{2\times n},\matr{B}\in\bits^{n\times n}$ (i.e. $m:=n$) such that $\matr{1}_{1\times n}\matr{B}=\matr{1}_{1\times n}$ and $[\Lcom]_2 = [\matr{L}]_2\matr{B}+[\matr{V}]_2\matr{R}$. If we think of $[\Lcom]_2$ as a commitment to $\matr{P}:=\matr{B}$, the proof system from Sect. \ref{sec:list-memb} allows to prove statements 1 and 2. The natural way to prove statement 3 should be
to prove that $
[\Lcom]_2\matr{1}_{n\times 1}-[\matr{L}]_2\matr{1}_{n\times 1} = [\matr{V}]_2\grkb{\gamma}
$, for some $\grkb{\gamma}\in\Z_q^2$, which implies that $[\matr{L}]_2(\matr{P}\matr{1}_{n\times 1}-\matr{1}_{n\times 1})=0$. If we define $L:=\{[a_1]_2,\ldots,[a_n]_2\}$, where $\vecb{a}\gets \dist_{n,1}$ and the $\dist_{n,1}\mbox{-}\kermdh$ problem is hard, then $\matr{P}\matr{1}_{n\times 1}=\matr{1}_{n\times 1}$, unless one can find a solution to the $\dist_{n,1}\mbox{-}\kermdh$ Assumption We conclude that $\matr{P}$ is a permutation matrix.

Now, given $([\matr{C}]_1,[\matr{D}]_1)\in\GG^{2\times n}\times\GG_1^{2\times n}$ and $[\Lcom]_2$, which defines a GS commitment to $\varvecx:=[\vecb{a}]_2\matr{P}^\top$, we show with a GS proof the satisfiability of
\begin{equation}
\dfrac{e([\matr{C}]_1,\varvecx^\top)}{e([\matr{D}]_1,[\vecb{a}]_2^\top)}
=
e([\vecb{u}]_1,\mathsf{y}),
\label{eq:shuffle}
\end{equation}
fro some $\mathsf{y}\in\GG_2$.

We briefly comment why this imply soundness. Given $\vecb{u}$, one can compute $\vecb{k}\in\Z_q^2$ such that $\vecb{k}^\top[\matr{C}]_1=[\vecb{m}]_1\in\GG_1^{1\times n}$, and $\vecb{k}^\top[\matr{D}]_1=[\vecb{m}']_1\in\GG_1^{1\times n}$, the decryptions of $[\matr{C}]_1$ and $[\matr{D}]_1$, and $\vecb{k}^\top[\vecb{u}]_1=[\vecb{0}]_1$.\footnote{In the Quasi-Adaptive setting, we prove membership in the CRS dependent language $\Lang_{\hvecb{u},n,\mathsf{shuffle}}$. This allows us to decrypt the ciphertexts in the soundness game, using the discrete logarithm of $\hvecb{u}$. Similarly, Groth and Lu proved \emph{co-soundness}, which requires the adversary to provide the decryption key.} If we multipliy Eq. \ref{eq:shuffle} by $\vecb{k}^\top$ on the right we get
$$
e([\vecb{m}]_1\matr{P}^\top-[\vecb{m}']_1,[\vecb{a}]_2^\top)=1,
$$
which implies that $([\matr{C}]_1,[\matr{D}]_1)\in\Lang_{[\vecb{u}]_1,n,\mathsf{shuffle}}$, unless $[\vecb{m}]_1\matr{P}^\top-[\vecb{m}']_1$ is a solution to the $\dist_{n,1}\mbox{-}\kermdh$ Assumption.

\begin{figure} 
$$
\begin{array}{ll}
\begin{array}{l}
\algK_1(\Gamma,[\vecb{v}]_2,n)\quad (\mathsf{S}_1(\Gamma,[\vecb{v}]_2,n))\\
\hline
S:=\{[s_1]_1,\ldots,[s_n]_1\}, \matr{S}\gets\distlin_{1,n}\\
\crs_\GS\gets\GS.\algK_1(\Gamma)\\
{\crs_\sfbin\gets\Pi_\sfbin.\algK_1(\Gamma,[\matr{M}]_1,[\matr{N}]_1,[\matr{\Lambda}]_1,n)}\\
\text{Return } \ \crs:=([\vecb{v}]_2,\crs_\GS,\crs_\sfbin).\\
(\tau_\GS\gets\GS.\algS_1(\Gamma)\\
\tau_\sfbin\gets\Pi_\sfbin.\algS_1(\Gamma,L,[\matr{M}]_1,[\matr{N}]_1,[\matr{\Lambda}]_1,n).\\
\tau := (\matr{S},\tau_\GS,\tau_\sfbin)).\\
\\
\end{array}
&
\begin{array}{l}
{\algP(\crs, [\matr{C}]_2, [\matr{D}]_2,\langle \matr{P}, \grkb{\delta}\rangle)}\\
\hline
\vvar{X} := [\matr{S}]_1\matr{P}^\top,
\vvar{y} := [\matr{S}]_1\grkb{\delta}\\
\eq_1:= \sum_{i\in[n]}[s_i]_1-\sum_{j\in[n]}\vvar{x}_i = [0]_1\\
\eq_2:= {e(\vvar{X},[\matr{C}]_2^\top)}/{e([\matr{S}]_1,[\matr{D}]_2^\top)}=e(\vvar{y},[\vecb{v}]_2^\top)\\
\pi_\GS\gets\GS.\algP(\crs_\GS,\{\eq_1,\eq_2\},\vvar{X},\vvar{y})\\
\pi_\sfbin \gets \Pi_\sfbin.\algP(\crs_\sfbin, [\matr{F}]_1,\langle\matr{R},\matr{P}^\top\rangle)\\
//{[\matr{F}]_1} = \GS.\Com_{\crs_\GS}(\vvar{X};\matr{R})\\
\text{Return } \  (\pi_\GS,\pi_\sfbin). \\
\\
\end{array}\\
\begin{array}{l}
{\algV(\crs,[\matr{C}]_2,[\matr{D}]_2,(\pi_\GS,\pi_\sfbin))}\\
\hline
\eq_1:= \sum_{i\in[n]}[s_i]_1-\sum_{j\in[n]}\vvar{x}_i = [0]_1\\
\eq_2:= {e(\vvar{X},[\matr{C}]_2^\top)}/{e([\matr{S}]_1,[\matr{D}]_2^\top)}=e(\vvar{y},[\vecb{v}]_2^\top)\\
\mathsf{ans}_1 \gets \GS.\algV(\crs_\GS,\{\eq_1,\eq_2\},\pi_\GS)\\
\mathsf{ans}_2 \gets \Pi_\sfbin.\algV(\crs_\sfbin,[\matr{F}]_1,\pi_\sfbin)\\
\text{Return } \ \mathsf{ans}_1\wedge\mathsf{ans}_2.
\end{array}
&
\begin{array}{l}
{\mathsf{S}_2(\crs,[\matr{C}]_1,[\matr{D}]_1,\tau)}\\
\hline
\eq_1:= \sum_{i\in[n]}[s_i]_1-\sum_{j\in[n]}\vvar{x}_i = [0]_1\\
\eq'_2:= {e(\vvar{X},[\matr{C}]_2^\top)}/{e([1]_1,\matr{S}^\top[\matr{D}]_2^\top)}=e(\vvar{y},[\vecb{v}]_2^\top)\\
\pi_\GS\gets \GS.\algS_2(\crs_\GS,\{\eq_1,\eq'_2\},\tau_\GS)\\
\pi_\sfbin\gets \Pi_\sfbin.\algS_2(\crs_\sfbin,[\matr{F}]_1,\tau_\sfbin)\\
\text{Return }  (\pi_\GS,\pi_\sfbin).
\end{array}
\end{array}$$
\caption{The proof system $\Pi_\mathsf{shuffle}$ for the language $\Lang_{[\vecb{v}]_2,n,\mathsf{shuffle}}$. The matrices $\matr{M},\matr{N},\matr{\Lambda}$ are defined as $\matr{M}:=\usebox{\smat},\matr{N}:=(\vecb{u}_1||\vecb{u}_2),\matr{\Lambda}:= (1,\ldots,1)$, where $\vecb{u}_1,\vecb{u}_2$ are the GS commitment keys from $\crs_\GS$. The proof size is $|\pi_\GS|+|\pi_\sfbin|=(2n+6)|\GG_1|+4|\GG_2|+1|\Z_q|+(2n+11)|\GG_1|+10|\GG_2|=(4n+17)|\GG_1|+14|\GG_2|+1|\Z_q|$.
\label{fig:shuffles}}
\end{figure}



Figure \ref{fig:shuffles} provides detailed description of the proof system for $\Lang_{[\vecb{u}]_1,n,\mathsf{shuffle}}$.

