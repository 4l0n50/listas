\subsubsection{Completeness.}
Follows directly from completeness of $\Psi_\sfbits$ and $\GS$, and for the verifier's third check follows from
\begin{eqnarray*}
{e([\matr{G}_1]_1,[\Lcom]_2^\top)}/{e([\matr{C}]_1,[\matr{L}]_2^\top)}
    & = &
    \dfrac
        {e([\matr{G}_1]_1,\matr{B}^\top[\matr{L}]_2^\top+\matr{R}^\top[\matr{V}]_2^\top)}
        {e([\matr{G}_1]_1\matr{B}^\top+[\vecb{g}_{n+1}]_1\vecb{w}^\top,[\matr{L}]_2^\top)}\\
    & = &
    \dfrac
        {e([\matr{G}_1]_1,\matr{B}^\top[\matr{L}]_2^\top+\matr{R}^\top[\matr{V}]_2^\top-\matr{B}^\top[\matr{L}]_2^\top)}
        {e([\vecb{g}_{n+1}]_1,\vecb{w}^\top[\matr{L}]_2)}\\
    & = &
    e\left([\vecb{g}_{n+1}]_1,\left([\matr{V}]_2\vecb{r}_{n+1}-[\matr{L}]_2\vecb{w}\right)^\top\right)\cdot
    e\left([\matr{G}]_1\pmatri{\matr{R}^\top\\-\vecb{r}_{n+1}^\top},[\matr{V}]_2^\top\right)\\
    & = &
    e([\vecb{g}_{n+1}]_1,[\grkb{\pi}]_2^\top)\cdot e([\matr{\Theta}]_1,[\matr{V}]_2^\top).
\end{eqnarray*}

\subsubsection{Soundness.}

\begin{theorem}
Let $\adv_{\mathcal{PS}}(\advA)$ be the advantage of an adversary $\advA$ against the soundness of the proof system described above. There exists PPT adversaries $\advD$, against $\mathcal{U}_1\mbox{-}\mddh$ in $\GG_1$, and $\advB$, against 1-coordinate Soundness of $\Psi_\sfbits$, such that
$$
\adv_{\mathcal{PS}}(\advA)\leq n\left(4/q + \adv_{\mathcal{U}_1,\GG_1}(\advD)+\adv_\sfbits(\advB)\right).
$$
\end{theorem}

The proof follows from the indistinguishability of the following games:
\begin{itemize}
\item[$\mathsf{Real}$] This is the real soundness game. The output is 1 if the adversary breaks soundness, that is, the adversary submits $[\Lcom]_2\notin\Lang_{[\matr{V}]_2,L,n}$ and the corresponding proof which is accepted by the verifier.
\item[$\sfGame_0$] This is identical as $\mathsf{Real}$ except that algorithm $\algK_1$ does not receive $[\matr{V}]_2$ as input but it samples $([\matr{V}]_2,\matr{V})\in\mathcal{R}_{par}$ according to $\dist^1_\Gamma$.
\item[$\sfGame_1$] This game is identical to $\sfGame_0$ except that the simulator picks a random $j^*\in[n]$, and uses $\matr{V}$ to check if the output of the adversary is such that $[x_{j^*}]_2\in L$. It aborts if $[x_{j^*}]_2\in L$.
\item[$\sfGame_2$] This is game is identical to $\sfGame_1$ but now $\vecb{g}_{j^*}\gets\Z_q^2$.
%\item[$\sfGame_3$] This is game is identical to $\sfGame_2$ but now $\vecb{h}_{i^*}\gets\Z_q^2$.
\end{itemize}

It is obvious that the first two games are indistinguishable. 
The rest of the argument goes as follows. 

\begin{lemma} $\Pr\left[ \mathsf{Game}_1(\advA)=1\right]\geq\dfrac{1}{n}\Pr\left[\mathsf{Game}_0(\advA)=1\right].$
\end{lemma}

\begin{proof}  The probability that
 $\mathsf{Game}_1(\advA)=1$ is the probability that  a) $\mathsf{Game}_0(\advA)=1$ and
b)  $[x_{i^*}]_2 \notin L$. The view of adversary $\advA$ is independent of $j^*$, while, if $\mathsf{Game_0}(\advA)=1$, then there is at least one index $\ell \in [n]$ such that $[x_{\ell}]_2 \notin L$. Thus, 
the probability that the event described in b) occurs conditioned on $\mathsf{Game_0}(\advA)=1$, is greater than or equal to $1/n$ and the lemma follows.
\end{proof}

\begin{lemma} There exists a $\mathcal{U}_1$-$\mddh_{\GG_1}$ adversary $\advD$ such that
$|\Pr\left[\mathsf{Game}_{1}(\advA)=1\right]$ $-\Pr\left[\mathsf{Game}_{2}(\advA)=1\right]|$ $\leq
    \mathsf{Adv}_{\mathcal{U}_1,\GG_1}(\advD).$
\end{lemma}
\begin{proof}
The adversary $\advD$ receives an instance of the $\mathcal{U}_1$-$\mddh_{\GG_1}$ problem, which is a pair
$([\vecb{s}]_1, [\vecb{t}]_1)$, where $\vecb{s}$ is a uniform vector 
of $\Z_q^2$ and $\vecb{t}$ is either a uniform vector in $\Z_q^2$ or 
$\vecb{t}=\gamma\vecb{s}$, for random $\gamma\in\Z_q$.
 
Adversary $\advD$ defines
$[\vecb{g}_{n+1}]_1:= [\vecb{s}]_1$ and the rest of the columns of $[\matr{G}]_1$ are honestly sampled
with the sole exception of $[\vecb{g}_{j^*}]_1$, which is set to $[\vecb{t}]_1$.
The rest of the parameters, namely $\crs_\sfbits$ and $\crs_\GS$, are generated honestly.

It is immediate to see that adversary $\advD$ perfectly simulates $\sfGame_1$ when $\vecb{t}=\gamma\vecb{s}$ and $\sfGame_2$ when $\vecb{t}$ is uniform.  
\end{proof}

%\begin{lemma} There exists a $\mathcal{U}_1$-$\mddh_{\GG_2}$ adversary $\advD_2$ such that
%$|\Pr\left[\mathsf{Game}_{2}(\advA)=1\right]$ $-\Pr\left[\mathsf{Game}_{3}(\advA)=1\right]|$ $\leq
%    \mathsf{Adv}_{\mathcal{U}_1,\GG_2}(\advD_2).$
%\end{lemma}
%\begin{proof}
%The adversary $\advD_2$ receives an instance of the $\mathcal{U}_1$-$\mddh_{\GG_2}$ problem, which is a pair
%$([\vecb{s}]_2, [\vecb{t}]_2)$, where $\vecb{s}$ is a uniform vector 
%of $\Z_q^2$ and $\vecb{t}$ is either a uniform vector in $\Z_q^2$ or 
%$\vecb{t}=\gamma\vecb{s}$, for random $\gamma\in\Z_q$.
% 
%Adversary $\advD_2$ defines
%$[\vecb{h}_{m+1}]_2:= [\vecb{s}]_2$ and the rest of the columns of $[\matr{H}]_2$ are honestly sampled
%with the sole exception of $[\vecb{h}_{i^*}]_2$, which is set to $[\vecb{t}]_2$. Adversary $\advD_2$ samples
%$\matr{G}$ as in $\sfGame_2$ and computes $\crs_\sfbits\gets\algK_1(\Gamma,\matr{G},[\matr{H}]_2)$ which, by Theorem \ref{theo:bits},
%is distributed exactly as $\algK_1(\Gamma,[\matr{G}]_1,\matr{H})$.
%Parameter $\crs_\Phi$ is generated honestly.
%
%It is immediate to see that adversary $\advD$ perfectly simulates $\sfGame_2$ when $\vecb{t}=\gamma\vecb{s}$ and $\sfGame_3$ when $\vecb{t}$ is uniform.  
%\end{proof}

\begin{lemma}
There exists an adversary $\advB$ such that $\Pr[\sfGame_2(\advA)=1]\leq2/q+\adv_{\sfbits}(\advB)$.
\end{lemma}

\begin{proof}
Let $E$ the event where $\vecb{g}_{j^*}$ is not linearly independent from $\{\vecb{g}_{j}:j\neq j^*\}$.
By the Schwartz-Zippel Lemma, $\Pr[E]\leq 2/q$ and thus $\Pr[\sfGame_2(\advA)=1]\leq2/q+\Pr[\sfGame_2(\advA)=1|\neg E]$. It is only left to bound this last probability to prove the lemma. 

If $\sfGame(\advA)=1$, then $[x_{j^*}]_2\notin L$ while all the verification equations are accepted. This implies that a) $\sum_{i\in[m]}\vecb{c}_i-\sum_{j\in[n]}\vecb{g}_j=\gamma\vecb{g}_{n+1}$, for some $\gamma\in\Z_q$, and b) $\matr{G}_1[\Lcom]_2^\top-\matr{C}[\matr{L}]_2^\top=\vecb{g}_{n+1}[\grkb{\pi}]_2^\top+\matr{\Theta}[\matr{V}]_2^\top$.

Given that $\vecb{g}_{j^*}$ is linearly independent from $\{\vecb{g}_{j}:i\neq j^*\}$, it holds that $\{\vecb{g}_{j^*},\vecb{g}_{n+1}\}$ is a basis for $\Z_q^2$ and thus we can define $b_{i,j^*},\overline{w}_{h,i}$ as the unique coefficients in $\Z_q$ such that $\vecb{c}_i = b_{i,j^*}\vecb{g}_{j^*}+\overline{w}_{h,i}\vecb{g}_{n+1}$, for each $i\in[m]$. Let $\grkb{\nu}\in\Z_q^2$ such that $\grkb{\nu}^\top\vecb{g}_{j^*}=1$ and $\grkb{\nu}^\top\vecb{g}_{j}=0$, if $j\neq j^*$, and suppose that $b_{i,j^*}\in\bits$ for all $i\in[m]$. Multiplying a) by $\grkb{\nu}^\top$ on the right one gets that $\sum_{i\in[m]}b_{i,j^*}=1$, and thus there is a unique $i^*\in[m]$ such that $b_{i^*,j^*}=1$. Multiplying b) by $\grkb{\nu}^\top$ on the right one gets that $[\lcom_{j^*}]_2-\begin{bmatrix}l_{i^*}\\0\end{bmatrix}_2=[\matr{V}]_2\matr{\Theta}^\top\grkb{\nu}$. This implies that $[x_{j^*}]_2=[l_{i^*}]_2$ and consequently $\sfGame_2(\advA)=0$. Therefore, when $\neg E$, $\sfGame_2(\advA)=1$ implies that there is some $i^*\in[m]$ such that $b_{i^*,j^*}\notin\bits$.

We construct an adversary $\advB$ against the \emph{1-coordinate Soundness} of $\Psi_\sfbits$ that receives as input $(\Gamma, \crs_\sfbits,\matr{G})$  and simulates $\sfGame_3(\advA)$ until $\advA$ outputs $[\matr{C}]_1$ and $\{\pi^i_\sfbits:i\in[m]\}$. Using $\matr{G}$ opens commitments $[\vecb{c}_1]_1,\ldots,[\vecb{c}_m]_1$ to find out $i^*\in[m]$ such that $b_{i^*,j^*}\notin\bits$, and outputs $[\vecb{c}_{i^*}]_1$  with the proof $\pi^{i^*}_\sfbits$. We conclude that $\Pr[\sfGame_2(\advA)=1|\neg E]\leq \adv_{\sfbits}(\advB)$ which proves the lemma.
\end{proof}

\subsubsection{Zero-Knowledge}

\begin{theorem}
The proof system is perfect quasi-adaptive zero-knowledge.
\end{theorem}

\begin{proof}
Let $[\Lcom]_2\in\Lang_{[\matr{V}]_2,L,n}$, then there exists $\matr{R}\in\Z_q^{2\times n}$ and $\matr{B}\in\bits^{m\times n}$ such that $\matr{1}_{1\times n} = \matr{1}_{1\times m}\matr{B}$ and $[\Lcom]_2=[\matr{L}]_2\matr{B}+[\matr{V}]_2\matr{R}$.

Given that $\matr{G}$ is of rank 1, by Theorem \ref{theo:bits}, the proofs generated by $\Psi_\sfbits.\algS_2$ are identically distributed to the real proof, for every $i\in[m]$. Note also that the equation $\sum_{i\in[m]}[\vecb{c}_i]_1-\sum_{j\in[n]}[\vecb{g}_i]_1=[\vecb{g}_{n+1}]_2\varz$ allows simulation and thus the simulated proof $\pi_\GS$ is correctly distributed.

It is left to show that $[\grkb{\pi}]_2$ and $[\matr{\Theta}]_1$ are uniformly distributed conditioned on satisfying the verification equation. 
%Note first that, given that $\Rck$ is a rank 1 matrix, for every $j\in[n]$ there is some $\overline{w}_{h,j}$ such that $[\vecb{d}_j]_2 = \sum_{i\in[m]}b_{i,j}[\vecb{h}_i]_2+\overline{w}_{h,j}$.
Indeed, the columns of $[\matr{\Theta}]_1$ are a uniformly random elements from $\Span([\vecb{g}_{n+1}]_2)$, as in the real proof, and $[\grkb{\pi}]_2$ is the unique element from $\GG_2^2$ that satisfies the verification equation, as shown below.

\begin{eqnarray*}
\dfrac{e([\matr{G}_1]_1,[\Lcom]_2^\top)}{e([\matr{C}]_1,[\matr{L}]_2^\top)} &=& e([\vecb{g}_{n+1}]_1,[\grkb{\pi}]_2^\top)\cdot e([\matr{\Theta}]_1,[\matr{V}]_2^\top)\\
\Leftrightarrow e([\vecb{g}_{n+1}]_2,[\grkb{\pi}]_2^\top)   & = &
    \dfrac{e([\vecb{g}_{n+1}]_1\grkb{\mu}^\top,[\Lcom]_2^\top)}{e([\vecb{g}_{n+1}\vecb{w}^\top]_1,[\matr{L}]_2^\top)\cdot e([\vecb{g}_{n+1}]\overline{\vecb{r}},[\matr{V}]_2^\top)}\\
\Leftrightarrow [\grkb{\pi}]_2 & = & [\Lcom]_2\grkb{\mu} - [\matr{L}]_2\vecb{w}^\top-[\matr{V}]_2\overline{\vecb{r}}.
\end{eqnarray*}
\end{proof}
