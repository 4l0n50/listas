\documentclass
{llncs}

%\makeatletter
%\def\input@path{{../}}
%\makeatother
%
\input{common-packages}

%\usepackage{fullpage}
\input{preamble}


%\author[1]{Alonso Gonz\'alez}
%\author[1]{Alejandro Hevia}
%\affil[1]{Departamento de Ciencias de la Computaci\'on, Universidad de Chile, Chile\\
%\tt ahevia@dcc.uchile.cl, alonso.gon@gmail.com}}
%\author[2]{Carla R\`afols}
%\affil[2]{Horst-G\"ortz Institute for IT Security and Faculty of Mathematics, Ruhr-Universit\"at Bochum, Germany\\
%{\tt carla.rafols@rub.de}}


\author{\vspace*{-.5cm}}
\institute{\vspace*{-.8cm}}
\title{Aggregated Non-Interactive Zero-Knowledge Set Membership Arguments}
\begin{document}
%\begin{doublespace}

\maketitle
%\vspace*{-.5cm}
\begin{abstract}
    %We consider the following problem: given commitments $c_1,\ldots,c_n$ to values $x_1,\ldots,x_n$, prove in zero-knowledge that each of $x_1,\ldots,x_n$ belongs to some set $S$. We construct a NIZK argument for this problem whose proof size is independent of the number of commitments. Our proof system is secure under the SXDH assumption plus the SSDP (introduced by XXX) in Type III groups, while we can prove it secure under the DLin assumption in Type I groups.
%
%We show how to use our NIZK argument to construct the most efficient \emph{Argument of Correctness of a Shuffle} and \emph{Range Argument} under standard and falsifiable assumptions. 
We construct the most efficient non-interactive \emph{Argument of Correctness of a Shuffle} and \emph{Range Argument} under falsifiable assumptions in asymmetric bilinear groups.
Our constructions use as a common building block a novel quasi-adaptive argument for proving that $n$ commitments open to messages in a public set $S$, with proof-size independent of $n$.  

\end{abstract} 

\section{Introduction}
    Zero-knowledge proofs are proofs which yield nothing beyond the validity of a certain statement. Although one can prove every NP statement in zero-knowledge (going through a proof of circuit satisfiability, for instance),  the literature has extensively explored 
more efficient alternatives for concrete statements which appear often in practice. Among them, some of the most important are: proofs of membership in linear spaces \cite{AC:JutRoy13,EC:LPJY14,C:JutRoy14,EC:AbdBenPoi15,EC:KilWee15}, range proofs \cite{AC:CamChaShe08,PAIRING:RiaKohPre09,FC:ChaLipZha12}, membership in a set \cite{PAIRING:RiaKohPre09,EC:BayGro13}, or correctness of a shuffle \cite{Chaum81,AC:GroLu07,SCN:LipZha12,EPRINT:FauLip15}. 

These problems have been studied following a variety of approaches and techniques. For instance, they have been studied both in the interactive \cite{EC:BayGro12,AC:CamChaShe08,AC:Groth11} and the non-interactive setting \cite{PAIRING:RiaKohPre09,FC:ChaLipZha12,AC:GroLu07,SCN:LipZha12,EPRINT:FauLip15}, and in the latter setting, both under falsifiable (but not always standard) \cite{PAIRING:RiaKohPre09,AC:GroLu07} and non-falsifiable assumptions \cite{FC:ChaLipZha12,SCN:LipZha12,EPRINT:FauLip15} (like knowledge of exponent type of assumptions). 

Generally speaking, non-interactive zero-knowledge proofs under falsifiable assumptions remain more inefficient than other approaches for the same problem (one notable exception being the recent QA-NIZK arguments of membership in linear spaces of  \cite{EC:LPJY14,C:JutRoy14,EC:KilWee15}). However, this is the most desirable alternative from a cryptographic point of view. Indeed, interaction is not  so convenient in practice and further, there is the additional problem of non-transferablility (a proof might not convince a third party who cannot check if the challenges were computed correctly \cite{C:Pass03}).  On the other hand, non-falsifiable assumptions are very strong assumptions whose use is, at the very least, controversial \cite{EPRINT:BCPR13a,AC:BoyPas15}. Although it might still be interesting to use these assumptions in practice, from a theoretical viewpoint it is definitely worth to explore how to improve efficiency based only on standard assumptions.  

This paper focuses on obtaining efficiency improvements for non-interactive arguments based on falsifiable assumptions for two of the interesting examples discussed above, namely, range proofs and interactive proofs of correctness of a shuffle.  

An argument of Correctness of a Shuffle is an essential tool in the construction of \emph{Mix-nets} \cite{Chaum81}. A Mix-net consists of a series of \emph{mixers}, each of which receives as input a set of $n$ ciphertexts and  outputs a \emph{shuffle} of the input ciphertexts. That is, a \emph{rerandomization} of the set of ciphertexts obtained after applying a \emph{random permutation} to the input set of ciphertexts. To enforce the honest behavior of mixers they are required to produce a zero-knowledge argument that the shuffle was correctly computed.  

A Range argument is a tool often required in e-voting and e-cash scenarios, with the purpose of showing that the opening $y$ of some commitment $c$ is an integer in some interval $[A,B]$. For simplicity, the range considered is usually $[0,2^n-1]$ since a proof in any interval can be reduced to a proof in this interval.

To derive efficiency improvements for these two languages we develop specific techniques that we can apply to both problems. Our resulting proofs are more efficient in terms of proof size and are based on more standard assumptions, but they have a rather large common reference string. They build on the recent arguments for membership in linear spaces of \cite{EC:LPJY14,C:JutRoy14,EC:KilWee15} and the argument for proving that some commitment to a vector of integers in $\Z_q^{n}$ opens to $\{0,1\}^n$ due to \cite{AC:GonHevRaf15}. 
         

    \subsection{Our Techniques}\label{sec:techniques}
        All our results are in a bilinear group $gk:=(q,\GG_1,\GG_2,\GG_T,e,\mathcal{P}_1,\mathcal{P}_2)$, where $\GG_1,\GG_2$
and $\GG_T$ are groups of prime order $q$, the elements $\mathcal{P}_1, \mathcal{P}_2$ are generators of 
$\GG_1,\GG_2$ respectively, and $e:\GG_1\times\GG_2\to\GG_T$ is an efficiently
computable, non-degenerate bilinear map. Given a generator $\mathcal{P}_\gamma$ of $\GG_\gamma$, for any $x\in\Z_q$ we define $[x]_\gamma:=x\mathcal{P}_\gamma$, $\gamma\in\{1,2\}$. We simply write $[x]_1[y]_2$ to denote $e([x]_1,[y]_2)$.

Note that in bilinear groups one can always use Groth-Sahai proofs to prove these statements (quadratic equations are complete and one can prove every statement in NP, \cite{EC:GroOstSah06}).  However, a  naive use of GS proofs results in a large proof size ($\Theta(n^2)$ for shuffles, $\Theta(n)$ for range proofs) and in fact, as we discuss below, they have been combined with other strategies to obtain improved asymptotic efficiency. 

\subsubsection{A Common Building Block.} Our starting point is the observation that range and shuffle proofs can be constructed by using as a common building block a 
``zero-knowledge aggregated set membership argument''. This is achieved by slightly modifying some previous strategies used for shuffle and range proofs.

More specifically,  given some publicly known set $S$, such an argument proves that $n$ commitments $c_1,\ldots,c_n$ open to values $x_1,\ldots,x_n \in S$.  The set $S$ is of polynomial size and is either $[0,d-1]\subset\Z_q$ or a subset of $\GG_\gamma$, $\gamma \in \{1,2\}$.  
In other words, an aggregated set membership argument proves that $(c_1,\ldots,c_n)$ is in the language
$$
\Lang_{ck,S}:=\{c: \exists x\in S, w\in\Z_q \text{ s.t. } c=\Com_{ck}(x;w)\}\text{, where }ck\gets\distk,
$$
and $c=\Com_{ck}(x;w)$ is a Groth-Sahai commitment to $x$ with randomness $w$. The proof is Quasi-Adaptive \cite{AC:JutRoy13}, in the sense that the common reference string depends on $ck$ and $S$, which are assumed to be sampled from some distribution $\dist$
and further, the marginal distribution of $ck$ is assumed to be witness samplable. The argument is said to be \textit{aggregated} because the size of the proof is independent of $n$ (but in the soundness proof we will loose a factor of $n$ in the reduction). Specifically, the cost is $\Theta(\log d)$ when $S=[0,d-1]$, and $\Theta(|S|)$ when $S\subset\GG_\gamma$.

Before discussing how to construct such an argument, we show how to use it as a building block for range and shuffle proofs.  
\subsubsection{Range Argument:}
Let $n,d\in\mathbb{N}$, $m:=\log d$, and $\ell:=n/m$. A commitment $c$ opens to a integer $x$ in the range $[0,2^n-1]$ if $(c_1,\ldots,c_\ell) \in\Lang_{ck,[0,d-1]}^\ell$ and  $x=\sum_{i\in[\ell]}x_id^{i-1}$, where $c_i=\Com_{ck}(x_i)$. Indeed, since $x_i\in[0,d-1]$
\begin{eqnarray*}
x & = & \sum_{i\in[\ell]} x_i d^{i-1}
   \in  [0,d^\ell-1]  =  [0,(d^{1/\log d})^n-1] = [0,2^n-1].
\end{eqnarray*}
The statement  $x=\sum_{i\in[\ell]}d^{i-1}x_i$ can be proven using standard techniques while membership in $\Lang_{ck,[0,d-1]}^\ell$ can be proved with an aggregated set membership proof. 

While this way of constructing range arguments has been widely used in the literature, the novelty of our approach is that the cost of proving $x_1,\ldots,x_\ell\in[0,d-1]$ is marginal.
Indeed, the total cost of the range proof depends linearly on $\ell$ and $m$ ($\ell$ is due to the size of the commitments $c_1,\ldots,c_\ell$ and $m$ to the size of an aggregated proof of membership in $\Lang_{ck,[0,d-1]}^\ell$).  Setting $d=n^{k}$ for arbitrary $k$ leads to a proof size of $\ell+m=\Theta(\frac{n}{k \log n})+\Theta(k\log n)=\Theta(\frac{n}{k \log n})$. 
%Note that the asymptotic improvement for this proof (compared to a naive strategy where the proof is of size $n$) relies fundamentally on the fact that the size of $\ell$ proofs of membership in $\Lang_{ck,[0,d-1]}$ is independent of $\ell$. 
  \subsubsection{Shuffle Argument:} The proof is partially inspired by the non-interactive shuffle of \cite{AC:GroLu07}. The statement we want to prove in a correctness of a shuffle argument is : ``Given two vectors of ciphertexts which open, respectively, to vectors of plaintexts $[\vecb{m}_1]_2, [\vecb{m}_2]_2$, prove that 
 $[\vecb{m}_2]$ is a permutation of $[\vecb{m}_1]$''. 
The strategy we follow is roughly the following:  
\begin{itemize}
\item[1)] Publish some vector of group elements $[\vecb{s}]_1 =([s_1]_1,\ldots,[s_n]_1)^\top$ (which we identify with the set $S$ of its components) in the common reference string, where $\vecb{s}$ is sampled from some distribution $\dist_{n,1}$.
\item[2)] The prover commits to $[\vecb{x}]_1=([x_1]_1,\ldots,[x_n]_1)^\top$, a permutation of the set $S$ and proves that the commitments to $[\vecb{x}]_1$ are in $\mathcal{L}^{n}_{ck,S}$.
\item[3)] The prover proves that $\sum_{i \in [n]} [x_i]_1 =\sum_{i \in [n]} [s_i]_1$.
\item[4)] Finally, the prover outputs a proof that:\footnote{Actually, since the prover (a mixer) does not know the randomness nor the decryptions of the ciphertexts but only the randomness of the re-encryptions, she is not able to prove exactly this statement. However, for a cleaner explanation, in this section we assume it is.} 
\begin{equation}\label{shuffle:ker}[\vecb{s}^{\top}]_1 [\vecb{m}_1]_2 =[\vecb{x}^{\top}]_1 [\vecb{m}_2]_2.
\end{equation}
\end{itemize}
The underlying computational assumption is that it is infeasible to find a non-trivial combination of elements of $S$ which adds to $0$, that is, given $[\vecb{s}]_1$ it is infeasible to find $[\vecb{k}]_2 \neq [\vecb{0}]_2$ such that
$\vecb{s}^{\top} \vecb{k}=\vecb{0}$ (this is the $\dist_{n,1}$-$\kermdh$ Assumption of \cite{EPRINT:MorRafVil15}). 

Soundness goes as follows. First, by the soundness of the aggregated set membership proof, $[\vecb{x}]_1 \in S^{n}$ and from the fact that 
 $\sum_{i \in [n]} x_i =\sum_{i \in [n]} s_i$, it holds that if 
 $\vecb{x}$ is not a permutation of $\vecb{s}$, then one can extract in the soundness game (assuming the extractor knows $ck$) a non-trivial linear combination of elements of $S$ which adds to $0$, which contradicts the security assumption. 
Finally, if $\vecb{x}$ is a permutation of $\vecb{s}$,  then equation (\ref{shuffle:ker}) implies that the shuffle is correct, or, again, 
one can extract from   $[\vecb{m}_1]_2,[\vecb{m}_2]_2$ the coefficients of some non-trivial combination of elements of $S$ which is equal to $0$ (breaking the $\dist_{n,1}$-$\kermdh$ Assumption). 

This soundness argument is an augmentation and translation into asymmetric groups of the argument of Groth and Lu \cite{AC:GroLu07}. Roughly speaking, the argument there also consists of two parts: one devoted to proving that some GS commitments open to a permutation of some set in the CRS (in \cite{AC:GroLu07} this is done via a non-standard assumption, the pairing permutation assumption), while the second part (Step 3) is proven very similarly (in particular, the second part also follows from some Kernel Assumption secure in symmetric bilinear groups). 

We note that it is crucial for our soundness argument that it is possible to decrypt the ciphertexts (otherwise we cannot extract solutions to the Kernel problems). This is possible in our case because the encryption public-key is assumed to be witness-samplable and the argument is quasi-adaptive. This explains why we do not have to rely on the notion of culpable soundness, as done in \cite{AC:GroLu07,EPRINT:FauLip15}.


\subsubsection{Set Membership Proofs.} One of the keys to our result is a proper characterization of  $\Lang^{n}_{ck,S}$. Before we move to the aggregated case, we start by discussing how to make one single proof of membership in a set $S$. Our first observation is that  
 membership in $S$ can be written as:
\begin{itemize}
\item If $S \subset \GG_{\gamma}$, and we identify $S$ with $[\vecb{s}]_\gamma=([s_1]_\gamma,\ldots,[s_m]_\gamma)^\top$ then, 
$c \in \Lang^{n}_{ck,S}$ if and only if $\exists \vecb{b} \in \Z_q^{m}$ such that
\vspace{-0.4cm}
$$ 1) \vecb{b} \in \{0,1\}^{m}, \ 2) c=\Com_{ck}(x;w), \ 3) x=\vecb{s}^{\top} \vecb{b}, \ 4) \sum_{i \in [m]} b_i=1.$$
\vspace{-0.2cm}
\item If $S=[0,d-1]$ and $m:=\log d$, then: 
$c \in \Lang^{n}_{ck,S}$ if and only if $\exists \vecb{b} \in \Z_q^{m}$ such that:   
\vspace{-0.2cm}
$$ 1) \vecb{b} \in \{0,1\}^{m}, 2) c=\Com_{ck}(x;w), \ 3) x=(1,2,\ldots,2^{m-1}) \vecb{b} .$$
\end{itemize} 
\vspace{-0.2cm}
That is, both languages can be written in a similar way, except that when $S \subset \GG_{\gamma}$ there is an additional linear constraint that $\vecb{b}$ must satisfy (condition 4)). 

 To avoid distinguishing all the time between both types of subsets, we note that both languages can be seen as special case of the language 
 $\Lang_{[\matr{M}]_1,[\matr{N}]_1,\matr{\Lambda},\grkb{\alpha}}\subseteq\GG_1^\la$, defined as: 
$[\vecb{x}]_1\in\Lang_{[\matr{M}]_1,[\matr{N}]_1,\matr{\Lambda},\grkb{\alpha}}$ if and only if $\exists \vecb{b}\in \Z_q^\lb,\vecb{w}\in\Z_q^\lc$ such that
\begin{eqnarray*}
1) \vecb{b}\in \{0,1\}^\lb
  \wedge \ 2)
\smallpmatrix
{
    \vecb{c}\\
    \grkb{\alpha}
}
=
\smallpmatrix
{
    \matr{M}       & \matr{N}\\
    \matr{\Lambda} & \matr{0}_{\ld\times \lc}
}
\smallpmatrix
{
    \vecb{b}\\
    \vecb{w}
}.
\end{eqnarray*}
The basic idea is that a GS commitment is a linear combination of the commitment keys whose coefficients are the randomness and the committed values, i.e.  
a commitment to a scalar $x \in \Z_q$ is defined as $\Com_{ck}(x;w)=x [\vecb{u}_1]_1+w [\vecb{u}_2]_1$, for $ck=(\vecb{u}_1,\vecb{u}_2)$, so essentially membership in this space amounts to some ``linear conditions'' plus proving that $\vecb{b}$ is binary. For instance in the case where $S=[0,d-1]$, it should hold that:
$
    \vecb{c} = \pmatri{\matr{M}&\matr{N}}\smallpmatrix{\vecb{b}\\ w}
=
\begin{pmatrix}
    (\vecb{u}_1 & \ &2 \vecb{u}_1  &   \ & \ldots    & \  &  2^{m-1} \vecb{u}_1)& \ &  \vecb{u}_2
\end{pmatrix}
\smallpmatrix
{
    \vecb{b}\\
    w
}$.
(In this case, because there is no condition 4) $\matr{\Lambda}, \grkb{\alpha}$ are zero and are ignored). For more details on how this matrices are constructed see Sect. \ref{sec:bin-lang}.
%For the purposes of this high-level explanation of our results, it is not too relevant to see how $[\matr{M}]_1$, $[\matr{N}]_1$, $\matr{\Lambda}$ and $\grkb{\alpha}$ look like in each of our two examples (they depend of the commitment key, the set $S$ and the linear constraints expressed by 4)). 


\subsubsection{Proof Strategy.} The most efficient strategy we are aware of for this type of statements follows a 
 commit-and-prove approach. Namely, to prove that such a vector $\vecb{b}$ exists, one computes 
GS commitments $[\vecb{d}_i]_1$, $i \in [m]$, to all coordinates of $\vecb{b}$ and then it proves two independent statements, namely that:
\vspace{-0.2cm}
\begin{itemize}
\item $\exists \vecb{b}\in \Z_q^\lb, r_i \in \Z_q$ such that  
$1') \vecb{b}\in \{0,1\}^\lb$ and $3') \forall i \in [\lb], \vecb{d}_i=\begin{pmatrix} \vecb{u}_1 &   \vecb{u}_2 \end{pmatrix}   \smallpmatrix{b_i  \\ r_i}$,
\item   $\exists \widetilde{\vecb{b}} \in \Z_q^\lb, \vecb{w} \in\Z_q^\lc$ such that  
   $2') \smallpmatrix
{
    \vecb{c}\\
    \grkb{\alpha}
}
=
\smallpmatrix{
    \matr{M}       & \matr{N}\\
    \matr{\Lambda} & \matr{0}_{\ld\times \lc}
}
\smallpmatrix
{
    \widetilde{\vecb{b}}\\
    \vecb{w}
}$ and $3') \forall i \in [m], \vecb{d}_i=\begin{pmatrix} \vecb{u}_1 &   \vecb{u}_2 \end{pmatrix}   \smallpmatrix{
\widetilde{b}_i  \\ r_i}$.
\end{itemize}
More specifically: use the QA-NIZK argument for bit-strings of \cite{AC:GonHevRaf15} for the first statement;
use the QA-NIZK argument for linear spaces of \cite{C:JutRoy14,EC:KilWee15} to the second statement
(note that conditions 2') and 3') can be written down as a single system of equations with a large matrix $\widetilde{\matr{M}}$ and then satisfiability of 
2') and 3') is equivalent to proving that the vector $(\vecb{c}^{\top},\grkb{\alpha}^{\top},\vecb{d}_1^{\top}, \ldots, \vecb{d}_m^{\top})^\top$ is in the span 
of this large matrix $\widetilde{\matr{M}}$).

Since both proofs are constant-size, the resulting proof size is dominated by the cost of the commitments to $b_i$, which is $O(m)$. 
For soundness, the important point here is that we never prove that $\vecb{b}=\widetilde{\vecb{b}}$, but, since GS commitments are perfectly binding (or, said otherwise, because $\begin{pmatrix} \vecb{u}_1 &   \vecb{u}_2 \end{pmatrix}$
has full rank), equality holds. This immediately proves that the statement is in the language.  
\subsubsection{Aggregated Set Membership Proofs.} An aggregated set membership proof amounts to proving membership in $\Lang_{[\matr{M}]_1,[\matr{N}]_1,\matr{\Lambda},\grkb{\alpha}}^n$. By definition, $([\vecb{c}_1]_1,\ldots, [\vecb{c}_n]_1) \in\Lang_{[\matr{M}]_1,[\matr{N}]_1,\matr{\Lambda},\grkb{\alpha}}^n$ if and only if  $\forall j \in [n], \exists \vecb{b}_j\in \Z_q^\lb,\vecb{w}_j\in\Z_q^\lc$ such that
$$
 1) \vecb{b}_j\in \{0,1\}^\lb
  \wedge \ 2)
\smallpmatrix
{
    \vecb{c}_j\\
    \grkb{\alpha}
}
=
\smallpmatrix
{
    \matr{M}       & \matr{N}\\
    \matr{\Lambda} & \matr{0}_{\ld\times \lc}
}
\smallpmatrix
{
    \vecb{b}_j\\
    \vecb{w}_j
}.
$$
Recall that we want a proof size independent of $n$ and this rules out the naive approach of committing to all the coordinates of $\vecb{b}_j$  for all $j \in [n]$, because the resulting size of the GS commitments is $\Theta(nm)$. 
Therefore, to improve on the asymptotic size of the proof, we are forced to use shrinking commitments to $b_{ij}$. We stress that it is far from clear how to do this, as it might break down the soundness argument completely (e.g. in the single proof, we used in a fundamental way the uniqueness of the openings). In fact, overcoming this problem is probably one of the main technical contributions of this paper. 

Our idea is to use as a shrinking commitment the two-dimensional generalization of Multi-Pedersen commitments used implictly by Gonz\'alez \textit{et al.} \cite{AC:GonHevRaf15}. Given some matrix $\matr{G} \in \Z_q^{2 \times (n+1)}$ sampled from some distribution $\dist_{2,n+1}$, $\mathsf{MP}.\Com(\vecb{y} \in \Z_q^{n}; r \in \Z_q)=[\matr{G}]_1 \smallpmatrix{\vecb{b} \\ r}$. The special thing about these commitments is that one can set a ``hidden'' linearly independent column of 
$\matr{G}$, and thus commitments are perfectly binding at some coordinate $j^*\in[n]$ which is hidden to the adversary.
%Intuitively, the new commitment is defined in a space of one extra-dimension so that one coordinate of information is preserved. Further, under the DDH Assumption, one can define computationally indistinguishable distributions for $\matr{G}$ which hide which is the binding coordinate.    

Define the matrix $\matr{B}=(\vecb{b}_1|| \ldots || \vecb{b}_n) \in \{0,1\}^{\lb \times n}$ and let $\vecb{b}_i^*$ be the $i$th row of $\matr{B}$. To prove $([\vecb{c}_1]_1,\ldots, [\vecb{c}_n]_1) \in\Lang_{[\matr{M}]_1,[\matr{N}]_1,\matr{\Lambda},\grkb{\alpha}}^n$, we first compute MP commitments $[\vecb{d}_i]_1$, $i \in [\lb]$, to $\vecb{b}_i^*$.  As before, 
the proof actually consists of two independent statements:
\begin{itemize}
\item $\exists \vecb{r} \in \Z_q^\lb, \matr{B} \in \Z_q^{\lb \times n}$ such that  
$1'') \matr{B} \in \{0,1\}^{\lb \times n}$ and $3'') \forall i \in [m], \vecb{d}_i=\matr{G}\smallpmatrix{\vecb{b}_i^*  \\ r_i}$,
\item $\exists \widetilde{\vecb{r}}\in\Z_q^\lb, \vecb{w}_1,\ldots,\vecb{w}_n \in\Z_q^\lc, \widetilde{\matr{B}} \in \Z_q^{\lb \times n}$, (whose rows are denoted as $\widetilde{\vecb{b}}_i^*$, $i \in [\lb]$, and the columns $\widetilde{\vecb{b}}_j$, $j \in [n]$), such that  
   $2')\forall i\in[n], \smallpmatrix
{
    \vecb{c}_j\\
    \grkb{\alpha}
}
=
\smallpmatrix
{
    \matr{M}       & \matr{N}\\
    \matr{\Lambda} & \matr{0}_{\ld\times \lc}
}
\smallpmatrix
{
    \vecb{b}_j\\
    \vecb{w}_j
}$ and $3') \forall i \in [\lb], \vecb{d}_i=\matr{G}   \smallpmatrix{\widetilde{\vecb{b}}^*_i  \\ \widetilde{r}_i}$.
\end{itemize}
Again, for the first statement we use a (slight modification) of \cite{AC:GonHevRaf15} and for the second
a QA-NIZK argument for linear spaces of \cite{C:JutRoy14,EC:KilWee15} (after writing all the equations appropriately). 
With this approach, the proof remains of size $O(m)$, the size of the commitments, while the rest of the proof is constant. 

The interesting part is the soundness argument. The previous argument for $n=1$ fails here because now there is no guarantee that 
 $\matr{B}=\widetilde{\matr{B}}$ (as the openings of $[\vecb{d}_i]_1$ are not unique).  However, as we said, the distribution of the MP commitment key can be chosen so that it is binding at some coordinate $j^*$ which is computationally hidden to the adversary. This implies that for all $i$, the $j^*$th coordinate of row $\vecb{b}_i^*$ and $\widetilde{\vecb{b}}_i^*$ is equal, i.e. the $j^*$th column $\vecb{b}_j$ of $\matr{B}$ and $\widetilde{\matr{B}}$ are equal.  

Thus, we have that for the coordinate $j^*$, the proof is sound (because we have the uniqueness of openings which was necessary to prove soundness), i.e. the adversary cannot break soundness for any tuple $([\vecb{c}_1]_1,\ldots, [\vecb{c}_n]_1)$ such that $[\vecb{c}_j^*]_1 \notin \Lang_{[\matr{M}]_1,[\matr{N}]_1,\matr{\Lambda},\grkb{\alpha}}$. 
But since $j^*$ is computationally hidden, we can reduce soundness to one coordinate soundness with a loss in the reduction of $1/n$. 


%    \subsection{Applications}\label{sec:intro-app}
%        As applications of the aggregated set membership NIZK argument we show how to construct efficient Shuffle and Range arguments under standard assumptions.
%For concreteness, assume that $ck$ defines perfectly binding Groth-Sahai (GS) commitments to scalars in the SXDH instantiation. That is, a commitment to $x$ using randomness $r$ is  $[\vecb{c}]_1:=\GS.\Com_{ck}(x;r)\in\GG_1^2$, which implies that $\Lang_{ck,S}^n\subseteq\GG^{2\times n}_1$. 

\begin{description}
\item[Range Argument:]
Let $n,d\in\mathbb{N}$, $m:=\log d$, and $\ell:=n/m$.
Commitment $c$ opens to a integer in the range $[0,2^n-1]$ iff $(c_1,\ldots,c_n)^\top\in\Lang_{ck,[0,d-1]}^\ell$ such that $c-\sum_{i\in[\ell]}d^{i-1}c_i=\Com_{ck}(0;r)$, for some $r\in\Z_q$. Indeed, let $x$ the opening of $c$ and $x_i$ the opening of $c_i$, $i\in[\ell]$. If $(c_1,\ldots,c_n)^\top\in\Lang_{ck,[0,d-1]}^\ell$ then $x_i\in[0,d-1]$, $i\in[\ell]$, and then $c-\sum_{i\in[\ell]}d^{i-1}c_i=\Com_{ck}(x-\sum_{i\in[\ell]}x_id^{i-1})=\Com(0;r)$ implies that
\begin{eqnarray*}
x & = & \sum_{i\in[\ell]} x_id^{i-1}
   \in  [0,d^\ell-1]  =  [0,(d^{1/\log d})^n-1] = [0,2^n-1].
\end{eqnarray*}

\item[Shuffle Argument:] We first construct a proof that a commitment opens to a permutation of the elements of a set $S\subseteq\GG_1$ from a proof of membership in $\Lang_{ck,S}^n$. In Sect. \ref{sec:shuffle} we show how to use this proof to construct a proof of correctness of a shuffle.

Let $[\vecb{s}]_1=([s_1]_1,\ldots,[s_n]_1)^\top\in\GG_1^n$ such that it is infeasible to find $[\vecb{k}]_2\in\GG_2^n\setminus\{\vecb{0}_{n\times 1}\}$ such that $e([\matr{s}]_1^\top,[\vecb{k}]_2)=1_T$ (a \emph{Kernel Diffie Hellman assumption} \cite{EPRINT:MorRafVil15}). Let $f_1:=\Com([x_1]_1;r_1)$, $\ldots$, $f_n:=\Com_{ck}([x_n]_1;r_n)$. If $(f_1,\ldots,f_n)^\top\in\Lang_{ck,S}^n$ and $\sum_{i\in[n]}f_i-\sum_{i\in[n]}\Com([s_i]_1;0)=\Com_{ck}([0]_1;r)$, for some $r\in\Z_q$, then $(f_1,\ldots,f_n)$ opens to a permutation of $S=\{[s_1]_1,\ldots,[s_n]_1\}$.
Indeed, if $(f_1,\ldots,f_n)^\top\in\Lang_{ck,S}^n$, then $\sum_{i\in[n]}[x_i]_1=\sum_{j\in[n]}\ell_j[s_j]_1$, where $\ell_j\in[n]$ is the number of $[x_i]_1$'s equal to $[s_j]_1$, and $\sum_{i\in[n]}f_i-\sum_{j\in[n]}\Com_{ck}([s_j]_1;0)=\Com_{ck}([0]_1;r)$ implies that
$$
\sum_{i\in[n]}[x_i]_1 - \sum_{j\in[n]}[s_j]_1 = \sum_{j\in[n]}\ell_j[s_j]_1 - \sum_{j\in[n]}[s_j]_1 = \sum_{j\in[n]} (l_j-1)[s_j]_1 = [0]_1.
$$
Therefore, $e([\matr{s}]_1^\top,[\vecb{k}]_2)=1_T$, where $[\vecb{k}]_2:=(\ell_1-1,\ell_2-1,\ldots,\ell_n-1)^\top\mathcal{P}_2$ and $\mathcal{P}_2$ is the generator of $\GG_2$, and, unless $[\vecb{k}]_2$ allows to break the Kernel Diffie Hellman Assumption, $\ell_1=\ell_2=\ldots=\ell_n=1$ and thus $([x_1]_1,\ldots,[x_n]_1)$ is a permutation of $([s_1]_1,\ldots,[s_n]_1)$.
\end{description}
%
%\subsubsection{1-coordinate Soundness.}
%We construct a constant size QA-NIZK argument of membership in the language
%
%$$
%\Lang_{[\matr{G}]_1,\sfbits} := \{[\vecb{c}]_1\in\GG_1^k : \exists \vecb{b}\in\bits^n,\vecb{w}\in\Z_q^m \text{ s.t. } \vecb{c} = \matr{G}_1\vecb{b}+\matr{G}_2\vecb{w}\},
%$$
%
%where $\matr{G}=\matr{G}_1||\matr{G}_2\in\GG_1^{k\times n}\times\GG_1^{k\times m}$. Unlike [AC:Losotros15], we do not require $k>n$ (perfectly binding commitments) and in fact $k=2$ suffices. When $k=2$ and $m=1$, $\matr{G}$ defines a natural variant of the \emph{length-reducing Pedersen Commitments}  widely used in the literature \cite{EPRINT:Groth09a,C:FurSak01,CCS:Neff01,SCN:KiaZho06,ManualAC:Lipmaa03}. For simplicity, from here onwards we stick to the case when $k=2$ and $m=1$, and we define $\MP.\Com_{[\matr{G}]_1}(\vecb{b};w):=[\matr{G}]_1\vecb{b}+w\vecb{g}_{n+1}$.
%
%However, we are no longer able to prove Soundness but only a relaxed (but still useful) notion which we call \emph{1-coordinate Soundness}.\footnote{The statement $[\vecb{c}]_1\in\Lang_{[\matr{G}]_1,\sfbits}$, when $\matr{G}\in\GG_1^{2\times (n+1)}$, seems to be not efficiently falsifiable even with access to the discrete logarithm of $[\matr{G}]_1$. In fact the only approach approach to falsify $[\vecb{c}]_1\in\Lang_{[\matr{G}],\sfbits}$ we are aware of is to use brute-force to iterate over the $2^n$ witnesses. This should be looked as an  indication that we can not prove full soundness under falsifiable assumptions.
%%Although is not clear if $\Lang_{[\matr{G}]_1,\sfbits}$ is a NP-hard language, which will gather the impossibility results from \cite{STOC:GenWic11}, a conservative perspective will be to suppose that there is no proof $o(n)$ under falsifiable assumptions.
%}
%Intuitively, this relaxed notion of soundness means that, if one can program $\matr{G}$ such that commitments are perfectly binding for coordinate $i^*$, then the adversary can not produce a valid proof for $[\vecb{c}]_1=\MP.\Com_{[\matr{G}]_1}(\vecb{b};w)$ if $b_{i^*}\notin\bits$. More specifically, the \emph{Completeness} and \emph{Zero-Knowlege} properties remains exactly as for QA-NIZK, and \emph{Soundness} is relaxed as follows:
%
%\begin{definition}[$1$-coordinate Soundness] If $\vecb{g}_{n+1}\neq\vecb{0}$ and there exists an index $i^*$ such that $\vecb{g}_{i^*} \notin \mathsf{Span}(\{\vecb{g}_{i} : i \neq i^*\})$, then:
%$$
%\Pr\left[
%    \begin{array}{l}
%        \crs\gets\algK_1(\gk,[\matr{G}]_1);\\
%        ([\vecb{c}]_1,\mathsf{proof}) \gets \advA(\mathsf{crs})
%    \end{array}:
%    \begin{array}{l}
%%    \exists b_{i^*}\in\Z_q, \vecb{w}\in\Z_q^n\text{ s.t. } b_{i^*} \notin \{0,1\} \wedge\\
%%        \vecb{c}= b_{i^*} \vecb{g}_{i^*} + \sum_{j \in [n+1], j\neq i^*} w_{j} \vecb{g}_{j}  \wedge
%        \algV(\mathsf{crs},[\vecb{c}]_1,\mathsf{proof})=1\wedge \exists \vecb{b}\in\Z_q^n,w\in\Z_q\text{ s.t. }\\
%        {[\vecb{c}]_1=\MP.\Com_{[\matr{G}]_1}(\vecb{b};w)\wedge b_{i^*}\notin\bits}
%    \end{array}
%    \right]\leq\negl(\lambda).$$
%\end{definition}
%
%We illustrate the usefulness of 1-coordinate soundness by describing, at a high level, how it can be used to construct a QA-NIZK proof system for $\Lang_{[\matr{U}]_1,\sfbits}$with ``full'' soundness, when $[\matr{U}]_1$ defines perfectly binding commitments, similarly as done in XXX.
%
%\begin{example}
% The proof system combines two proof systems: one for showing equal openings of commitments computed with $[\matr{U}]_1$ and $[\matr{G}]_1:=[\matr{G}_1||\vecb{g}_{n+1}]_1$ as commitments keys, where $\matr{G}_1:=(\mu_1\vecb{g}_{n+1}||\ldots||\mu_n\vecb{g}_{n+1})$ and $\mu_1,\ldots,\mu_n\gets\Z_q$, and another proof system for showing membership in $\Lang_{[\matr{G}]_1,\sfbits}$ with 1-coordinate soundness.
%
%Given $[\vecb{c}]_1\in\GG_1^{n+m}$, $\vecb{b}\in\bits^n$, and $\vecb{w}\in\Z_q^m$ such that $\vecb{c}=\matr{U}_1\vecb{b}+\matr{U}_2\vecb{w}$ , the prover computes an additional commitment $[\vecb{c}']_1\gets\MP.\Com_{[\matr{G}]_1}(\vecb{b})$ and show that: a) $[\vecb{c}]_1$ and $[\vecb{c}']_1$ opens to the same value and b) $[\vecb{c}']_1\in\Lang_{[\matr{G}]_1,\sfbits}$ with 1-coordinate soundness.
%\end{example}
%
%We briefly show why 1-coordinate soundness for the proof of membership in $\Lang_{[\matr{G}]_1,\sfbits}$ implies full soundness for the proof of membership in $\Lang_{[\matr{U}]_1,\sfbits}$.
%
%Recall that each $\vecb{c}$ defines a unique witness $\vecb{b}\in\Z_q^n$, which can be extracted with knowledge of the discrete logarithm of $[\matr{U}]_1$, and that $[\vecb{c}]_1\in\Lang_{\matr{U},\sfbits}$ iff $ b_i\in\bits$ for all $i\in[n]$. We consider three games:
%\begin{description}
%\item[$\mathsf{Real}$:] The adversary wins if produces a valid proof for some $[\vecb{c}]_1\notin\Lang_{[\matr{U}]_1,\sfbits}$.
%\item[$\sfGame_0$:] Exactly as the real soundness game but $\matr{U}$ is internally sampled and used to check if $b_{i^*}\in\bits$ for a random $i^*\in[n]$. Abort if this is not the case.
%\item[$\sfGame_1$:] Exactly as $\sfGame_1$ but now $\vecb{g}_{i^*}\gets\Z_q^2$.
%\end{description}
%
%Given that an adversary wins only if $[\vecb{c}]_1\notin\Lang_{[\matr{U}]_1,\sfbits}$, is not hard to see that $\Pr[\mathsf{Real}(\advA)=1] \leq n\Pr[\sfGame_0(\advA)=1]$.
%Additionally, using a decisional assumption in $\GG_1$, one can prove that $\Pr[\sfGame_0(\advA)=1]\approx\Pr[\sfGame_1(\advA)=1]$.
% 
%Note that in $\sfGame_1$ with overwhelming probability $\vecb{g}_{i^*}\notin\Span(\{\vecb{g}_i:i\neq i^*\})$. Therefore, with overwhelming probability, there is a unique $x_{i^*}$ such that $\vecb{c}'=x_{i^*}\vecb{g}_{i^*}+\sum_{i\neq i^*}x_i\vecb{g}_i$, a) implies that necessarily $x_{i^*}=b_{i^*}$ were $b_{i^*}$ is $i$ th coordinate of the unique opening of $\vecb{c}$. Given that the game aborts if $b_{i^*}\in\bits$, then a sucessfull adversary should be breaking the 1-coordinate Soundness property.
%
%\subsubsection{The power of 1-Coordinate Soundness.}
%Lets interpret the previous proof in a more abstract way. Given that the statement $[\vec{c}]_1\in\Lang_{[\matr{U}]_1,\sfbits}$ can be equivalently expressed as the set of statements $\{\stmnt_i:i\in[n]\}$, where $\stmnt_i:=b_i\in\bits$, in $\sfGame_0$ it is checked if a random statement, $\stmnt_{i^*}$, do not hold. If the verifier accept the proof that $[\vecb{c}]_1$ and $[\vecb{c}']$ open to the same value in $\sfGame_1$, if $\stmnt_{i^*}$ do not hold, then 1-coordinate soundness also don not hold. Now we will see how to apply the same idea to solve different problems.
%
%\subsubsection{Range Proofs.}
%In a range proof, one wants to prove that some commitment $\Com(x;r)$ is such that $x\in[0,2^n-1]$ for some $n\in\mathbb{N}$, which can be equivalent expressed as $x=\sum_{i\in[l]}x_id^{i-1}$, where $l=n/\log d$, and $\{\stmnt_i:i\in[n]\}$, where $\stmnt_i:=x_i\in[0,d-1]$. Now we need to give a proof such that $\neg\stmnt_{i^*}$ implieas a violation of 1-coordinate soundness. Note that $x_i\in[0,d-1]$ iff $x_i=\sum_{i\in[\log d]}b_{i,j}2^{i-1}$, therefore we compute $d\gets\Com(\vecb{x})$ and $[\vecb{c}_i]_1:=\MP.\Com(\vecb{b}_i)$, for each $i\in[\log d]$, and show that a) $d$ and $\sum_{i\in[\log d]}[\vecb{c}_i]_12^{i-1}$ open to the same value, b) $[\vecb{c}_i]_1\in\Lang_{[\matr{G}]_1,\sfbits}$, for each $i\in[n]$, and c) $x=\sum_{i\in[l]}x_id^{i-1}$.
%
%%In the abstract, the proof for $\Lang_{\matr{U},\sfbits}$ works because the following conditions holds
%%\begin{enumerate}
%%\item There exists a set of equations $\{\eq_i(b_i,b'_i):i\in[n]\}$, in variables $\vecb{b}$ and $\vecb{b}'$ and where $n=\poly(\lambda)$, such that $[\vecb{c}]\notin\Lang_{\matr{U},\sfbits}$ implies that there is some $i\in[n]$ such that $\eq_i$ does not hold. (In the specific case of $\Lang_{\matr{U},\sfbits}$, $\eq_i(b_i,b'_i):=\{b_i(b_i-1)=0,b_i=b_i'\}$).
%%\item $\MP.\Com_{[\matr{G}]_1}(\vecb{b}')\in\Lang_{[\matr{G}],\sfbits}$ with 1-coordinate soundness.
%%\item If $[\vecb{c}]_1\notin\Lang_{[\matr{U}]_1,\sfbits}$, then, for any $i\in[n]$, there is way to check if $\eq_i(b_i,b'_i)$ holds without knowledge of $\matr{G}$.
%%\item For any $i\in[n]$, if $\eq_i$ does not holds, then $b'_i\notin\bits$ (perhaps because otherwise some computational assumption might be broken, as in the case of $\Lang_{[\matr{U}]_1,\sfbits}$).
%%\end{enumerate}
%%
%%Condition 1 allows to reduce the statement $x\in\Lang$ to a random check of a single equation with only a small security loss. Condition 2 allows to randomly check a single equation independent of how $[\matr{G}]_1$ is sampled, say if $\hvecb{g}_{i^*}\in\Span(\{\vecb{g}_i: i\neq i^*\})$ or not. 
%%
%%Next we show that this three requirements
%%The statement $[\vecb{c}']_1\in\Lang_{[\matr{G}]_1,\sfbits}$, when $\matr{G}\in\GG_1^{2\times (n+1)}$, seems to be not efficiently falsifiable even with access to the discrete logarithm of $[\matr{G}]_1$. In fact the only approach approach to falsify $[\vecb{c}']_1\in\Lang_{[\matr{C}]}$ we are aware of is to use brute-force to iterate over the $2^n$ witnesses. Although is not clear if $\Lang_{[\matr{G}]_1,\sfbits}$ is a NP-hard language, which will gather the impossibility results from \cite{STOC:GenWic11}, a conservative perspective will be to suppose that there is no proof $o(n)$ under falsifiable assumptions.
%
%%The proof system for  can be reinterpreted as  
%%Intuitively, 1-coordinate soundness allows to show the satisfiability of a set of quadratic use length-reducing commitments
%
%\subsubsection{QA-NIZK argument with 1-coordinate soundness for $\Lang_{\matr{G},\sfbits}^n$.}
%We give a QA-NIZK argument, with 1-coordinate soundness, of membership in the language 
%$$
%\Lang_{\matr{G},\sfbits}^n := \{[\matr{C}]_1\in\GG_1^{2\times n} : \exists \matr{B}\in\bits^{m\times n},\vecb{w}\in\Z_q^{n} \text{ s.t. } \matr{C} = \matr{G}_1\matr{B}+\vecb{g}_{n+1}\vecb{w}^\top\},
%$$
%Is easy to construct a proof system for $\Lang_{\matr{G},\sfbits}^n$, with communication $\Theta(n)$, from a proof system for $\Lang_{[\matr{G}]_1,\sfbits}$.
%Indeed, $n$ proofs for $[\vecb{c}_i]_1\in\Lang_{[\matr{G}]_1,\sfbits}$ are a proof for  $[\matr{C}]\in\Lang^n_{[\matr{G}],\sfbits}$. Further, in Sect. XXX we show how to construct a constant size proof for this language, which allows us to construct more efficient Range and Shuffle arguments.
%
%\subsubsection{$\Theta(\log n)$ proof of membership in a Set.}
%For our Shuffle argument is useful to prove that $n$ commitments opens to values in some set $\{[a]_1,\ldots,[a]_n\}$. We show how to construct a single proof of size $\Theta(\log n)$ whenever $[a]_1,\ldots,[a]_n$ are fixed in advance. 

    \subsection{Related Work}
        \subsubsection{Zero Knowledge Set Membership Arguments.}
Camenisch et al. constructed $\Theta(1)$ interactive Zero-Knowledge set membership arguments using Boneh-Boyen Signatures, and they prove them secure under the $q$-SDH assumption \cite{AC:CamChaShe08}. Bayer and Groth constructed $\Theta(\log |S|)$ interactive Zero-Knowledge arguments for polynomial evaluation, which can be used to construct set membership arguments, relying only on the discrete logarithm assumption \cite{EC:BayGro13}. There is also a vast literature on \emph{Accumulators} \cite{EC:BenDeM93,C:CamLys02,AC:TsuXu03,RSA:LNguyen05,PKC:CamKohSor09,ICICS:WanWanPie07}, which can be used to construct $\Theta(1)$ proofs {\color{red}(creo q con accumulators el statement no es (no puede ser?) un commitment)}. However, none of the previous constructions have addressed the problem of aggregating many proofs, and a direct use of them will end up with a proof of size $\Omega(n)$.
%Further, none of them seems to consider the case when $S$ is an arbitrarily chosen subset of $\GG_1$.

\subsubsection{Vector Commitments.}
The two dimensional Multi-Pedersen commitments introduced in this work bears some similarities with \emph{Vector Commitments} introduced by Catalano and Fiore \cite{PKC:CatFio13}. In fact, one can commit to a vector $\vecb{m}\in\Z_q^n$ as $[\vecb{c}]_1:=\MP.\Com_{ck}(\vecb{m};0)$ (see more details in Sect. \ref{sec:comm}) and show that it opens to $m_i$ at coordinate $i$ with a proof that $([\vecb{c}]_1-\MP.\Com_{ck}(m_i\vecb{e}_i;0))\in\Span(\{\MP.\Com_{ck}(\vecb{e}_j;0):j\neq i\}))$, where $\vecb{e}_i\in\Z_q^n$ is the $i$ th canonical vector. However, we do not elaborate more in this application since Catalano and Fiore's construction is (by a constant factor) more efficient in terms of CRS size and commitment size, and also relies on weaker assumptions.


\subsubsection{NIZK Shuffle and Range Arguments.}
The most efficient NIZK Shuffle argument under falsifiable assumptions is the one from Groth and Lu \cite{AC:GroLu07}, which works for BBS ciphertexts. The proof size is linear in the number of ciphertexts, specifically $15n + 120$ group elements in Type I groups. The security of their construction relies on two assumptions: the \emph{Paring Product Assumption} and the \emph{Permutation Pairing Assumption}. The first assumption is a $\dist_{2,n}\mbox{-}\kermdh$ Assumption, when $\matr{M}\gets\dist_{2,n}$ is of the form $\matr{M}:=\pmatri{x_1,\ldots,x_n\\x_1^2,\ldots,x_n^2}$ for $x_i\gets\Z_q$, $i\in[n]$. The second assumption is proven generically secure in \cite{AC:GroLu07} but it seems to be unrelated with any other assumption.

Using \emph{Non-falsifiable Assumptions} (i.e. Knowledge of exponent type of assumptions), Lipmaa and Zhang constructed a shuffle argument with communication $6n\sG+11\sH$ \cite{SCN:LipZha12}, and recently Fauzi and Lipmaa constructed a shuffle argument with communication $(5n+2)\sG+2n\sH$ \cite{EPRINT:FauLip15}.

Rial, Kohlweiss, and Preneel constructed a range argument in $[0,2^n-1]$ with communication $\Theta(\frac{n}{\log n -\log\log n})$ and prove it secure under the $q$-HSDH assumption \cite{PAIRING:RiaKohPre09}. One might get rid of the $q$-HSDH assumption replacing the \emph{P-signature} with any \emph{Structure Preserving Signature}, but, since the proof requires $\frac{n}{\log n-\log \log n}$ Groth-Sahai proofs of the satisfiability of the signature's verification equation the signature's size is at least 7 group elements \cite{C:KilPanWee15,C:LibPetYun15}, the resulting protocol far less efficient.
Using Non-falsifiable Assumptions, Chaabouni, Lipmaa, and Zhang constructed a range argument with constant communication \cite{FC:ChaLipZha12}. 

A detailed comparison of our Shuffle and Range arguments with the most efficient constructions under falsifiable assumptions is depicted in Table \ref{table:eff}.

\input{intro/table-eff}

\subsubsection{Quasi-Adaptive NIZK and Culpable Soundness.} The protocols in this work are proven Quasi-Adaptive NIZK (QA-NIZK) \cite{AC:JutRoy13}.
%basically means that the language can be set based on the constants sampled in the CRS and vice versa. More specifically, and
In the scope of this work, it means that the CRS defines \emph{commitment keys} (resp. \emph{public keys}) and we consider the languages of commitments (resp. ciphertexts) computed under those commitment keys (resp. public keys), where the committed value (resp. plaintext) satisfy an additional property. From a technical point of view, 
%In order to not trivialize the zero-knowledge definition, the simulator must not have access to the discrete log of the commitment-keys (resp. the secret key).

In the case of shuffle arguments, Groth and Lu \cite{AC:GroLu07}, Lipmaa and Zhang \cite{SCN:LipZha12}, and also Fauzi and Lipmaa \cite{EPRINT:FauLip15}, considered languages which only depend on the group description and thus no trapdoors that trivialize the zero-knowledge definition exists. However, known impossibility results imply that their shuffle arguments can not be NIZK \cite{TCC:AbeFeh07}. They by-passed the impossibility result proving the weaker notion of \emph{Culpable Soundness}, which means that the soundness adversary must provide a decryption key for the ciphertexts. This bears some similarity with the QA-NIZK case: in the soundness game the decryption key is also known since is sampled by the reduction itself. From a practical point of view, both notions model the fact that some trusted party knows the secret-key.

%In the case of interactive set membership arguments \cite{AC:CamChaShe08,EC:BayGro13}, since no trapdoor is needed in the simulation, they are able to prove \emph{honest-verifier zero-knowledge}. In the non-interactive case \cite{PAIRING:RiaKohPre09} or when removing interaction via the \emph{Fiat-Shamir heuristic}, the simulator do have acces to trapdoors. However, since all the previously cited works consider Pedersen commitments, which are perfectly hiding,  

    \subsection{Organization}
        The remainder of this paper is organized as follows. In Sect. \ref{secc:prelim} we introduce the notation, assumptions, and cryptographic primitives used in this work. In Sect. \ref{sec:mp-comm} we introduce the generalized Multi-Pedersen commitment scheme.
In Sect. \ref{sec:bits} we construct a QA-NIZK argument that a two-dimensional Multi-Pedersen commitment open to a bit-string, and in Sect. \ref{sec:bits-n} we show how to aggregate many of this proofs (this is necessary for obtaining more efficient applications).
In Sect. \ref{sec:bin-lang} we present the aggregated NIZK set membership argument. Finally, in Sects. \ref{sec:shuffle} and \ref{sec:range-proof} we present our Shuffle and Range Argument, respectively.

\section{Preliminaries} \label{secc:prelim}
    \input{prelim/prelim}
    \subsection{Decisional Assumptions} 
        
\begin{definition}   \label{def:matrixdef}
Let $\ell,k \in \N$.
We call $\dist_{\ell,k}$ a matrix distribution if it outputs (in poly time, with overwhelming probability) matrices in $\Z_q^{\ell \times k}$. We define $\dist_k := \dist_{k+1,k}$. 
\end{definition}

For the following decisional assumption to hold, it is a necessary condition that $\ell>k$. However, in other contexts, we might need $\dist_{\ell,k}$ distributions where 
$\ell \geq k$. 

\begin{definition}[Matrix Diffie-Hellman Assumption in $\GG_{\gamma}$, $\gamma \in \{1,2\}$ \cite{C:EHKRV13}]\label{def:mdh}
Let $\dist_{\ell,k}$ be a matrix distribution  and $\gk\gets \ggen_a(1^\lambda)$. We say that the $\dist_{\ell,k}$-Matrix Diffie-Hellman ($\dist_{\ell,k}$-$\mddh_{\GG_\gamma}$)
Assumption holds relative to $\ggen_a$ if for all PPT adversaries $\advD$,
\begin{eqnarray*}
\adv_{\dist_{\ell,k},\ggen_a}(\advD) & := &
    \left|
        \Pr[\advD(\group,[\matr{A}]_\gamma,[\matr{A}\vecb{w}]_\gamma)=1]-
        \Pr[\advD(\group,[\matr{A}]_\gamma, [\vecb{z}]_\gamma) =1]
    \right|
\end{eqnarray*}
is negligible in $k$,
where the probability is taken over $\gk \gets \ggen_a(1^\lambda)$, $\matr{A} \gets \dist_{\ell,k}, \vecb{w} \gets \Z_q^k, [\vecb{z}]_\gamma  \gets \GG_\gamma^{\ell}$ and the coin tosses of adversary $\advD$.
\end{definition}
 

In this paper we will refer to the following matrix distributions: 
\[
\distlin_{k}:\matrA = \left( \begin{smallmatrix}
    a_1 & 0 &  \ldots & 0 \\
    0 &  a_2 &  \ldots & 0\\
    \tiny{\vdots} &  \tiny{\vdots}  &  \tiny{\ddots} & \tiny{\vdots} \\
    0 & 0 &  \ldots  & a_{k}\\
    1 & 1 & \ldots & 1
\end{smallmatrix} \right),
\ 
\mathcal{U}_{\ell,k}: \matrA = \left( \begin{smallmatrix}
    a_{1,1} &  \ldots & a_{1,k}  \\
    \tiny{\vdots} & \tiny{\ddots}  & \tiny{\vdots} \\
    a_{\ell,1} &  \ldots &  a_{\ell,k} 
\end{smallmatrix} \right),
\]
where $a_i,a_{i,j}\leftarrow \Z_q$, for each $i,j\in[k]$.  The $\distlin_{k}$-$\mddh$ Assumption is the $k$-linear family of Decisional Assumptions
\cite{C:HofKil07,cryptoeprint:2007:074} and corresponds to 
 the Decisional Diffie-Hellman (DDH)
Assumption in $\GG_\gamma$ when $k=1$. The SXDH Assumption states that DDH holds in $\GG_\gamma$ for all $\gamma \in \{1,2\}$. The $\mathcal{U}_{\ell,k}$ Assumption is the \textit{Uniform} Assumption and is the weakest of all assumptions of size $\ell \times k$. 

Further, given any matrix distribution $\dist_{k}$, $m \in \mathbb{N}$ and any $i \in [m]$, we will repeatedly make reference to the distribution $\distink$, which is defined as follows: 
\[ \distzeronk: \matrA = \left(\begin{smallmatrix} \matr{B}\vecb{w}_1 & \ldots &  
  \matr{B}\vecb{w}_{m} & \matr{B}  \end{smallmatrix} \right)  \qquad
\distink:\matrA = \left(\begin{smallmatrix} \matr{B}\vecb{w}_1 & \ldots & \matr{B}\vecb{w}_{i-1} 
& \vecb{z} &  \matr{B}\vecb{w}_{i+1} & \ldots &  
  \matr{B}\vecb{w}_{m} & \matr{B}  \end{smallmatrix} \right) 
\]
where $\matr{B} \leftarrow \dist_{k}$, $\vecb{w}_i \leftarrow \Z_q^k$ and $\vecb{z} \leftarrow \GG^{k+1}$. The following are two trivial properties of the $\distink$ distribution. 

\begin{lemma} Under the $\dist_{k}$-$\mddh$ Assumption in $\GG_\gamma$ , for any $0 < i \leq n$, the distribution of  $[\matrA]_\gamma$ when $\matrA \leftarrow \distzeronk$ and when $\matrA \leftarrow \distink$ are computationally indistinguishable. Further, if $\ell>k$, for any $i>0$, if $\matr{A} \leftarrow \distink$, then with overwhelming probability its ith column is linearly independent of the rest. \label{lemma:dist-i}
\end{lemma}
 \label{sec:dec-ass}
            \subsection{Computational Assumptions}\label{sec:comp-assump}
        Additionally, we will be using the following family  computational assumptions:
\begin{definition}[Kernel Diffie-Hellman Assumption in $\GG_{\gamma}$ \cite{EPRINT:MorRafVil15}]Let  $\gk 
\hspace*{-1pt}
\gets
\hspace*{-1pt}
\ggen_a(1^\lambda)$.
The Kernel Diffie-Hellman Assumption in $\GG_\gamma$  ($\dist_{\ell,k}\mbox{-}\kermdh_{\GG_\gamma}$) says that every PPT Algorithm has negligible advantage in the following  game: given $[\matr{A}]_\gamma$, where $\matrA \gets \dist_{\ell,k}$, find $[\vecb{x}]_{3-\gamma} \in \GG_{3-\gamma}^{\ell}$, $\vecb{x} \neq \vecb{0}$, such that 
$[\vecb{x}]_{3-\gamma}^{\top}[\matr{A}]_{\gamma}=[\vecb{0}]_T$. 
\end{definition}

The  Simultaneous Pairing Assumption in $\GG_\gamma$  (\SP$_{\GG_{\gamma}}$) is the $ \mathcal{U}_1\mbox{-}\kermdh_{\GG_{\gamma}}$ Assumption. The Kernel Diffie-Hellman assumption is a generalization and abstraction of this assumption to other matrix distributions. 
The $\dist_{\ell,k}\mbox{-}\kermdh_{\GG_{\gamma}}$ Assumption is weaker than the $\dist_{\ell,k}\mbox{-}\mddh_{\GG_{\gamma}}$ Assumption, since a solution allows to decide membership in $\rmIm([\matr{A}]_{\gamma})$.

In asymmetric bilinear groups, there is a natural variant of this assumption which was introduced in \cite{AC:GonHevRaf15}.  
\begin{definition}[Split Kernel Diffie-Hellman Assumption]
Let  $\gk \hspace*{-3pt} \gets
\hspace*{-3pt}
\ggen_a(1^\lambda)$.
The Split Kernel Diffie-Hellman Assumption in $\GG_1,\GG_2$  ($\dist_{\ell,k}\mbox{-}\skermdh$) says that every PPT Algorithm has negligible advantage in the following  game: given $([\matr{A}]_1,[\matr{A}]_2)$, $\matr{A} \leftarrow \dist_{\ell,k}$, find a pair of vectors $([\vecb{r}]_1,[\vecb{s}]_2) \in \GG_1^{\ell} \times \GG_2^{\ell}$, $\vecb{r} \neq \vecb{s}$, such that 
$[\vecb{r}]_1^{\top}[\matr{A}]_2=[\vecb{s}]_2^{\top}[\matr{A}]_1$. 
\end{definition}

While the Kernel Diffie-Hellman Assumption says one cannot find a non-zero vector in one of the groups which is in the co-kernel of $\matr{A}$, the split assumption says one cannot find a pair of vectors in $\GG_1^{\ell} \times \GG_2^{\ell}$ such that the difference of the vector of their discrete logarithms is in the co-kernel of $\matr{A}$. 
As a particular case we consider the \emph{Split Simultaneous Double Pairing Assumption in} $\GG_1,\GG_2$ ($\SSDP$) which is the $\distrlin_{2}\mbox{-}\skermdh$ Assumption, where 
$\distrlin_{2}$ is the distribution which results of sampling a matrix from $\distlin_{2}$ and replacing the last row by random elements. 


%    \subsection{Commitment Schemes} \label{sec:comm}
%         %We will be using two kinds of commitment schemes: Groth-Sahai commitments \cite{EC:GroSah08} and 

\begin{definition} A commitment scheme is a tuple of three algorithms 
	$(\algK,\Com,\algVrfy)$ such that:
	\begin{itemize} 
		\item  $\algK$ is a randomized algorithm, which on input a group key $gk$ outputs a commitment key $ck$,
		\item $\Com$ is a randomized algorithm which, on input the commitment key $ck$ and a message 
		$m$ in the message space $\mathcal{M}_{ck}$ outputs a commitment $c$ in the commitment space $\mathcal{C}_{ck}$ and an opening $Op$, 
	\item $\algVrfy$ is a deterministic algorithm which, on input the commitment key $ck$,  a message 
	$m$ in the message space $\mathcal{M}_{ck}$ and an opening $Op$, outputs $1$ if $Op$ is a valid opening of $c$ to the message $m$ and $0$ otherwise. 
	\end{itemize}
\end{definition}

Correctness requires that for any $m \in \mathcal{M}_{ck}$
$$\Pr\left[ck \gets \algK(gk); m \gets \mathcal{M}_{ck}; 
(c, Op) \gets \Com_{ck}(m): \algVrfy(ck,c,m,Op)=1 \right]=1.$$


\begin{definition}  A commitment scheme is computationally binding (resp. perfectly binding) if, for any polynomial-time (resp. unbounded)  adversary $\advA$, 
	$$\Pr\left[ck \gets \algK(gk); (c,m,Op,m',Op') \gets \advA(gk,ck): \begin{array}{l}\algVrfy(ck,c,m,Op)=1 \wedge\\ \algVrfy(ck,c,m',Op')=1 \wedge m\neq m'\end{array}\right] $$
	is negligible (resp. zero).  It is computationally hiding (resp. perfectly hiding)  if, for any polynomial-time (resp. unbounded) adversary $\advA$,
	$$\left|\Pr\left[ \begin{array}{l} ck \gets \algK(gk); (m_0,m_1,st) \gets \advA(gk,ck); b \gets \{0,1\};\\ (c,Op) \gets \Com_{ck}(m_b); b' \gets \advA(st,c)
                      \end{array} : b'=b\right] -\dfrac{1}{2} \right|$$
	is negligible (resp. zero).
\end{definition}

In this paper we will be using two commitment schemes, one is the GS commitment scheme, and the other is a generalization of the Multi-Pedersen commitment. Multi-Pedersen commitments will be introduced in Sect. \ref{sec:mp-comm} and GS commitments are described below.
%commitment to a vector of scalars is a vector of two group elements, instead of the usual Multi-Pedersen where commitment is only one group element.


%The message space is naturally extended to row-vectors $\vecb{m}\in\Z_q^{1\times m}$  defining $\GS.\Com_{[\matr{U}]_1}(\vecb{m};\vecb{w}):=\pmatri{{[\vecb{m}]_1}\\{[\vecb{0}_{1\times m}]_1}}+[\matr{U}]_1\pmatri{\vecb{m}\\\vecb{w}}\in\GG_1^{2\times m}$, for $\vecb{w}\in\Z_q^{1\times m}$. Similarly, for $[\vecb{m}]_1\in\GG_1^{1\times m}$, $\GS.\Com_{[\matr{U}]_1}([\vecb{m}]_1;\matr{W}):=\pmatri{{[\vecb{m}]_1}\\{[\vecb{0}_{1\times m}]_1}}+[\matr{U}]_1\matr{W}\in\GG_1^{2\times m}$, for $\matr{W}\in\Z_q^{2\times m}$. 



    \subsection{Groth-Sahai NIZK Proofs} \label{GSproofs} \label{sec:gs}
        The GS proof system allows to prove satisfiability of a set of quadratic equations in a bilinear group. The admissible equation types must be in the following form:
\begin{equation}\label{gseq}
\sum_{j=1}^{m_y} f(\alpha_j, \vary_j)+\sum_{i=1}^{m_x} f(\varx_i, \beta_i)+\sum_{i=1}^{m_x} \sum_{j=1}^{m_y}  f(\varx_i,\escQE_{i,j} \vary_j)=t,
\end{equation}
 where $\boldsymbol \alpha  \in \Am_1^{m_y}$, $\boldsymbol \beta  \in \Am_2^{m_x}$, $\matr{\EscQE}=(\escQE_{i,j}) \in \Z_q^{m_x\times m_y}$, $t \in \Am_T$, and $\Am_1,\Am_2,\Am_T$ are $\Z_q$-vector spaces equipped with some bilinear map $f:\Am_1\times \Am_2 \rightarrow \Am_T$. When $t=f(t_1,1)$ or $t=f(1,t_2)$, for some $t_1\in\Am_1$ or $t_2\in\Am_2$, we say that the equation allows simulation (see \cite{SIAM:GroSah12}). The modules and the map $f$ can be defined in different ways as: (a) in pairing-product equations (PPEs), $\Am_1=\GG_1$, $\Am_2=\GG_2$, $\Am_T=\GG_T$, $f([x]_1,[y]_2)=e([x]_1,[y]_2) \in \GG_T$, in which case $t=0_{T}$, (b1) in multi-scalar multiplication equations in $\GG_1$ (MMEs), $\Am_1=\GG_1$, $\Am_2=\Z_q$, $\Am_T=\GG_1$, $f([x]_1,y)=y [x]_1 \in \GG_1$,  (b2) MMEs in $\GG_2$ (MMEs),  $\Am_1=\Z_q$, $\Am_2=\GG_2$, $\Am_T=\GG_2$, $f(x,[y]_2)=x [y]_2 \in \GG_2$, and (c) in quadratic equations in $\Z_q$ (QEs), $\Am_1=\Am_2=\Am_T=\Z_q$, $f(x,y)=xy \in \Z_q$. 
 %An equation is linear if $\matr{\EscQE}=\vecb{0}$, 
 %it is \textit{two-sided linear} if both $\boldsymbol \alpha \neq \vecb{0}$ and $\boldsymbol \beta \neq \vecb{0}$, and \textit{one-sided} otherwise.

Before givng more details on GS proofs, we introduce the GS commitment scheme
\begin{definition} The Groth-Sahai commitment scheme in the SXDH instantiation in the group $\GG_\gamma$, $\gamma\in\{1,2\}$, is  specified by the following three algorithms 
	$(\mathsf{GS}.\algK,\mathsf{GS}.\Com,\mathsf{GS}.\algVrfy)$ such that:
	\begin{itemize} 
		\item  $\mathsf{GS}.\algK$ is a randomized algorithm, which on input the group key $gk$ and the (optional and if not given assumed to be $\mathsf{true}$) flag $\mathsf{binding}$, outputs a commitment key $ck:=[\matr{U}]_\gamma=[(\vecb{u}_1||\vecb{u}_2)]_\gamma \in\GG_\gamma^{2\times 2}$. It samples $\vecb{u}_2\gets\distlin_{1}$ and $\mu\gets\Z_q$, and, 
if $\mathsf{binding}=\mathsf{true}$, $[\matr{U}]_\gamma$ is the {perfectly binding key} and $\vecb{u}_1:=\mu\vecb{u}_2$ and else it is the {perfectly hiding key} and $\vecb{u}_1:=\mu\vecb{u}_2-\vecb{e}_2$.
		\item $\mathsf{GS}.\Com$ is a randomized algorithm which, on input a commitment key $ck=\bmatr{U}_\gamma$, and a message 
		$\mathsf{m}$ in the message space $\mathcal{M}_{ck}=\Z_q\cup\GG_\gamma$, it proceeds as follows. If $\mathsf{m}=m\in\Z_q$, it samples $r \gets \Z_q$ and outputs a commitment $\bvecb{c}_\gamma := m[\vecb{e}_2+\vecb{u}_1]_\gamma+r[\vecb{u}_2]_\gamma$ in the commitment space $\mathcal{C}_{ck}=\GG_\gamma^2$ and an opening $Op=r$. If $\mathsf{m}=[m]_\gamma\in\GG_\gamma$, it samples $\vecb{r} \gets \Z_q^2$ and outputs a commitment $\bvecb{c}_\gamma := [m]_\gamma\vecb{e}_2+[\matr{U}]_\gamma\vecb{r}$ in the commitment space $\mathcal{C}_{ck}=\GG_\gamma^2$ and an opening $Op=\vecb{r}$.
		\item $\mathsf{GS}.\algVrfy$ is a deterministic algorithm which, on input the commitment key $ck=\bmatr{U}_\gamma$, a commitment $\bvecb{c}_\gamma$,  a message 
		$m \in \mathcal{M}_{ck}$ and an opening $Op$, outputs $1$ if $\bvecb{c}_\gamma=\GS.\Com_{ck}(m;Op)$
		and $0$ otherwise.
	\end{itemize}
\end{definition}

\begin{theorem}[\cite{SIAM:GroSah12}] If $ck\gets\algK(gk)$ (resp. $ck\gets\algK(gk,\mathsf{false})$) the Groth-Sahai commitment scheme is perfectly binding (resp. computationally binding) and computationally hiding (resp. perfectly hiding).
\end{theorem}



The GS proof system consists of five algorithms $\GS:=(\algK,\algP,\algV,\algS_1,\algS_2)$. Below we describe the properties which are relevant to this work.
\begin{description}
\item[$\algK$:] The CRS generator, on input a group key $\gk$ outputs the \emph{perfectly binding CRS}.  In the SXDH instantiation the CRS is $\crs_\GS:=(\gk,ck_1,ck_2)$, 
$ck_1=([\vecb{u}_1]_1,[\vecb{u}_2]_1)$,
 $ck_2=([\vecb{v}_1]_2,[\vecb{v}_2]_2)$, 
 where $\vecb{u}_2,\vecb{v}_2 \gets \distlin_1$, $\vecb{u}_1:=\mu \vecb{u}_2 $, $\vecb{v}_1:=\epsilon \vecb{v}_2$, $\mu,\epsilon\gets\Z_q$.
\item[$\algP$:] The prover, on input an equation $\eq$ of the form (\ref{gseq}) and solution $\varx_i,\vary_j$, $i\in[m_x],j\in[m_y]$, outputs a proof $\pi=(\{\GS.\Com_{ck_1}(\varx_i):i\in[m_x]\},\{\GS.\Com_{ck_2}(\vary_j):j\in[m_y]\},\pi')$.
\item[$\algV$:] The verifier, on input an equation $\eq$ of the form (\ref{gseq}) and a proof $\pi$, accepts or rejects the proof for $\eq$.
\item[$\algS_1$:]  The CRS simulator, on input a group description $\gk$ outputs the \emph{perfectly hiding CRS}  and a trapdoor. In the SXDH instantiation, the CRS is identical to the one output by $\algK$, except that $\vecb{u}_1:=\mu \vecb{u}_2-\vecb{e}_1^\top$ and $\vecb{v}_1:=\epsilon \vecb{v}_2-\vecb{e}_1^\top$, and the simulation trapdoor is $(\mu,\epsilon)$.
\item[$\algS_2$:] The prover simulator, on input an equation $\eq$ of the form (\ref{gseq}) which allows simulation, and a trapdoor $\tau$, outputs a simulated proof $\pi$.
\end{description}

    \subsection{Quasi-Adaptive NIZK Arguments} \label{zkdef} \label{sec:zkdef}
        A Quasi-Adaptive NIZK proof system \cite{AC:JutRoy13} enables
to prove membership in a language defined by a relation $\R_\rho$, which in turn is completely determined by some parameter
$\rho$ sampled from a distribution $\dist_\gk$.
We say that $\dist_\gk$ is \emph{witness samplable} if there exists an efficient
algorithm that samples $(\rho,\omega)$ from a distribution $\dist_\gk^{\mathsf{par}}$ such that $\rho$ is distributed according to $\dist_\gk$, and membership of $\rho$
in the \emph{parameter language} $\Lang_\mathsf{par}$ can be efficiently verified with $\omega$.
While the Common Reference String can be set based on $\rho$, the zero-knowledge simulator is required to be a single probabilistic polynomial time
algorithm that works for the whole collection of relations $\R_\gk$. 

\iffull
    \input{prelim/app-qanizk}
\else
    The details of the QA-NIZK definition can be found in Appendix \ref{app:qanizk}.
\fi

        \subsubsection{QA-NIZK Argument for Linear Subspaces.} \label{Sect:QANIZKlinspace}
            In this section we describe the languages for which there exist constant-size QA-NIZK arguments of membership which will be used as building blocks in our constructions.  These languages are (i) linear subspaces of $\GG_s^m$, $s \in \{1,2\}$ \cite{C:JutRoy14,EC:KilWee15,EC:LPJY14}, (ii) linear subspaces of $\GG_1^m\times\GG_2^n$ \cite{AC:GonHevRaf15}, (iii) equal commitment opening \cite{AC:GonHevRaf15}, and (iv) \emph{sum in subspace} \cite{AC:GonHevRaf15}. More specifically, the languages are defined as follows for $\gamma,\nu \in \{1,2\}$, 
\begin{align}
&\Lang_{[\matr{M}]_\gamma}:=\{ [\vecb{x}]_{\gamma} \in \GG_{\gamma}^{n}:  \exists \vecb{w} \in \Z_q^{t}, \  \vecb{x}=\matr{M}\vecb{w} \}, \tag{i}\\
&\Lang_{[\matr{M}]_1,[\matr{N}]_2}:=\{ ([\vecb{x}]_1,[\vecb{y}]_2) \in \GG_1^{m}\times\GG_2^n:  \exists \vecb{w} \in \Z_q^{t}, \  \vecb{x}=\matr{M}\vecb{w}, \   \vecb{y}=\matr{N}\vecb{w} \}, \tag{ii}\\
&\Lang_{ck,ck',\sfcom}:=\left\{([\vecb{c}]_\gamma,[\vecb{d}]_\nu):\exists (\vecb{w}, \vecb{r},\vecb{s}) 
\in \Z_q^n \times \Z_q^{t_1} \times \Z_q^{t_2},  
\begin{array}{l}
    {[\vecb{c}]}_\gamma=\Com_{ck}(\vecb{w};\vecb{r}),\\
    {[\vecb{d}]}_\nu=\Com_{ck'}(\vecb{w};\vecb{s})
\end{array}
\right\}, \tag{iii}\\
&\Lang_{[\matr{M}]_1,[\matr{N}]_2,\sfsum}:=\{([\vecb{x}]_1,[\vecb{y}]_2):\exists\vecb{w}, \vecb{x}+\vecb{y}=(\matr{M}+\matr{N})\vecb{w}\}. \tag{iv}
\end{align}
In the above definitions, $\matr{M} \in \Z_q^{m \times t_1}$,
  $\matr{N} \in \Z_q^{n \times t_2}$ and $ck$ (resp. $ck$') defines some commitments to vectors of $\Z_q^{n}$ where the randomness space is $\Z_q^{t_1}$ (resp. $\Z_q^{t_2}$). In (iv), $t_1=t_2$. The commitment scheme $\Com$ is assumed to be of the form 
 $\Com_{ck}(\vecb{w};\vecb{r})=[\matr{A}]_\gamma \vecb{w}+[\matr{B}]_\gamma \vecb{r}$, 
 for some matrices $[\matr{A}]_\gamma,[\matr{B}]_\gamma$ defined in 
  $ck$. 

We denote indistinctly by $\Pi_\sflin$ the proof systems for (i) and (ii), by $\Pi_\sfcom$ the proof system for (iii), and by $\Pi_\sfsum$ the proof system for (iv).

 To compute the proof sizes of our constructions, we will use the most efficient instantiations for each of these languages, which are described in Table \ref{table:lin-sp}. We note that the argument of \cite{AC:GonHevRaf15} for $\Lang_{ck,ck',\sfcom}$ is for the case $\gamma=1,\nu=2$. It is not hard to see that when $\nu=\gamma$, membership in $\Lang_{ck,ck',\sfcom}$ (for commitments of the form we specified) amounts to prove membership in some linear space in $\GG_\gamma$, which explains the second row of the table.

\iffull
In Sect. \ref{app:proofbits} we require this constructions to satisfy a stronger notion of soundness, as defined in Sect. \ref{sec:zkdef}. It have been proven that the same constructions satisfy this stronger notion of soundness \cite{EPRINT:GonHevRaf15}.
\fi


\begin{table}[h]
\begin{center}
\begin{minipage}{\textwidth}
\begin{center}
\begin{tabular}{|l|lll|lll|}
\hline
Proof system                                & Language                            & Proof Size & Assumption \\
\hline\hline
\cite{EC:KilWee15}, \cite{C:JutRoy14}  & $\Lang_{[\matr{M}]_s}$ & $|\GG_{\gamma}|$          & $\SP_{\GG_{3-\gamma}}$ \\
  & $\Lang_{ck,ck',\com}, \gamma =\nu$ & $|\GG_{\gamma}|$          & $\SP_{\GG_{3-\gamma}}$ \\
\hline
                                            & $\Lang_{[\matr{M}]_1,[\matr{N}]_2}$      & (2,2)      & $\SSDP$ \\
\cite{AC:GonHevRaf15}                       & $\Lang_{ck,ck',\com}, \gamma \neq \nu$                    & (2,2)      & $\SSDP$ \\
                                            & $\Lang_{[\matr{M}]_1,[\matr{N}]_2,\sfsum}$ & (2,2)      & $\SSDP$\\
\hline
\end{tabular}
\end{center}
\caption{QA-NIZK arguments for linear subspaces used in this work. When the proof size is given by $(a,b)$ it means $a$ elements of $\GG_1$ and $b$ elements of $\GG_2$, otherwise $|\GG_{\gamma}|$ means one element of $\GG_{\gamma}$.  \label{table:lin-sp}} 
\end{minipage}
\vspace{-0.54cm}

\end{center}
\end{table}





%The proof system for (i) is instantiated under the $\SP$ assumption and the proof consists of one element of $\GG_1$. Proof systems for (ii), (iii), and (iv), are instantiated under the $\SSDP$ assumption and the proof consists of 2 elements of each $\GG_1$ and $\GG_2$.
%
%While (iii) is written for any commitment scheme, the proof system can be build only if $\Com_{ck}(\vecb{w};r)=[\matr{M}]_i\vecb{w}+[\matr{N}]_i\vecb{r}$, for some $[\matr{M}]_i\in\GG_i^{m\times n}$, $[\matr{N}]_i\in\GG_i^{m\times t}$, $i\in\{1,2\}$.

% Although the proof system for (iii) from \cite{AC:GonHevRaf15} only considers the case $i\neq j$, it is direct to extend the proof system to the case $i=j$ and in fact, when $i=j$, the proof consists of only one element of $\GG_i$ when security is proven under the $\SP$ assumption (and not the $\SSDP$ assumption). 
%
%In Appendix XXX we prove that this proof systems satisfy the stronger notion of soundness defined in Sect. \ref{sec:zkdef}. 

\section{Extended Multi-Pedersen Commitments} \label{sec:mp-comm}
    \input{mp-comm/mp-comm}
\section{QA-NIZK For Bit-Strings, Revisited} \label{sec:bits}
    \begin{description}


\item[$\algK_1(\Gamma, \Lck,\Rck,\{\hmatr{W}_{i,j}\}_{(i,j) \in \indexSet{n}{1}})$]

Let $\QANIZKsum$ be the proof system for Sum in Subspace 
(Sect. \ref{sec:QANIZKsum}) and $\QANIZKcomms$
%=(\algK_0,\algK_1,\algP,\algV,\algS_1,\algS_2)$
be an instance of our proof system for Equal Opening (Sect. \ref{sec:aggcomms}).

Let
$\crs_\QANIZKsum \gets \algK_1(\Gamma, \{\hmatr{C}_{i,j},\cmatr{D}_{i,j}\}_{(i,j)\in\indexSet{n}{1}})$.\footnote{We identify
matrices in $\Gr^{2 \times 2}$ (resp. in $\Hr^{2 \times 2}$) with vectors in $\Gr^{4}$ (resp. in $\Hr^{4}$).} Generate $\crs_\QANIZKcomms \gets \algK_1(\Gamma, \Lck,\Rck,n)$. 

The common reference string is given by:
\begin{eqnarray*}
\mathsf{crs}_P&:=&\left(  \Lck,
    \check{\Lrck}, \{\matr{W}_{i,j} \}_{(i,j) \in \indexSet{n}{1}},\crs_\QANIZKsum,\crs_\QANIZKcomms \right).
 \end{eqnarray*}
\item[$\algP(\mathsf{crs}_P, \hvecb{c}, \langle \vecb{b}, w_g \rangle)$:]
Pick $w_h \gets \Z_q$,  $\matr{R} \gets \Z_q^{2\times 2}$ and then: 
\begin{enumerate}
\item Define 
$$\cvecb{d} := \Rck \begin{pmatrix} \vecb{b} \\ w_h \end{pmatrix}.$$ 
\item Compute 
 $(\hmatr{\Theta}, \cmatr{\Pi})\, := \sum_{i \in [n]}
 b_i w_h  \matr{W}_{i,n+1}+
 w_g(b_i-1) \matr{W}_{n+1,i}
 $
\begin{eqnarray} \label{eq:ThetaPi}
& &
    +
       \sum_{i \in [n]}  \sum_{\substack{j \in [n]\\ j\neq i}} b_i (b_j-1) \matr{W}_{i,j}  +
       w_gw_h \matr{W}_{n+1,n+1} +  (\hmatr{R},-\cmatr{R}).
 \end{eqnarray}ƒ

\item Compute a proof $(\hat{\boldsymbol \rho}_1,\check{\boldsymbol \sigma}_1)$
that $\matr{\Theta}+\cmatr{\Pi}$ is in the span of 
$\{\matr{C}_{i,j}+\matr{D}_{i,j}\}_{(i,j)\in\indexSet{n}{1}}$.
and a proof $(\hgrkb{\rho}_{2}, \cgrkb{\sigma}_{2})$
that
$(\hvecb{c},\cvecb{d})$ open to the same value,
using $\vecb{b},w_g$, and $w_h$. \\
\end{enumerate}

\item[$\algV(
    \mathsf{crs},
    \hvecb{c},
   \cvecb{d},
        (\hmatr{\Theta}, \cmatr{\Pi}), 
        \{(\hat{\boldsymbol \rho}_{i}, \check{\boldsymbol \sigma}_{i})\}_{i \in [2]} )$:] ~
%   
\begin{enumerate}
\item Check if 
\begin{equation}\label{eq:ver1}\hvecb{c}
\left(
    \check{\vecb{d}}-
    \sum_{j \in [n]} \rck_{j}
\right)^{\top} =
    \hmatr{\Theta} \cmatr{I}_{2 \times 2} +
    \hmatr{I}_{2 \times 2}\cmatr{\Pi}.
    \end{equation}  
  \item Verify that $(\hgrkb{\rho}_1, \cgrkb{\sigma}_1),(\hgrkb{\rho}_{2},\cgrkb{\sigma}_2)$ are valid proofs for %\linebreak 
  $(\hmatr{\Theta},$ $\cmatr{\Pi})$
        and $(\hvecb{c},\cvecb{d})$ using $\crs_\QANIZKsum$ and $\crs_\QANIZKcomms$ respectively.
\end{enumerate}
If any of these checks fails, the verifier outputs $0$, else it outputs $1$. \\

%The simulators $\algS_1$ and $\algS_2$ are defined as follows.
%\begin{description}
%\item[$\mathsf{S}_1(\Gamma,\hat{\matr{U}})$:] The simulator receives as input a description of an asymmetric bilinear group $\Gamma$ and a matrix $\hat{\matr{U}} \in \Gr^{(n+1) \times (n+1)}$ sampled according to distribution $\dist_{\Gamma}$. It generates and outputs the CRS in the same way as $\algK_1$, but additionally it also  outputs the simulation trapdoor 
%$$\tau=\left(\Lrck,\tau_\QANIZKsum, \tau_\QANIZKcomms\right),$$
%where $\tau_\QANIZKsum$ and $\tau_\QANIZKcomms$ are, respectively, $\QANIZKsum$'s and $\QANIZKcomms$'s simulation trapdoors.
\item[$\mathsf{S}_2(\crs,\vc,\left(\Lrck,\tau_\QANIZKsum, \tau_\QANIZKcomms\right))$:]  Given the matrix $\mat{H}$ of discrete logarithms of $\cmatr{H}$, $\tau_\QANIZKsum$ and $\tau_\QANIZKcomms$ which are, respectively, $\QANIZKsum$'s and $\QANIZKcomms$'s simulation trapdoors, this algorithm samples 
$\overline{w}_h \gets \Z_q$, $\matr{R} \gets \Z_q^{2 \times 2}$ and defines 
 $\vecb{d}:= \overline{w}_{h} \lrck_{n+1}.$
 Then it sets:
\begin{align*} 
\hmatr{\Theta} & :=  \hvecb{c}\left(\vecb{d}-\sum_{i \in [n]} \lrck_i\right)^\top + \hmatr{R},
    &
    \cmatr{\Pi} & := - \cmatr{R}.
%\\
%\hmatr{\Theta}_{b-\overline{b}} & :=  \hvecb{c}_\Delta \left( \sum_{i\in[n]}\lrck_{i}^\top \right) -
%                          \left( \sum_{i\in[n]}\hvecb{u}_{i} \right)\vecb{d}^\top + \hmatr{S},
%    &
%    \cmatr{\Pi}_{b-\overline{b}}& := - \cmatr{S}.
\end{align*}
Finally, it simulates proofs $(\hat{\boldsymbol \rho}_{X},
  \check{\boldsymbol \sigma}_{X})$
%for $(\hmatr{\Theta}_{X}, \cmatr{\Pi}_{X})$
for $X \in \{1,2\}$  using $\tau_\QANIZKsum$ and $\tau_\QANIZKcomms$.
\end{description}

\begin{theorem} Algorithms satisfy that...   
\begin{itemize}
 \item[a)] Completeness:
\item[b)] $1$-coordinate soundness: if there exists an index $i^*$ such that $\hvecb{g}_{i^*} \notin \mathsf{Span}(\{\hvecb{g}_{i} : i \neq i^*\})$ and $\hvecb{h}_{i^*} \notin \mathsf{Span}(\{\hvecb{h}_{i} : i \neq i^*\})$, then:
\begin{multline*}
\Pr\Big[ b_{i^*} \notin \{0,1\} \cap \Big(\exists (b_{i^*}, \vecb{w}): \ \hvecb{c}= b_{i^*} \hvecb{g}_{i^*} + \sum_{j \in [n+1], j\neq i^*} w_{j} \hvecb{g}_{j}  \Big)  \cap  \\ \algV(
\mathsf{crs},\vecb{c},\mathsf{proof})=1 | (\vecb{c},\mathsf{proof}) \gets \advA(\mathsf{crs})     \Big]=0$$
\end{multline*}
\item[c)] Perfect Zero-Knowledge: if xxxx 
\end{itemize}
\end{theorem}	
\begin{proof}
Completeness follows from XXX. Perfect Soundness follows from XXX. Perfect ZK follows from XXX.
\end{proof}
  
    \subsection{Constant-Size Argument for $\Lang_{ck,\sfbits}^n$} \label{sec:bits-n}
        We give a QA-NIZK argument of membership in the language $\Lang_{ck,\sfbits}^n  = \Lang_{ck,\sfbits} \times \ldots \times \Lang_{ck,\sfbits}$ with a proof size which is independent of $n$ (but with a loss factor in the proof of soundness of $n$). The result will be crucial to get improved proof sizes for more complex statements. More specifically, we prove: 

\begin{theorem} \label{theo:bitsnm} There exists a QA-NIZK argument $\Pi_\sfbitsn$  for membership in $\Lang_{ck,\sfbits}^n$ with proof size  
$10|\GG_1|+10|\GG_2|$ with perfect completeness, perfect-zero knowledge and computational soundness. 
\end{theorem}

\iffull
\subsection{Intuition}
 \input{bits/matr-bits-intuition.tex}
\fi

The description of the protocol and the full proof of Theorem \ref{theo:bitsnm} are in Appendix~\ref{app:bits-n}.



       

\section{Aggregated NIZK Set Membership Arguments} \label{sec:bin-lang}
    
In this section we construct a QA-NIZK argument that many commitments open to elements in a set $[0,d-1] \subset \Z_q$ or  $S \subset \GG_{\gamma}$. With the purpose of expressing the problem in a unified way, 
we define a broader class of languages which comprises both these languages as a special case. 


\begin{definition}
Denote by $\Lang_{[\matr{M}]_1,[\matr{N}]_1,\matr{\Lambda},\grkb{\alpha}}\subseteq\GG_1^\la$ the language parameterized by $[\matr{M}]_1 \in\GG_1^{\la\times \lb},\matr{N}\in\GG_1^{\la\times \lc},\matr{\Lambda}\in\Z_q^{\ld\times\lb},$ and $\grkb{\alpha}\in\Z_q^\ld$ such that
\begin{equation}\label{eq:definition1}
[\vecb{x}]_1\in\Lang_{[\matr{M}]_1,[\matr{N}]_1,\matr{\Lambda},\grkb{\alpha}} \Longleftrightarrow \exists \vecb{b}\in\bits^\lb,\vecb{w}\in\Z_q^\lc \text{ s.t. }
\pmatri
{
    \vecb{x}\\
    \grkb{\alpha}
}
=
\begin{pmatrix}
    \matr{M}       & \matr{N}\\
    \matr{\Lambda} & \matr{0}_{\ld\times \lc}
\end{pmatrix}
\pmatri
{
    \vecb{b}\\
    \vecb{w}
}.
\end{equation}
Additionally, we require $(\matr{N},[\matr{N}]_1)$ to be efficiently samplable and that membership in $\Lang_{[\matr{M}]_1,[\matr{N}]_1,\matr{\Lambda},\grkb{\alpha}}$ is efficiently testable with the trapdoor $\matr{N}$, that is, that there exists an efficient algorithm $\algF$ such that $\algF([\matr{M}]_1,\matr{N},[\vecb{x}]_1)=1\Longleftrightarrow [\vecb{x}]_1\in\Lang_{\matr{M},\matr{N},\matr{\Lambda},\grkb{\alpha}}.$ The witness of 
$[\vecb{x}]_1\in\Lang_{[\matr{M}]_1,[\matr{N}]_1,\matr{\Lambda},\grkb{\alpha}}$ is $(\vecb{b},\vecb{w})$, and the bit-witness is $\vecb{b}$. The size of the bit-witness is $\lb$. 
\end{definition}

These languages easily express languages of set membership as shown below.


\begin{example}
The language of GS commitments to group elements in the set $S:=\{[s_1]_1,\ldots,[s_\lb]_1\}$, $\Lang_{ck,S}$, where $ck:=([\vecb{u}_1]_1||[\vecb{u}_2]_1)$, is equal to $\Lang_{[\matr{M}]_1,[\matr{N}]_1,\matr{\Lambda},\grkb{\alpha}}$, where
$\matr{M}:=\smallpmatrix{s_1 & \cdots & s_\lb \\ 0 &\cdots & 0}$, $\matr{N}:=(\vecb{u}_1-\vecb{e}_1||\vecb{u}_2)$, $\alpha=1$, and $\matr{\Lambda}=(1,\ldots,1)$. The bit-witness size is $|S|$ and membership $\Lang_{ck,S}$ is efficiently testable given $\vecb{u}_1,\vecb{u}_2$ and whenever $|S|=\poly(\lambda)$. 
\end{example}

Although in the previous example $S$ could be an integer interval $[0,d-1]$, this case allows a smaller witness size.

\begin{example}
The language of GS commitments to integers in the range $[0,d-1]$, $\Lang_{ck,[0,d-1]}$, where $ck:=([\vecb{u}_1]_1||[\vecb{u}_2]_1)$, is equal to $\Lang_{[\matr{M}]_1,[\matr{N}]_1,\matr{\Lambda},\grkb{\alpha}}$, where
$\matr{M}:=\vecb{u}_1(2^0,2^1,\ldots,2^{\log d-1})\in\Z_q^{2\times \log d}$, $\matr{N}:=\vecb{u}_2\in\Z_q^{2}$, and $\ld:=0$. The bit-witness size is $\log d$ and membership in $\Lang_{ck,[0,d-1]}$ is easily testable given $\vecb{u}_2$ and whenever $d=\poly(\lambda)$. 
\end{example}

In both examples, one may think of $[\vecb{x}]_1$ as a commitment to $\vecb{b}$ using commitment key $ck:=([\matr{M}]_1||[\matr{N}]_1)$. Although in general this may be false, if one thinks in this way is easy to construct a proof of size $\Theta(\lb)$ for this language,  which roughly follows these steps:
\begin{enumerate}
\item Construct a GS commitment $[\vecb{c}_j]_1$ to $b_j$, for all $j \in [\lb]$.
\item Prove that $b_j \in \{0,1\}$, for all $j \in [\lb]$.
\item Prove that $[\vecb{c}]_1:=[\vecb{c}_1]_1 \oplus \ldots \oplus [\vecb{c}_n]_1$ and $[\vecb{x}]_1$ open to the same value.
\item Prove that $\pmatri
{
    [\vecb{x}]_1\\
    [\grkb{\alpha}]_1
}$ is in the image of the matrix $
\begin{pmatrix}
    [\matr{M}]_1       & [\matr{N}]_1\\
    [\matr{\Lambda}]_1 & [\matr{0}_{\ld\times \lc}]_1.
    \end{pmatrix}.$
\end{enumerate}

Steps 3, 4 can be done in constant-size with one of the QA-NIZK arguments we gave in Section \ref{Sect:QANIZKlinspace}. Step 2 can be done in constant-size with the results of \cite{AC:GonHevRaf15}. The total size of the proof is of size $2\lb|\GG_1|+\Theta(1)$, though, because 
of Step 1.\footnote{It would be tempting to try to construct a constant-size proof by replacing in Step 1 the GS commitments by a MP commitment to $\vecb{b}$ and Step 2 by a proof that the resulting commitment opens to a bit-string with the proof system $\mathcal{L}_{ck,\sfbits}.$ However, we do not know how to prove soundness in this case.}

    \subsection{QA-NIZK Argument of Membership in $\Lang_{\matr{M},\matr{N},\matr{\Lambda},\grkb{\alpha}}^n$} \label{sec:bin-lan-constr}
       The main result of this Section is a proof, of the same size $2\lb|\GG_1|+\Theta(1)$, that several statements $[\vecb{x}_1]_1,\ldots,[\vecb{x}_n]_1$ are in $\Lang_{[\matr{M}]_1,[\matr{N}]_1,\matr{\Lambda},\grkb{\alpha}}$.

Similarly as before, one may think of $[\vecb{x}_1]_1,\ldots,[\vecb{x}_n]_1\in\LangMN$ as commitments to $\vecb{b}_1,\ldots,\vecb{b}_n\in\bits^n$. Let $\matr{B}:=(\vecb{b}_1||\cdots||\vecb{b}_n)\in\bits^{\lb\times n}$. To get a proof of size independent of $n$ we commit to $\matr{B}$ ``compressing the rows'', that is, computing a MP commitment to each row of $\matr{B}$.\footnote{Similarly as before, it would be tempting to try to construct a constant-size proof ``compressing also the columns'' of $\matr{B}$. But, again, we do not know how to prove soundness in this case.}
%We prove that by changing the distibution of the MP commitment key in a clever way we get a proof of size independent of $n$ and prove soundness.
 
%We already announced that one runs into difficulties in the soundness proof when trying to replace the GS commitments by MP commitments to get a more compressed proof of $[\vecb{x}]_1 \in \Lang_{[\matr{M}]_1,[\matr{N}_1],\matr{\Lambda},\grkb{\alpha}}$. However for $n$ proofs we can use this idea in combination with changing the distribution of the MP commitment key in a clever way to get a proof of size independent of $n$. 

For all $j \in [n]$, let $(\vecb{b}_j,\vecb{w}_j) \in \{0,1\}^{\lb} \times \Z_q^{\lc}$ be the witness of $\vecb{x}_j \in \Lang_{[\matr{M}]_1,[\matr{N}]_1,\matr{\Lambda},\grkb{\alpha}}.$ Let $b_{i,j}$ be the ith coordinate of $\vecb{b}_j$ and let 
 $\vecb{b}^*_{i}$ the ith row of $\matr{B}$. The basic steps of the proof are as follows:
\begin{enumerate}
\item Construct a MP commitment $[\vecb{c}_i]_1$ to $\vecb{b}^*_i$, for all $i \in [\lb]$,
\item Prove that $\vecb{b}^*_i \in \{0,1\}^{n}$ for all $i \in [\lb]$, 
\item Prove that $[\vecb{c}]_1:=[\vecb{c}_1]_1 \oplus \ldots \oplus [\vecb{c}_n]_1$ and $[\vecb{x}_1]_1 \oplus \ldots \oplus [\vecb{x}_n]_1$ open to the same value,
\item Prove that $[\widetilde{\vecb{x}}]_1 \in \mathbf{Im}([\matr{\Sigma^n}]_1)$, 
\end{enumerate}
where we define: $\tilde{\vecb{x}}_i:=  \smallpmatrix
{
    \vecb{x}_i\\
    \grkb{\alpha}
}$, $\tilde{\vecb{x}}:= \tilde{\vecb{x}}_1 \oplus 
\ldots \oplus  \tilde{\vecb{x}}_n$, and
$\matr{\Sigma}:=\smallpmatrix{
    \matr{M}       & \matr{N}\\
    \matr{\Lambda} & \matr{0}
    }$. 

The soundness intuition is as follows. Suppose that there is some $j\in[n]$ such that $[\vecb{x}_j]_1\notin\LangMN$ and suppose also that $ck$ is perfectly binding at coorinate $j$. Since the opening of $[\vecb{x}_j]$ can not be in $\bits^\lb$, this implies that the $j$ th column of $\matr{B}$ is not in $\bits^\lb$. Perfectly bindingness at coordinate $j$ implies that $[\vecb{c}_1],\ldots,[\vecb{c}_\lb]$ univocally defines the $j$ th column of $\matr{B}$ and thus there is at least one $[\vecb{c}_i]_1\notin\Lang_{ck,\sfbits}$. Althought we assume that $ck$ is perfectly binding at coordinte $j$, is not hard to see that the same argument works, with probability $1/n$, when $j$ is randomly chosen from $[n]$. 
 \begin{theorem} \label{theo:aggset} There exists a QA-NIZK argument $\Pi_\sfset$ for membership in $\Lang_{[\matr{M}]_1,[\matr{N}]_1,\matr{\Lambda},\grkb{\alpha}}^n$ with proof size  $(2\lb+11)|\GG_1|+10|\GG_2|$, perfect completeness, perfect-zero knowledge and computational soundness. 
\end{theorem}
 


\section{Proof of Correctness of a Shuffle} \label{sec:shuffle}
In a NIZK Shuffle argument one wants to prove that two lists of ciphertexts open to the same values when second list is permuted under some hidden permutation.
If we represent each list of ciphertexts as a matrix where each column is an El-Gamal ciphertexts, then the language of correct shuffles under public key $[\vecb{v}]_2\in\GG^2_2$ can can be defined as 
$$
\Lang_{[\vecb{v}]_2,n,\mathsf{shuffle}}:=\{([\matr{C}]_2,[\matr{D}]_2)\in\GG_2^{2\times n}\times \GG_2^{2\times n} : \exists \matr{P}\in\mathcal{S}_n,\grkb{\delta}\in\Z_q^n \text{ s.t. } [\matr{C}]_2\matr{P}-[\matr{D}]_2 = [\vecb{v}]_2\grkb{\delta}^\top\}.$$
Where $\mathcal{S}_n$ is the set of permutation matrices of size $n\times n$. This definition can be generalized for any ``El-Gamal like'' encryption scheme as, for example, the BBS encryption scheme from \cite{C:BonBoySha04}.

The most efficient NIZK Shuffle argument under falsifiable assumptions is the one from Groth and Lu \cite{AC:GroLu07}, which works for BBS ciphertexts. The proof size is linear in the number of ciphertexts, specifically $15n + 120$ group elements in Type I groups. The security of their construction relies on two assumptions: the \emph{Paring Product Assumption} and the \emph{Permutation Pairing Assumption}. The first assumption is a $\dist_{2,n}\mbox{-}\kermdh$ Assumption, where $\matr{M}\gets\dist_{2,n}$ is of the form $\matr{M}:=\pmatri{x_1,\ldots,x_n\\x_1^2,\ldots,x_n^2}$ for $x_i\gets\Z_q$, $i\in[n]$. The second assumption is proven generically secure in \cite{AC:GroLu07} but it seems to be unrelated with any other assumption.

We construct a NIZK Shuffle argument with smaller proof size, specifically $4n+17$ elements of $\GG_1$, $14$ elements of $\GG_2$, and 1 element of $\Z_q$ in type III groups, and $6n+34$ group elements in Type I groups. The security is based on assumptions which are weaker than the \ddh~plus the \SSDP~Assumption in  Type III groups. In Type I groups our construction can be based on assumptions which are all weaker than \lin{2}.\footnote{In the symmetric case, the proof system from Sect. \ref{sec:bits} can be based on assumptions which are all weaker than $\lin{2}$, similarly as done in XXX Appendix XXX}
 
\subsection{Our construction}

Our proof system builds on a proof that a set of GS commitments opens to elements in the set $S=\{[s_1]_1,\ldots,[s_n]_1\}$, when the $\matr{S}=(s_1,\ldots,s_n)\gets\dist_{1,n}$ and the $\dist_{1,n}\mbox{-}\kermdh$ Assumption holds. Given $[\matr{F}]_1\gets\GS.\Com(\vvar{X})$, where $\vvar{X}=(\vvar{x}_1,\ldots,\vvar{x}_n):=[\matr{S}]_1\matr{P}^\top$, and $\vvar{y}:=[\matr{S}]_1\grkb{\delta}$, we prove that $([\matr{C}]_2,[\matr{D}]_2)\in\Lang_{[\vecb{v}]_2,n,\mathsf{shuffle}}$ as follows:
\begin{enumerate}[label=\alph*)]
\item Show that $[\matr{F}]_1\in\Lang_{S,\vecb{u}_1,\vecb{u}_2}^n$, where $\vecb{u}_1,\vecb{u}_2$ are GS commitment keys.\label{shuffle:a}
\item Give a GS proof for the satisfiability of $\sum_{i\in[n]}[s_i]_1-\sum_{j\in[n]}\vvar{x}_j=[0]_1$.\label{shuffle:b}
\item Give a GS proof for the satisfiability of
$
e(\vvar{X},[\matr{C}]_2^\top)/e([\matr{S}]_1,[\matr{D}]_2^\top)=e(\vvar{y},[\vecb{v}]^\top_2).\label{shuffle:c}
$
\end{enumerate}

A detailed description of our construction can be find in Fig. \ref{fig:shuffles}.

\subsubsection{Soundness Intuition.} Conditions \ref{shuffle:a} and \ref{shuffle:b} implies that the opening of $[\matr{F}]_1$ is a permutation of $[\matr{S}]$. That is, if $[\matr{X}]_1=[\matr{S}]_1\matr{B}$ is the opening of $[\matr{F}]_1$, then $\matr{B}\in\bits^{n\times n}$, $\matr{1}_{1\times n}\matr{B}=\matr{1}_{1\times n}$, and $\matr{B}\matr{1}_{n\times 1}=\matr{1}_{n\times 1}$. Condition \ref{shuffle:a} implies that $[\matr{X}]_1=[\matr{S}]_1\matr{B}$, where $\matr{B}\in\bits^{n\times n}$ and $\matr{1}_{1\times n}\matr{B}=\matr{1}_{1\times n}$. Condition \ref{shuffle:b} implies that $[\matr{S}]_1(\matr{1}_{n\times 1}-\matr{B}\matr{1}_{n\times 1})=[0]_1$, which in turn implies that $\matr{B}\matr{1}_{n\times 1}=\matr{1}_{n\times 1}$ unless $\matr{1}_{n\times 1}-\matr{B}\matr{1}_{n\times 1}$ is solution to the $\dist_{1,n}\mbox{-}\kermdh$ problem.

Suppose that $[\matr{C}]_2=\Enc_{[\vecb{v}]_2}([\matr{M}]_2)$ and $[\matr{C}]_2=\Enc_{[\vecb{v}]_2}([\matr{M}']_2)$. Let $\vecb{k}=(-v_2/v_1,1)^\top$ the ``decryption key'' (i.e. $\vecb{v}^\top\vecb{k}=0$ and $(0,1)\vecb{k}=1$)\footnote{The availability of the decryption key $\vecb{k}$ in the soundness reduction is possible since the reduction samples by itself the language parameter $\vecb{v}$. Correspondingly Groth and Lu \cite{AC:GroLu07} proved \emph{co-soundness}, which essentially requires the soundness adversary to produce the decryption key.}, we can point-wise exponentiate to $\matr{k}$ the equation from condition $\ref{shuffle:c}$ to ``decrypt'' $[\matr{C}]_2,[\matr{D}]_2$ and $[\vecb{v}]_2=\Enc_{[\vecb{v}]_2}([0]_2)$. We obtain that
$e([\matr{S}]_1\matr{B},[\matr{M}]_2)/e([\matr{S}]_1,[\matr{M'}]_2)=1_{\GG_T}$, which implies that $[\matr{M}]_2\matr{B}^\top=[\matr{M}']_2$ unless $[\matr{M}]_2\matr{B}^\top-[\matr{M}']_2$ is a solution to the $\dist_{1,n}\mbox{-}\kermdh$. Finally this implies that $[\matr{C}]_2\matr{B}^\top-[\matr{D}]_2$ is an encryption of $[\matr{0}_{n\times 1}]_2$ and thus $([\matr{C}]_2,[\matr{D}]_2)\in\Lang_{[\vecb{v}]_2,n,\mathsf{shuffle}}$. 

\begin{figure} 
$$
\begin{array}{ll}
\begin{array}{l}
\algK_1(\Gamma,[\vecb{v}]_2,n)\quad (\mathsf{S}_1(\Gamma,[\vecb{v}]_2,n))\\
\hline
S:=\{[s_1]_1,\ldots,[s_n]_1\}, \matr{S}\gets\distlin_{1,n}\\
\crs_\GS\gets\GS.\algK_1(\Gamma)\\
{\crs_\sfbin\gets\Pi_\sfbin.\algK_1(\Gamma,[\matr{M}]_1,[\matr{N}]_1,[\matr{\Lambda}]_1,n)}\\
\text{Return } \ \crs:=([\vecb{v}]_2,\crs_\GS,\crs_\sfbin).\\
(\tau_\GS\gets\GS.\algS_1(\Gamma)\\
\tau_\sfbin\gets\Pi_\sfbin.\algS_1(\Gamma,L,[\matr{M}]_1,[\matr{N}]_1,[\matr{\Lambda}]_1,n).\\
\tau := (\matr{S},\tau_\GS,\tau_\sfbin)).\\
\\
\end{array}
&
\begin{array}{l}
{\algP(\crs, [\matr{C}]_2, [\matr{D}]_2,\langle \matr{P}, \grkb{\delta}\rangle)}\\
\hline
\vvar{X} := [\matr{S}]_1\matr{P}^\top,
\vvar{y} := [\matr{S}]_1\grkb{\delta}\\
\eq_1:= \sum_{i\in[n]}[s_i]_1-\sum_{j\in[n]}\vvar{x}_i = [0]_1\\
\eq_2:= {e(\vvar{X},[\matr{C}]_2^\top)}/{e([\matr{S}]_1,[\matr{D}]_2^\top)}=e(\vvar{y},[\vecb{v}]_2^\top)\\
\pi_\GS\gets\GS.\algP(\crs_\GS,\{\eq_1,\eq_2\},\vvar{X},\vvar{y})\\
\pi_\sfbin \gets \Pi_\sfbin.\algP(\crs_\sfbin, [\matr{F}]_1,\langle\matr{R},\matr{P}^\top\rangle)\\
//{[\matr{F}]_1} = \GS.\Com_{\crs_\GS}(\vvar{X};\matr{R})\\
\text{Return } \  (\pi_\GS,\pi_\sfbin). \\
\\
\end{array}\\
\begin{array}{l}
{\algV(\crs,[\matr{C}]_2,[\matr{D}]_2,(\pi_\GS,\pi_\sfbin))}\\
\hline
\eq_1:= \sum_{i\in[n]}[s_i]_1-\sum_{j\in[n]}\vvar{x}_i = [0]_1\\
\eq_2:= {e(\vvar{X},[\matr{C}]_2^\top)}/{e([\matr{S}]_1,[\matr{D}]_2^\top)}=e(\vvar{y},[\vecb{v}]_2^\top)\\
\mathsf{ans}_1 \gets \GS.\algV(\crs_\GS,\{\eq_1,\eq_2\},\pi_\GS)\\
\mathsf{ans}_2 \gets \Pi_\sfbin.\algV(\crs_\sfbin,[\matr{F}]_1,\pi_\sfbin)\\
\text{Return } \ \mathsf{ans}_1\wedge\mathsf{ans}_2.
\end{array}
&
\begin{array}{l}
{\mathsf{S}_2(\crs,[\matr{C}]_1,[\matr{D}]_1,\tau)}\\
\hline
\eq_1:= \sum_{i\in[n]}[s_i]_1-\sum_{j\in[n]}\vvar{x}_i = [0]_1\\
\eq'_2:= {e(\vvar{X},[\matr{C}]_2^\top)}/{e([1]_1,\matr{S}^\top[\matr{D}]_2^\top)}=e(\vvar{y},[\vecb{v}]_2^\top)\\
\pi_\GS\gets \GS.\algS_2(\crs_\GS,\{\eq_1,\eq'_2\},\tau_\GS)\\
\pi_\sfbin\gets \Pi_\sfbin.\algS_2(\crs_\sfbin,[\matr{F}]_1,\tau_\sfbin)\\
\text{Return }  (\pi_\GS,\pi_\sfbin).
\end{array}
\end{array}$$
\caption{The proof system $\Pi_\mathsf{shuffle}$ for the language $\Lang_{[\vecb{v}]_2,n,\mathsf{shuffle}}$. The matrices $\matr{M},\matr{N},\matr{\Lambda}$ are defined as $\matr{M}:=\usebox{\smat},\matr{N}:=(\vecb{u}_1||\vecb{u}_2),\matr{\Lambda}:= (1,\ldots,1)$, where $\vecb{u}_1,\vecb{u}_2$ are the GS commitment keys from $\crs_\GS$. The proof size is $|\pi_\GS|+|\pi_\sfbin|=(2n+6)|\GG_1|+4|\GG_2|+1|\Z_q|+(2n+11)|\GG_1|+10|\GG_2|=(4n+17)|\GG_1|+14|\GG_2|+1|\Z_q|$.
\label{fig:shuffles}}
\end{figure}




%    \subsection{Security Proof}
%        \subsubsection{Completeness.} If $\matr{P}$ is a permutation matrix then $\vvar{X}=[\matr{S}]_1\matr{P}^\top=([s_{\pi(1)}]_1,\ldots,[s_{\pi_{n}}]_1)$, for some permutation $\pi$, which implies that $\GS.\Com(\vvar{X})\in\Lang_{S,\vecb{u}_1,\vecb{u}_2}^n$ and $\sum_{i\in[n]}[s_i]_1-\sum_{j\in[n]}\vvar{x}_j=[0]_1$. If $\vary=[\matr{S}]_1\grkb{\delta}$ then
\begin{eqnarray*}
\dfrac
    {e(\vvar{X},[\matr{C}]_2^\top)}
    {e([\matr{S}]_1,[\matr{D}]_2^\top)}
 = 
    e([\matr{S}]_1,([\matr{C}]_2\matr{P}-[\matr{D}]_2)^\top)
 = 
    e([\matr{S}]_1,([\vecb{v}]_2\grkb{\delta}^\top)^\top)
 =
    e(\vary,[\vecb{v}]_2^\top).
\end{eqnarray*}

\subsubsection{Soundness.}
\begin{theorem}
For any adversary $\advA$ against the soundness of the proof system from Figure \ref{fig:shuffles}, there exist an adversary $\advB_1$ against soundness of $\Pi_{\sfbin}$ and an adversary $\advB_2$ against the $\dist_{1,n}\mbox{-}\kermdh$ Assumption such that
$$\adv(\advA) \leq \adv_{\Pi_{\sfbin}}(\advB_1) + \adv_{\dist_{1,n}\mbox{-}\kermdh}(\advB_2).
$$
\end{theorem}

\begin{proof}
The adversary $\advB_1$ receives as input $\crs_\sfbin$ and honestly samples the rest of the CRS. Then $\advB_1$ runs $\advA$ until it halts and outputs $[\matr{F}]_1$ with the proof $\pi_\sfbin$.

The adversary $\advB_2$ receives as input $[\matr{S}]_1\in\GG_1^{1\times n}$, samples $\vecb{u}_1,\vecb{u}_2,\vecb{v}\gets \Z_q^2$, honestly simulates the rest of the CRS, and runs $\advA$ until it halts. It extracts $[\matr{X}]_1$, the opening of $[\matr{F}]_1$, using $\vecb{u}_1,\vecb{u}_2$, and aborts if $[\matr{F}]_1\notin\Lang_{S,\vecb{u}_1,\vecb{u}_2}^n$. Else $[\matr{X}]_1=[\matr{S}]_1\matr{B}$, where $\matr{B}\in\bits^{n\times n}$ and $\matr{1}_{1\times n}\matr{B}=\matr{1}_{1\times n}$. If $\sum_{i\in[n]}[s_i]_1-\sum_{j\in[n]}[x_j]_1\neq[0]_1$, $\advB$ outputs $\matr{1}_{n\times 1}-\matr{B}\matr{1}_{n\times 1}$. Else, using $\vecb{v}$, $\advB_2$ decrypts $[\matr{C}]_2$ and $[\matr{D}]_2$ obtaining $[\vecb{M}]_2\in\GG_2^{1\times n}$ and $[\vecb{M}']_2\in\GG_2^{1\times n}$, respectively, and returns $[\vecb{M}]_2\matr{B}^\top-[\vecb{M}']_2$.

Let $E_1$ the event where $([\matr{C}]_2,[\matr{D}]_2)\in\Lang_{[\vecb{v}]_2,n,\mathsf{shuffle}}$, $E_2$ the event where $[\matr{F}]_1\in\Lang_{S,\vecb{u}_1,\vecb{u}_2}^n$, and $E_3$ the event where $\matr{1}_{n\times 1}-\matr{B}\matr{1}_{n\times 1}=\vecb{0}_{n\times 1}\wedge[\vecb{M}]_2\matr{B}-[\vecb{M}']_2=0$. Note that $E_2\wedge E_3\Longrightarrow E_1$, because $E_2$ implies that $[\matr{X}]_1=[\matr{S}]_1\matr{B}$, where $\matr{B}\in\bits^{n\times n}$ and $\matr{1}_{1\times n}\matr{B}=\matr{1}_{1\times n}$. Together with $E_3$ implies that $\matr{B}$ is a permutation and thus $[\matr{C}]_2\matr{B}^\top-[\matr{D}]_2=\Enc([\matr{0}_{1\times n}]_2)\Longleftrightarrow E_1$. Note also that $\eq_1\wedge\eq_2\wedge\neg E_3$ implies that $\matr{1}_{n\times 1}-\matr{B}\matr{1}_{n\times 1}$ or $[\vecb{M}]_2\matr{B}-[\vecb{M}']_2$ are solutions to the $\dist_{1,n}\mbox{-}\kermdh$. Then it holds that
\begin{eqnarray*}
\adv(\advA) & = &
    \Pr[\neg E_1 \wedge \algV(\crs,([\matr{C}]_2,[\matr{D}]_2),\pi)=1]\\
& = &
    \Pr[\neg E_1 \wedge \algV(\crs,([\matr{C}]_2,[\matr{D}]_2),\pi)=1 \wedge \neg E_2]+\\
& &
    \Pr[\neg E_1 \wedge \algV(\crs,([\matr{C}]_2,[\matr{D}]_2),\pi)=1 \wedge E_2]\\
& \leq &
    \Pr[\neg E_2 \wedge \Pi_\sfbin.\algV(\crs_\sfbin,[\matr{F}]_2,\pi_\sfbin)=1]+
    \Pr[\neg E_1 \wedge \eq_1 \wedge \eq_2 \wedge E_2]\\
& \leq &
    \adv_{\Pi_\sfbin}(\advB_1)+\Pr[\neg E_1 \wedge \eq_1 \wedge \eq_2 \wedge E_2\wedge E_3] + \Pr[\neg E_1 \wedge \eq_1 \wedge \eq_2 \wedge E_2\wedge \neg E_3]\\
& \leq &
    \adv_{\Pi_\sfbin}(\advB_1) + \Pr[\neg E_1\wedge E_2 \wedge E_3]+
    \Pr[\eq_1\wedge\eq_2\wedge\neg E_3]\\
& = &
    \adv_{\Pi_\sfbin}(\advB_1)+0+\adv_{\dist_{1,n}\mbox{-}\kermdh}(\advB_2).
\end{eqnarray*} 
\end{proof}

\subsubsection{Zero-Knowledge}. We need to check that the inputs to the simulators are true statements and, for the GS simulator, that the equations allow simulation. This is certainly true for $\mathsf{eq}_1$ and, if $([\matr{C}]_1,[\matr{D}]_1)\in\Lang_{[\vecb{u}]_1,n,\mathsf{shuffle}}$, then is also true for $\mathsf{eq}'_2$. Furthermore, it is guaranteed that $\pi_\GS$ is computationally indistinguishable from a real proof for $\{\eq_1,\eq'_2\}$, which is identically distributed to a real proof for $\{\eq_1,\eq_2\}$ since $\eq'_2$ and $\eq_2$ accepts the same set of solutions.
Finally, since the perfectly hiding $\crs_\GS$ is such that $\rank(\vecb{u}_1||\vecb{u}_2)=2$, then $\Lang_{S,\vecb{u}_1,\vecb{u}_2}^n=\GG_1^{2\times n}$ and thus $[\matr{F}]_1\in\Lang_{S,\vecb{u}_1,\vecb{u}_2}^n$ is always true.

\section{Range Argument in the Interval $[0,2^n-1]$} \label{sec:range-proof}
    \input{range-proofs/range-proofs}
\bibliographystyle{abbrv}
\bibliography{cryptobib/abbrev2,cryptobib/crypto,manualbib}

\appendix
\section{Quasi-Adaptive NIZK} \label{app:qanizk}
    \input{prelim/app-qanizk}
\section{Extended Multi-Pedersen Commitments: Proofs} \label{app:mp}
    In this section we prove Theorem~\ref{theo:mp}. 

\begin{proof}
\begin{description}
\item[Computationally binding:] (This follows a proof due to Villar). Let $[a]_\gamma\in\GG_{\gamma}$ be the discrete logarithm challenge. To sample the commitment key according to $\distink$, choose $\matr{G}_{2} \gets \dist_k$, and define the last $k$ columns of $\bmatr{G}$ as $[\matr{G}_{2}]_{\gamma}$. For the rest of the columns of $[\matr{G}]_\gamma$, independently  for each $j \in [n]$, $i \neq j$, sample a pair $\vecb{\alpha}_j,\vecb{\beta}_j$ and define $[\vecb{g}_{j}]_{\gamma}=[\matr{G}_{2} (a \vecb{\alpha}_j+\beta_j)]$,
which can be computed as $[a] \matr{G}_{2}\vecb{\alpha}_j+[ \matr{G}_{2} \beta_j]$. If $i \neq 0$, set $\vecb{g}_{i} \gets \Z_q^{k+1}$. In this case, with overwhelming probability,  $\vecb{g}_{i}$ is linearly independent of the rest of the columns and we will assume so in the following. 
The commitment key is then given to the adversary against the binding property of the scheme, and it outputs a commitment $[\vecb{c}]_{\gamma}$, together with two valid openings 
$(\vecb{m},\vecb{r}), (\vecb{m}',\vecb{r}')$ such that $\vecb{m}\neq \vecb{m}'$. It follows that $[\vecb{c}]_{\gamma}=[\matr{G}]_{\gamma} \smallpmatrix{\vecb{m} \\ \vecb{r}} = \bmatr{G}_{\gamma}\smallpmatrix{\vecb{m}' \\ \vecb{r}'}$, which implies that $[\vecb{0}]_{\gamma}=[\matr{G}]_{\gamma} \smallpmatrix{\vecb{m}-\vecb{m}' \\ \vecb{r}-\vecb{r}'}$. Further, because $\vecb{g}_i$ is linearly independent of the rest of the columns, it holds that:
\begin{equation}\label{recovera}
a \left(\matr{G}_{2} (\sum_{j \neq i} (m'_j-m_j)  \boldsymbol{\alpha}_j) \right) = \left(  \matr{G}_{2}  (\vecb{r}-\vecb{r}' + \sum_{j \neq i} \boldsymbol{\beta}_j  (m_j-m_j'))\right).
\end{equation}
W.l.o.g we can assume that $\matr{G}_{2}$ has full rank (it can be shown that if $\dist_k$-$\mddh$ is a generically hard assumptions in $k$-linear groups, then matrices sampled from $\dist_k$ have full rank with overwhelming probability).  Then, we can recover $a \in \Z_q$ from equation \ref{recovera} except if $\sum_{j \neq i} (m'_j-m_j)  \boldsymbol{\alpha}_j = \vecb{0}$. But since, for all $j$, $\boldsymbol{\alpha}_j$ is information theoretically hidden from the adversary, the probability of this event is at most 
$1/q^k$. 


\item[Perfectly binding at coordinate $i$:] With overwhelming probability, $\vecb{g}_i$ is linearly independent of the rest of the columns of $\matr{G}$. Therefore, given any $[\vecb{c}]_{\gamma} \in \GG_{\gamma}^{k+1}$, if $\vecb{m}\in\Z_q^n, \vecb{r} \in\Z_q^k$ are such that $\vecb{c}=\matr{G}\smallpmatrix{\vecb{m}\\ \vecb{r}}$, there exists a unique $\tilde{m}_i \in \Z_q$ such 
that $m_i=\tilde{m}_i$. 

\item[Perfectly hiding at coordinate $j$, $j \neq i$:] This follows immediately from the fact that $\vecb{g}_j$ is in the image 
of $\matr{G}_{2}$. 
\end{description}
\end{proof}



%\section{QA-NIZK for Bit-Strings, Revisited} \label{app:bits}
%In this section, we give the details of the QA-NIZK argument to prove that a MP commitment opens to a bit-string.
%    \subsection{The scheme} \label{sec:bits-scheme} 
%        \input{bits/proof-system}
%    \subsection{Security Proof} \label{app:proofbits}
%        \begin{theorem} \label{theo:bits}
The proof system described on Sect. \ref{sec:bits-scheme} is a QA-NIZK proof system with Perfect Completeness, Computational Soundness, and Perfect Zero-Knowledge.
\end{theorem}	
\begin{proof}
We remark that proof of Completeness and Zero-Knowledge is the same for any distribution $\distlininone$.
\begin{description}
\item[Perfect Completeness:]
Note that,
by definition of $\matr{C}_{i,j}$ and $\matr{D}_{i,j}$, 
$[\matr{C}_{i,j}]_1[\matr{I}]_2+[\matr{I}]_1[\matr{D}_{i,j}]_2=$
$[\vecb{g}_{i}]_1[\vecb{h}_j]_2^\top$.  Since $b_i(b_i-1) = 0$ for each $i\in[m]$,
\begin{eqnarray*}
\lefteqn{
[\vecb{c}]_1( [\vecb{d}]_2 - \sum_{i\in[m]} [\vecb{h}_{i}]_2)^\top}\\
& = & 
    \sum_{i \in [m]}\left(
        b_i s[\vecb{g}_{i}]_1[\vecb{h}_{m+1}]_2^{\top}+
        r(b_i-1)[\vecb{g}_{m+1}]_1[\vecb{h}_i]_2^{\top}+
        \sum_{j \in [m]} b_i (b_j-1)[\vecb{g}_{i}]_1[\vecb{h}_{j}]_2^{\top}
    \right)
\\ & & \mbox{ }
    + rs[\vecb{g}_{m+1}]_1[\vecb{h}_{m+1}]_2^{\top}\\
& = & 
    \left(\sum_{i\in[m]}b_is[\vecb{g}_{i}]_1\vecb{h}_{m+1}^{\top}+r(b_i-1)[\vecb{g}_{m+1}]_1\vecb{h}_i^{\top}+
        \sum_{\substack{j \in [m]\\j\neq i}}b_i(b_j-1)[\vecb{g}_{i}]_1\vecb{h}_{j}^{\top}\right)[\matr{I}]_2\\
& & \mbox{ }
    +rs[\vecb{g}_{m+1}]_1\vecb{h}_{m+1}^{\top}[\matr{I}]_2
    +[\matr{R}]_1[\matr{I}]_2 + [\matr{I}]_1[-\matr{R}]_2
\\ & = &
    [\matr{\Theta}]_1[\matr{I}]_2+
    [\matr{I}]_1[\matr{\Pi}]_2.
\end{eqnarray*}
Finally, the rest of the proof follows from completeness of $\Pi_\sfsum$ and $\Pi_\sfcom$. 

\item[Soundness:] When $\matr{G}$ is sampled from $\distlinizeroone$ it suffices to prove that the commitment $[\vecb{c}]_1$ output by the adversary is in $\Span([\matr{G}]_1)$ since, by the perfect hiding property, $[\vecb{c}]_1$ can be opened to any $\vecb{b}\in\bits^m$ thus $[\vecb{c}]_1\in\Lang_{ck,\sfbits}$. If $[\vecb{c}]_1\notin\Span([\matr{G}]_1)$, then we can break the (strong) soundness of the proof that $[\vecb{c}]_1$ and $[\vecb{d}]_2$ open to the same value, since that proof implies that there exist $\vecb{x},r,s$ such that $[\vecb{c}]_1=[\matr{G}]_1\smallpmatrix{\vecb{x}\\r}$ and $[\vecb{d}]_2=[\matr{H}]_2\smallpmatrix{\vecb{x}\\s}$. Therefore, we construct an adversary $\advB$ against the strong soundness of $\Pi_\sfcom$ that simulates $\advA$ until it halts and outputs $([\vecb{c}]_1,[\vecb{d}]_2,\pi_\sfcom)$. Note that, in order to simulate the CRS $\advB$ requires $\matr{H}$, but this is not a problem since is part of the input in the strong soundness game.
 
When $\matr{G}$ is sampled from $\distlinisnone$, ${i^*}>0$, the proof follows from the indistinguishability of the following three games:
\begin{itemize}
\item[$\mathsf{Real}$:] This is the real Soundness game. The output is 1 if the adversary submits some $[\vecb{c}]_1\notin\Lang_{ck,\sfbits}$ and the corresponding proof which is accepted by the verifier.
\item[$\sfGame_0$:] This identical to $\mathsf{Real}$, except that $\algK_1$ does not receive $[\matr{G}]_1$ as a input but
it samples $\matr{G}$ itself according to $\distlinisnone$.
\item[$\sfGame_1$:] This game is identical to $\sfGame_0$ except that now $\matr{H}\gets\distlinisnone$.
\end{itemize}

It is obvious that the first two games are indistinguishable. The rest of the argument goes as follows.

\begin{lemma} There exists a\ $\distlin_1$-$\mddh_{\GG_2}$ adversary $\advD$ such that
$|\Pr\left[\mathsf{Game}_{0}(\advA)=1\right]$ $-\Pr\left[\mathsf{Game}_{1}(\advA)=1\right]|$ $\leq \mathsf{Adv}_{\distlin_1,\ggen_a}(\advD).$
\end{lemma}
\begin{proof}
We construct an adversary $\advD$ that receives 
a challenge $([\matr{A}]_2,[\vecb{u}]_2)$ of the 
$\distlin_1$-$\mddh_{\GG_2}$ Assumption. From this challenge, $\advD$ just defines the matrix  $[\matr{H}]_2\in\GG_2^{2\times(m+1)}$ as the matrix whose last column consists of $[\matr{A}]_2$, the ith column consists of $[\vecb{u}]_2$ and the rest of the columns are random vectors in the image of $[\matrA]_2$. 
Obviously, when $[\vecb{u}]_2$ is sampled from 
the image of $[\matr{A}]_2,$ $\matr{H}$ follows the distribution $\distlinizeroone$, while if $[\vecb{u}]_2$ is a uniform element of $\GG^2$, $\matr{H}$ follows the distribution $\distlinisnone$. 
 
Adversary $\advD$ samples
$\matr{G} \gets \distlinisnone$. Given that $\advD$ does not know the discrete logarithms of $[\matr{H}]_2$, it cannot compute the pairs $(\matr{C}_{i,j},\matr{D}_{i,j})$ exactly as in $\sfGame_0$. Nevertheless, for each $(i,j)\in\indexSet{m}{1}$ it can compute identically distributed pairs by picking $\matr{T}\gets\Z_q^{2\times 2}$ and defining
$$
([\matr{C}_{i,j}]_1,[\matr{D}_{i,j}]_2):=([\matr{T}]_1,\vecb{g}_i[\vecb{h}_j]_2^\top-[\matr{T}]_2).
$$
The rest of the elements of the CRS, namely $\crs_\sfcom$ and $\crs_\sfsum$, are honestly computed. When $\matr{H}\gets\distlinizeroone$, $\advD$ perfectly simulates $\sfGame_0$, and when $\matr{H}\gets\distlinisnone$, $\advD$ perfectly simulates $\sfGame_1$, which concludes the proof. 
\end{proof}

\begin{lemma}
There exist adversaries $\advB_1$, against the strong soundness of $\Pi_\sfcom$, and $\advB_2$, against the soundness of $\Pi_\sfsum$, such that $\Pr[\sfGame_1(\advA)=1]\leq 4/q+ \adv_{\Pi_\sfcom}(\advB_1)+\adv_{\Pi_\sfsum}(\advB_2)$.
\end{lemma}
\begin{proof}
With probability $1-4/q$, $\{\vecb{g}_{i^*},\vecb{g}_{m+1}\}$ and $\{\vecb{h}_{i^*},\vecb{h}_{m+1}\}$ are both bases of $\Z_q^2$,
we can define $b_{i^*},\overline{w}_g,\overline{w}_h,\overline{b}_{i^*}$ as the unique coefficients in $\Z_q$ such that $\vecb{c}=b_{i^*}\vecb{g}_{i^*} + \overline{w}_g \vecb{g}_{m+1}$ and $\vecb{d}= \bb_{i^*} \vecb{h}_{i^*} + \overline{w}_h \vecb{h}_{m+1}$.

In particular, if $\advA$ breaks soundness, this implies that $b_{i^*} \notin \{0,1\}$ (since for $i\neq i^*$, 
$\vecb{c}$ can always be opened to 
  choose $b_i=0$). Further, the verifier accepts the proof proof:
$ (
        [\vecb{d}]_2,
        ([\matr{\Theta}]_1, [\matr{\Pi}]_2), 
        \pi_\sfcom,\pi_\sfsum )$
  produced by $\advA$.
We distinguish two cases:
\begin{description}
\item[$b_{i^*} \neq \overline{b}_{i^*}$:] Given that $[\vecb{c}]_1$ and $[\vecb{d}]_2$ are perfectly binding at coordinate $i^*$, if $b_{i^*}\neq\bb_{i^*}$ it is not possible that $[\vecb{c}]_1$ and $[\vecb{d}]_2$ open to the same value. We construct an adversary $\advB_1$ against the strong soundness 
of $\Pi_\sfcom$ that simulates game $\sfGame_1$ with $\advA$ (using $\matr{H}$ to simulate the CRS) until it halts and outputs $([\vecb{c}]_1,[\vecb{d}]_2,\pi_\sfcom)$. If $b_{i^*}\neq\bb_{i^*}$, $\pi_\sfcom$ is a fake proof for $([\vecb{c}]_1,[\vecb{d}]_2)$ opening to the same value and then $\advB_1$ breaks the strong soundness of $\Pi_\sfcom$.
\item[$b_{i^*} = \overline{b}_{i^*}$, 
$b_{i^*}(\overline{b}_{i^*} -1) \neq 0$:]
If we express $\matr{\Theta}+\matr{\Pi}$
as a linear combination of $\{\vecb{g}_{i}\vecb{h}_{j}^{\top}:i,j\in[n+1]\}$, the coordinate of
$\vecb{g}_{i^*}\vecb{h}_{i^*}^\top$ is $b_{i^*}(\bb_{i^*}-1)\neq 0$ and thus $\matr{\Theta}+\matr{\Pi}\notin\Span(\{\matr{C}_{i,j}+\matr{D}_{i,j}:(i,j)\in\indexSet{m}{1}\})$. We construct an adversary $\advB_2$ against the soundness of $\Pi_\sfsum$ that simulates game $\sfGame_1$ with $\advA$ until it halts and outputs $([\matr{\Theta}]_1,[\matr{\Pi}]_2,\pi_\sfsum)$. If $b_{i^*} = \overline{b}_{i^*}$ but $b_{i^*}(\overline{b}_{i^*} -1) \neq 0$, $\pi_\sfsum$ is a fake proof for $([\matr{\Theta}]_1,[\matr{\Pi}]_2)$ and then $\advB_2$ breaks the soundness of $\Pi_\sfsum$.
\end{description}
\end{proof}

\item[Perfect Zero-Knowledge:] First, note that the vector $[\vecb{d}]_2 \in \GG_2^2$ output by the prover and the vector output by $\algS_2$ follow exactly the same distribution. This is because $\matr{H}\gets\distlinizeroone$ defines perfectly hiding commitments. In particular, although the simulator $\algS_2$ does not know $\vecb{b} \in \{0,1\}^{m}$ such that $[\vecb{c}]_1=[\matr{G}]_1\smallpmatrix{\vecb{b}\\r}$, for some $r\in\Z_q$, 
there exists $s \in \Z_q$ such that $[\vecb{d}]_2=[\matr{H}]_2\smallpmatrix{\vecb{b}\\ s}$. 

Since $\matr{R}$ is chosen uniformly at random in $\Z_q^{2 \times 2}$, the proof $([\matr{\Theta}]_1, [\matr{\Pi}]_2)$ is uniformly distributed conditioned on satisfying check 1) of algorithm $\algV$.
 Finally, the rest of the proof follows from Zero-Knowledge of $\Pi_\sfsum$ and $\Pi_\sfcom$.
\end{description}
\end{proof}

\section{Details of the Constant-size Argument for $\Lang_{ck,\sfbits}^n$} \label{app:bits-n}
        \input{bits/matr-bits-intuition}
        \subsection{The Scheme}

%   If $\matr{G}\gets\dist_{2,m+1}^0$ the proof that $[\vecb{c}]_1$ and $[\ovecb{c}]_1$ open to the same value implies that $\ovecb{c}_j\in\Span(\matr{G})$ for all $j\in[n]$ and, by the perfect hiding property, $([\vecb{c}_1]_1,\ldots,[\vecb{c}_n]_1)$ can be opened to any $\matr{B}\in\bits^{m\times n}$ thus $([\vecb{c}_1]_1,\ldots,[\vecb{c}_n]_1)\in\Lang_{ck,\sfbits}^n$. If $\matr{G}\gets\dist_{2,m+1}^{i^*}$ and $([\vecb{c}_1]_1,\ldots,[\vecb{c}_n]_1)\notin\Lang_{ck,\sfbits}^n$, then the $i^*$ th row of $\matr{B}$ is not in $\bits^{1\times n}$ which implies that there is some $j\in[n]$ such that $[\vecb{c}_j]_1\notin\Lang_{ck,\sfbits}$. If we pick $j^*\gets[n]$ and $\overline{\matr{G}}\gets\dist_{2,mn+1}^{m(i^*-1)+j^*}$, soundness of the proof that $[\ovecb{c}]_1\in\Lang_{\overline{ck},\sfbits}$ is violated with probability at least $1/n$.
\begin{figure} 
$$
\begin{array}{ll}
\begin{array}{l}
\algK_1(\gk,[\matr{G}]_1,n)\quad (\mathsf{S}_1(\gk,[\matr{G}]_1,n))\\
\hline\\[-.45cm]
[\overline{\matr{G}}]_1 \gets \MP.\algK(1^\lambda,mn,\distlin_1^{mn,0})\\
\crs_\sfcom\gets\Pi_\sfcom.\algK_1(\gk,[\vecb{G}^n]_1,[\ovG]_1, mn)\\
\crs_\sfbits\gets\Pi_\sfbits.\algK_1(\gk,[\overline{\matr{G}}]_1)\\
\text{Return } \ \mathsf{crs}:=(\crs_\sfcom,\crs_\sfbits).\\
(\tau_\sfcom\gets\Pi_\sfcom.\mathsf{S}_1(\gk,[\matr{G}^n]_1,[\ovG]_1,mn)\\
\tau_\sfbits\gets\Pi_\sfbits.\mathsf{S}_1(\gk,[\overline{\matr{G}}]_1).\\
\tau := (\vecb{a},\tau_\sflin,\tau_\sfbits)).\\
\\
\end{array}
&
\begin{array}{l}
{\algP(\mathsf{crs}, ([\vecb{c}]_1,\ldots,[\vecb{c}_n]_1), \langle (\vecb{b}_1,\ldots,\vecb{b}_n), \vecb{w}\rangle)}\\
\hline\\[-.45cm]
{[\ovecb{c}]_1 :=\MP.\Com_{[\ovG]_1}(\vecb{b};\overline{w})},\overline{w}\gets\Z_q\\
{\pi_\sfcom \gets \Pi_{\sfcom}.\algP(\crs_\sfcom,[\vecb{c}]_1,[\ovecb{c}]_1,}{\langle\vecb{b},\vecb{w},\overline{w}\rangle)}\\
\pi_\sfbits \gets \Pi_\sfbits.\algP(\crs_\sfbits,[\ovc]_1,\langle \vecb{b},\overline{w}\rangle)\\
\text{Return } \  ([\ovc]_1,\pi_\sfcom,\pi_\sfbits). \\
\\
\\
\\
\\
\end{array}\\
\begin{array}{l}
{\algV(\mathsf{crs},([\vecb{c}_1]_1,\ldots,[\vecb{c}_n]_1),([\ovc]_1,\pi_\sfcom,\pi_\sfbits))}\\
\hline
\mathsf{ans}_1 \gets \Pi_\sfcom.\algV(\crs_\sfcom,[\vecb{c}]_1,[\ovc]_1,\pi_\sfcom)\\
\mathsf{ans}_2 \gets \Pi_\sfbits.\algV(\crs_\sfbits,[\ovc]_1,\pi_\sfbits)\\
\text{Return } \ \mathsf{ans}_1\wedge\mathsf{ans}_2.
\\
\\
\end{array}
&
\begin{array}{l}
{\mathsf{S}_2(\crs,([\vecb{c}_1]_1,\ldots,[\vecb{c}_n]_1),[\matr{D}]_1,\tau)}\\
\hline
{[\ovc]_1 \gets \MP.\Com_{[\ovG]_1}(\vecb{0}_{mn\times 1})}\\
\pi_\sfcom\gets \Pi_\sfcom.\algS_2(\crs_\sfcom,[\vecb{c}]_1,[\ovc]_1,\tau_\sfcom)\\
\pi_\sfbits \gets \Pi_\sfbits.\algS_2(\crs_\sfbits,[\ovc]_1,\tau_\sfbits)\\
\text{Return }  ([\ovc]_1,\pi_\sfcom, \pi_\sfbits).
\end{array}
\end{array}$$
\caption{The proof system for the language $\Lang_{[\matr{G}]_1,\sfbits}^n$. $\Pi_\sfbits$ is the proof system from Sect. \ref{sec:bits-scheme}. The matrix $\matr{G}^n$ is defined as
$\matr{G}^n:=(\matr{G}_1\oplus\ldots\oplus\matr{G}_1||\vecb{g}_{n+1}\oplus\ldots\oplus\vecb{g}_{n+1})$, and $\vecb{c}:=\vecb{c}_1\oplus\ldots\oplus\vecb{c}_n$ and $\vecb{b}:=\vecb{b}_1\oplus\ldots\oplus\vecb{b}$.
\label{fig:bitsn}
}
\end{figure}



\subsection{Security Proof}

\begin{proof}
\begin{description}
\item[Completeness.]
Follows from the fact that $([\vecb{c}]_1,[\ovc]_1)\in\Lang_{\sfcom,[\matr{G}^n]_1,[\ovG]_1,mn}$ and that $[\ovc]_1\in\Lang_{[\ovG]_1,\sfbits}$.

\item[Soundness.]
When $\matr{G}\gets\distlinizeroone$ the proof follows from the proof that $[\vecb{c}]_1$ and $[\ovecb{c}]_1$ open to the same value.
When $\matr{G}\gets\distlinisnone$, the proof follows from the indistinguishability of the following games.

\begin{description}
\item[$\mathsf{Real}$:] This is the real soundness game. The adversary wins if it outputs $([\vecb{c}_1]_1,\ldots,[\vecb{c}_n]_1)\notin\Lang_{ck,\sfbits}^n$ and the corresponding proof which is accepted by the verifier.
\item[$\sfGame_0$:] This game is exactly as $\mathsf{Real}$ except that $\algK_1$ does not receive $[\matr{G}]_1$ as a input but it samples $\matr{G}$ itself according to $\distlinisnone$.
\item[$\sfGame_1$:] This game is exactly as $\sfGame_0$ except that the simulator picks a random $j^*\in[n]$ and uses $\matr{G}$ to check whether $\vecb{c}_{j^*}=b_{i^*,j^*}\vecb{g}_{i^*}+\tilde{w}\vecb{g}_{n+1}$ such that $b_{i^*,j^*}\notin\bits$. It aborts if this is not the case.
\item[$\sfGame_2$:] This game is exactly as $\sfGame_1$ except that $\overline{\matr{G}}\gets\distlin_1^{mn, m(i^*-1)+j^*}$.
\end{description}

It is obvious that the first two games are indistinguishable. 
The rest of the argument goes as follows. 

\begin{lemma} $\Pr\left[ \mathsf{Game}_1(\advA)=1\right]\geq\dfrac{1}{n}\Pr\left[\mathsf{Game}_0(\advA)=1\right].$
\end{lemma}

\begin{proof}  The probability that
 $\mathsf{Game}_1(\advA)=1$ is the probability that  a) $\mathsf{Game}_0(\advA)=1$ and
b)  $b_{i^*,j^*} \notin \bits$. The view of adversary $\advA$ is independent of $j^*$, while, if $\mathsf{Game_0}(\advA)=1$, then there is at least one index $\ell \in [n]$ such that $[\vecb{c}_\ell]_1\notin\Lang_{[\matr{G}]_1,\sfbits}\Longrightarrow b_{i^*,\ell} \notin \bits$. Thus, 
the probability that the event described in b) occurs conditioned on $\mathsf{Game_0}(\advA)=1$, is greater than or equal to $1/n$ and the lemma follows.
\end{proof}

\begin{lemma} There exists a\ $\dist_1$-$\mddh_{\GG_1}$ adversary $\advD$ such that
$|\Pr\left[\mathsf{Game}_{1}(\advA)=1\right]$ $-\Pr\left[\mathsf{Game}_{2}(\advA)=1\right]|$ $\leq
    \mathsf{Adv}_{\distlin_1,\ggen_a}(\advD).$
\end{lemma}
\begin{proof}
We construct an adversary $\advD$ that receives 
a challenge $([\matr{A}]_1,[\vecb{u}]_1)$ of the 
$\distlin_1$-$\mddh_{\GG_1}$ Assumption. From this challenge, $\advD$ just defines the matrix  $[\overline{\matr{G}}]_1\in\GG_1^{2\times(mn+1)}$ as the matrix whose last column consists of $[\matr{A}]_1$, the ith column consists of $[\vecb{u}]_1$ and the rest of the columns are random vectors in the image of $[\matrA]_1$. Then $\advD$ honestly simulates the rest of the CRS, gives it as input to $\advA$, and outputs whatever $\advA$ outputs.

Obviously, when $[\vecb{u}]_1$ is sampled from 
the image of $[\matr{A}]_1,$ $\overline{\matr{G}}$ follows the distribution $\distlinizeroone$ and $\advD$ perfectly simulates $\sfGame_1$, while if $[\vecb{u}]_1$ is a uniform element of $\GG^2_1$, $\overline{\matr{G}}$ follows the distribution $\distlinisnone$ and $\advD$ perfectly simulates $\sfGame_2$. 
%It is immediate to see that adversary $\advD$ perfectly simulates $\sfGame_1$ when $\overline{\matr{G}}\gets\dist_{2,mn+1}^0$ and $\sfGame_2$ when $\overline{\matr{G}}\gets\dist_{2,mn+1}^{m(i^*-1)+j^*}$. The rest of the proof follows from Lemma \ref{lemma:dist-i}.  
\end{proof}

\begin{lemma}
There exists adversaries $\advB_1,\advB_2$ such that $\Pr[\sfGame_2(\advA)=1]\leq\adv_{\Pi_\sfcom}(\advB_1)+\adv_{\Pi_\sfbits}(\advB_2)$.
\end{lemma}

\begin{proof}
If $\sfGame(\advA)=1$, then $b_{i^*,j^*}\notin\bits$ while all the verification equations are accepted. Given that $\ovecb{g}_{m(i^*-1)+j^*}$ is linearly independent from $\{\ovecb{g}_j:j\neq m(i^*-1)+j^*\}$, it holds that $\{\ovecb{g}_{m(i^*-1)+j^*},\ovecb{g}_{mn+1}\}$ is a basis for $\Z_q^2$ and thus we can define $\bb_{i^*,j^*},\overline{w}_{h,i}$ as the unique coefficients in $\Z_q$ such that $\ovecb{c} = \bb_{i^*,j^*}\ovecb{g}_{m(i^*-1)+j^*}+\overline{w}_{h,i}\ovecb{g}_{mn+1}$. If $b_{i^*,j^*}\neq\bb_{i^*,j^*}$, then $([\vecb{c}]_1,[\ovecb{c}]_1)$ can not open to the same value and we can construct an adversary $\advB_1$ against $\Pi_\sfcom$. Else, it must be the case that $\bb_{i^*,j^*}=b_{i^*,j^*}\notin\bits$. Therefore, if an adversary $\advB_2$ simulates $\sfGame_2$ until $\advA$ halts and outputs $([\ovecb{c}]_1,\pi_\sfbits)$, then $\advB_2$ breaks soundness of $\Pi_\sfbits$. 
\end{proof}

\item[Zero-Knowledge.] Given that $\overline{\matr{G}}$ defines perfectly hiding commitments, $[\ovecb{c}]_1$ can be opened to any value. Therefore $[\ovecb{c}]_1$ and $[\vecb{c}]_1$ share a common opening and $[\ovecb{c}]_1\in\Lang_{\overline{ck},\sfbits}$, and thus $\pi_\sfcom$ and $\pi_\sfbits$ are correctly distributed.  
\end{description}
\end{proof}



\section{Details of Aggregated NIZK Set Membership Arguments} \label{app:bin-lang}
    As announced in Section~\ref{sec:bin-lan-constr}, our construction proves two different statements. For the second statement, conditions 2''), 3'') are written as a single system of equations with a single matrix $\matr{\Xi}$ and use $\Pi_\sflin$ to prove that certain vector of $\GG_1$ is in the span of $\matr{\Xi}$.

This matrix is defined as:
$$\matr{\Xi}(\matr{M},\matr{N},\matr{\Lambda},\matr{G}):=
\left(\begin{array}{ccc|ccc}
\matr{\Sigma}             & \cdots & \matr{0}                  & \matr{0} \\
\vdots                    & \ddots & \vdots                    & \vdots   \\
\matr{0}                  & \cdots & \matr{\Sigma}             & \matr{0} \\
\hline
\overline{\vecb{g}}_1^\lb & \cdots & \overline{\vecb{g}}_n^\lb & \vecb{g}_{n+1}^\lb
\end{array}\right),
$$ where $\matr{\Sigma}:=\begin{pmatrix}
    \matr{M}       & \matr{N}\\
    \matr{\Lambda} & \matr{0}
\end{pmatrix}$ and
 $\overline{\vecb{g}}_i^\lb:=\vecb{g}_i^\lb||\vecb{0}$, $i\in[n]$. 
 
Define $\vecb{y}:=\vecb{c}_1\oplus \grkb{\alpha}
\oplus \ldots \oplus \vecb{c}_n\oplus \grkb{\alpha} 
\oplus \vecb{d}_1 \oplus  \ldots \oplus  \vecb{d}_m$
and 
 $\vecb{v}:=\vecb{b}_1\oplus \vecb{w}_1 \oplus \ldots \oplus \vecb{b}_n\oplus \vecb{w}_n \oplus r_1 \oplus \ldots \oplus r_n.$ The statement we want to prove is that $[\vecb{y}]_1 \in \mathbf{Im}([\matr{\Xi}]_1)$, and the witness is $\vecb{v}$. The upper left block of the matrix guarantees condition 2''), while the two lower blocks guarantee 
condition 3'').

\iffull
\else
\input{bin-languages/protocol}
\fi

    \subsection{Security Proofs}
        \input{bin-languages/bin-lang-proof}
\section{Details of the Proof of Correctness of a Shuffle} \label{app:shuffle}
    \begin{figure} 
$$
\begin{array}{ll}
\begin{array}{l}
\algK_1(\Gamma,[\vecb{v}]_2,n)\quad (\mathsf{S}_1(\Gamma,[\vecb{v}]_2,n))\\
\hline
S:=\{[s_1]_1,\ldots,[s_n]_1\}, \matr{S}\gets\distlin_{1,n}\\
\crs_\GS\gets\GS.\algK_1(\Gamma)\\
{\crs_\sfbin\gets\Pi_\sfbin.\algK_1(\Gamma,[\matr{M}]_1,[\matr{N}]_1,[\matr{\Lambda}]_1,n)}\\
\text{Return } \ \crs:=([\vecb{v}]_2,\crs_\GS,\crs_\sfbin).\\
(\tau_\GS\gets\GS.\algS_1(\Gamma)\\
\tau_\sfbin\gets\Pi_\sfbin.\algS_1(\Gamma,L,[\matr{M}]_1,[\matr{N}]_1,[\matr{\Lambda}]_1,n).\\
\tau := (\matr{S},\tau_\GS,\tau_\sfbin)).\\
\\
\end{array}
&
\begin{array}{l}
{\algP(\crs, [\matr{C}]_2, [\matr{D}]_2,\langle \matr{P}, \grkb{\delta}\rangle)}\\
\hline
\vvar{X} := [\matr{S}]_1\matr{P}^\top,
\vvar{y} := [\matr{S}]_1\grkb{\delta}\\
\eq_1:= \sum_{i\in[n]}[s_i]_1-\sum_{j\in[n]}\vvar{x}_i = [0]_1\\
\eq_2:= {e(\vvar{X},[\matr{C}]_2^\top)}/{e([\matr{S}]_1,[\matr{D}]_2^\top)}=e(\vvar{y},[\vecb{v}]_2^\top)\\
\pi_\GS\gets\GS.\algP(\crs_\GS,\{\eq_1,\eq_2\},\vvar{X},\vvar{y})\\
\pi_\sfbin \gets \Pi_\sfbin.\algP(\crs_\sfbin, [\matr{F}]_1,\langle\matr{R},\matr{P}^\top\rangle)\\
//{[\matr{F}]_1} = \GS.\Com_{\crs_\GS}(\vvar{X};\matr{R})\\
\text{Return } \  (\pi_\GS,\pi_\sfbin). \\
\\
\end{array}\\
\begin{array}{l}
{\algV(\crs,[\matr{C}]_2,[\matr{D}]_2,(\pi_\GS,\pi_\sfbin))}\\
\hline
\eq_1:= \sum_{i\in[n]}[s_i]_1-\sum_{j\in[n]}\vvar{x}_i = [0]_1\\
\eq_2:= {e(\vvar{X},[\matr{C}]_2^\top)}/{e([\matr{S}]_1,[\matr{D}]_2^\top)}=e(\vvar{y},[\vecb{v}]_2^\top)\\
\mathsf{ans}_1 \gets \GS.\algV(\crs_\GS,\{\eq_1,\eq_2\},\pi_\GS)\\
\mathsf{ans}_2 \gets \Pi_\sfbin.\algV(\crs_\sfbin,[\matr{F}]_1,\pi_\sfbin)\\
\text{Return } \ \mathsf{ans}_1\wedge\mathsf{ans}_2.
\end{array}
&
\begin{array}{l}
{\mathsf{S}_2(\crs,[\matr{C}]_1,[\matr{D}]_1,\tau)}\\
\hline
\eq_1:= \sum_{i\in[n]}[s_i]_1-\sum_{j\in[n]}\vvar{x}_i = [0]_1\\
\eq'_2:= {e(\vvar{X},[\matr{C}]_2^\top)}/{e([1]_1,\matr{S}^\top[\matr{D}]_2^\top)}=e(\vvar{y},[\vecb{v}]_2^\top)\\
\pi_\GS\gets \GS.\algS_2(\crs_\GS,\{\eq_1,\eq'_2\},\tau_\GS)\\
\pi_\sfbin\gets \Pi_\sfbin.\algS_2(\crs_\sfbin,[\matr{F}]_1,\tau_\sfbin)\\
\text{Return }  (\pi_\GS,\pi_\sfbin).
\end{array}
\end{array}$$
\caption{The proof system $\Pi_\mathsf{shuffle}$ for the language $\Lang_{[\vecb{v}]_2,n,\mathsf{shuffle}}$. The matrices $\matr{M},\matr{N},\matr{\Lambda}$ are defined as $\matr{M}:=\usebox{\smat},\matr{N}:=(\vecb{u}_1||\vecb{u}_2),\matr{\Lambda}:= (1,\ldots,1)$, where $\vecb{u}_1,\vecb{u}_2$ are the GS commitment keys from $\crs_\GS$. The proof size is $|\pi_\GS|+|\pi_\sfbin|=(2n+6)|\GG_1|+4|\GG_2|+1|\Z_q|+(2n+11)|\GG_1|+10|\GG_2|=(4n+17)|\GG_1|+14|\GG_2|+1|\Z_q|$.
\label{fig:shuffles}}
\end{figure}


    \subsection{Security Proofs}
        \subsubsection{Completeness.} If $\matr{P}$ is a permutation matrix then $\vvar{X}=[\matr{S}]_1\matr{P}^\top=([s_{\pi(1)}]_1,\ldots,[s_{\pi_{n}}]_1)$, for some permutation $\pi$, which implies that $\GS.\Com(\vvar{X})\in\Lang_{S,\vecb{u}_1,\vecb{u}_2}^n$ and $\sum_{i\in[n]}[s_i]_1-\sum_{j\in[n]}\vvar{x}_j=[0]_1$. If $\vary=[\matr{S}]_1\grkb{\delta}$ then
\begin{eqnarray*}
\dfrac
    {e(\vvar{X},[\matr{C}]_2^\top)}
    {e([\matr{S}]_1,[\matr{D}]_2^\top)}
 = 
    e([\matr{S}]_1,([\matr{C}]_2\matr{P}-[\matr{D}]_2)^\top)
 = 
    e([\matr{S}]_1,([\vecb{v}]_2\grkb{\delta}^\top)^\top)
 =
    e(\vary,[\vecb{v}]_2^\top).
\end{eqnarray*}

\subsubsection{Soundness.}
\begin{theorem}
For any adversary $\advA$ against the soundness of the proof system from Figure \ref{fig:shuffles}, there exist an adversary $\advB_1$ against soundness of $\Pi_{\sfbin}$ and an adversary $\advB_2$ against the $\dist_{1,n}\mbox{-}\kermdh$ Assumption such that
$$\adv(\advA) \leq \adv_{\Pi_{\sfbin}}(\advB_1) + \adv_{\dist_{1,n}\mbox{-}\kermdh}(\advB_2).
$$
\end{theorem}

\begin{proof}
The adversary $\advB_1$ receives as input $\crs_\sfbin$ and honestly samples the rest of the CRS. Then $\advB_1$ runs $\advA$ until it halts and outputs $[\matr{F}]_1$ with the proof $\pi_\sfbin$.

The adversary $\advB_2$ receives as input $[\matr{S}]_1\in\GG_1^{1\times n}$, samples $\vecb{u}_1,\vecb{u}_2,\vecb{v}\gets \Z_q^2$, honestly simulates the rest of the CRS, and runs $\advA$ until it halts. It extracts $[\matr{X}]_1$, the opening of $[\matr{F}]_1$, using $\vecb{u}_1,\vecb{u}_2$, and aborts if $[\matr{F}]_1\notin\Lang_{S,\vecb{u}_1,\vecb{u}_2}^n$. Else $[\matr{X}]_1=[\matr{S}]_1\matr{B}$, where $\matr{B}\in\bits^{n\times n}$ and $\matr{1}_{1\times n}\matr{B}=\matr{1}_{1\times n}$. If $\sum_{i\in[n]}[s_i]_1-\sum_{j\in[n]}[x_j]_1\neq[0]_1$, $\advB$ outputs $\matr{1}_{n\times 1}-\matr{B}\matr{1}_{n\times 1}$. Else, using $\vecb{v}$, $\advB_2$ decrypts $[\matr{C}]_2$ and $[\matr{D}]_2$ obtaining $[\vecb{M}]_2\in\GG_2^{1\times n}$ and $[\vecb{M}']_2\in\GG_2^{1\times n}$, respectively, and returns $[\vecb{M}]_2\matr{B}^\top-[\vecb{M}']_2$.

Let $E_1$ the event where $([\matr{C}]_2,[\matr{D}]_2)\in\Lang_{[\vecb{v}]_2,n,\mathsf{shuffle}}$, $E_2$ the event where $[\matr{F}]_1\in\Lang_{S,\vecb{u}_1,\vecb{u}_2}^n$, and $E_3$ the event where $\matr{1}_{n\times 1}-\matr{B}\matr{1}_{n\times 1}=\vecb{0}_{n\times 1}\wedge[\vecb{M}]_2\matr{B}-[\vecb{M}']_2=0$. Note that $E_2\wedge E_3\Longrightarrow E_1$, because $E_2$ implies that $[\matr{X}]_1=[\matr{S}]_1\matr{B}$, where $\matr{B}\in\bits^{n\times n}$ and $\matr{1}_{1\times n}\matr{B}=\matr{1}_{1\times n}$. Together with $E_3$ implies that $\matr{B}$ is a permutation and thus $[\matr{C}]_2\matr{B}^\top-[\matr{D}]_2=\Enc([\matr{0}_{1\times n}]_2)\Longleftrightarrow E_1$. Note also that $\eq_1\wedge\eq_2\wedge\neg E_3$ implies that $\matr{1}_{n\times 1}-\matr{B}\matr{1}_{n\times 1}$ or $[\vecb{M}]_2\matr{B}-[\vecb{M}']_2$ are solutions to the $\dist_{1,n}\mbox{-}\kermdh$. Then it holds that
\begin{eqnarray*}
\adv(\advA) & = &
    \Pr[\neg E_1 \wedge \algV(\crs,([\matr{C}]_2,[\matr{D}]_2),\pi)=1]\\
& = &
    \Pr[\neg E_1 \wedge \algV(\crs,([\matr{C}]_2,[\matr{D}]_2),\pi)=1 \wedge \neg E_2]+\\
& &
    \Pr[\neg E_1 \wedge \algV(\crs,([\matr{C}]_2,[\matr{D}]_2),\pi)=1 \wedge E_2]\\
& \leq &
    \Pr[\neg E_2 \wedge \Pi_\sfbin.\algV(\crs_\sfbin,[\matr{F}]_2,\pi_\sfbin)=1]+
    \Pr[\neg E_1 \wedge \eq_1 \wedge \eq_2 \wedge E_2]\\
& \leq &
    \adv_{\Pi_\sfbin}(\advB_1)+\Pr[\neg E_1 \wedge \eq_1 \wedge \eq_2 \wedge E_2\wedge E_3] + \Pr[\neg E_1 \wedge \eq_1 \wedge \eq_2 \wedge E_2\wedge \neg E_3]\\
& \leq &
    \adv_{\Pi_\sfbin}(\advB_1) + \Pr[\neg E_1\wedge E_2 \wedge E_3]+
    \Pr[\eq_1\wedge\eq_2\wedge\neg E_3]\\
& = &
    \adv_{\Pi_\sfbin}(\advB_1)+0+\adv_{\dist_{1,n}\mbox{-}\kermdh}(\advB_2).
\end{eqnarray*} 
\end{proof}

\subsubsection{Zero-Knowledge}. We need to check that the inputs to the simulators are true statements and, for the GS simulator, that the equations allow simulation. This is certainly true for $\mathsf{eq}_1$ and, if $([\matr{C}]_1,[\matr{D}]_1)\in\Lang_{[\vecb{u}]_1,n,\mathsf{shuffle}}$, then is also true for $\mathsf{eq}'_2$. Furthermore, it is guaranteed that $\pi_\GS$ is computationally indistinguishable from a real proof for $\{\eq_1,\eq'_2\}$, which is identically distributed to a real proof for $\{\eq_1,\eq_2\}$ since $\eq'_2$ and $\eq_2$ accepts the same set of solutions.
Finally, since the perfectly hiding $\crs_\GS$ is such that $\rank(\vecb{u}_1||\vecb{u}_2)=2$, then $\Lang_{S,\vecb{u}_1,\vecb{u}_2}^n=\GG_1^{2\times n}$ and thus $[\matr{F}]_1\in\Lang_{S,\vecb{u}_1,\vecb{u}_2}^n$ is always true.

\section{Details of the Range Argument} \label{app:range-proof}
    \begin{figure} 
$$
\begin{array}{ll}
\begin{array}{l}
\algK_1(\Gamma,[\vecb{u}]_1,[\vecb{u}_2]_1,n)\quad (\mathsf{S}_1(\Gamma,[\vecb{u}_1]_1,[\vecb{u}_2]_1,n))\\
\hline
d:=\log n^k, m:=\log d, \ell := n/m\\
\crs_\GS\gets\GS.\algK_1(\Gamma)\\
{\crs_\sfbin\gets\Pi_\sfbin.\algK_1(\Gamma,[\matr{M}]_1,[\matr{N}]_1,n)}\\
\text{Return } \ \crs:=([\vecb{u}_1]_1,[\vecb{u}_2]_1,\crs_\GS,\crs_\sfbin).\\
(\tau_\GS\gets\GS.\algS_1(\Gamma)\\
\tau_\sfbin\gets\Pi_\sfbin.\algS_1(\Gamma,[\matr{M}]_1,[\matr{N}]_1,n).\\
\tau := (\matr{S},\tau_\GS,\tau_\sfbin)).\\
\\
\end{array}
&
\begin{array}{l}
{\algP(\crs, [\matr{c}]_1, \langle y, r\rangle)}\\
\hline
\vvar{Y}\in\Z_q^{\ell\times1} \text{ is s.t. } y=\sum_{i\in[\ell]}\vvar{y}_id^{i-1}\\
\vvar{z}:=r\\
\pi_\GS\gets\GS.\algP(\crs_\GS,\eq,\vvar{Y},\vvar{z})\\
\pi_\sfbin \gets \Pi_\sfbin.\algP(\crs_\sfbin, [\matr{X}]_1,\langle\vvar{Y},\matr{R}\rangle)\\
//{[\matr{X}]_1} = \GS.\Com_{\crs_\GS}(\vvar{Y};\matr{R})\\
\text{Return } \  (\pi_\GS,\pi_\sfbin). \\
\\
\\
\end{array}\\
\begin{array}{l}
{\algV(\crs,[\vecb{c}]_1,(\pi_\GS,\pi_\sfbin))}\\
\hline
\mathsf{ans}_1 \gets \GS.\algV(\crs_\GS,\eq,\pi_\GS)\\
\mathsf{ans}_2 \gets \Pi_\sfbin.\algV(\crs_\sfbin,[\matr{X}]_1,\pi_\sfbin)\\
\text{Return } \ \mathsf{ans}_1\wedge\mathsf{ans}_2.
\\
\\
\end{array}
&
\begin{array}{l}
{\mathsf{S}_2(\crs,[\vecb{c}]_1,\tau)}\\
\hline
\pi_\GS\gets \GS.\algS_2(\crs_\GS,\eq,\tau_\GS)\\
\pi_\sfbin\gets \Pi_\sfbin.\algS_2(\crs_\sfbin,[\matr{X}]_1,\tau_\sfbin)\\
//{[\matr{X}]_1=\GS.\Com_{\crs_\GS}(\vvar{Y};\matr{R})}\\
\text{Return }  (\pi_\GS,\pi_\sfbin).
\end{array}
\end{array}$$
\caption{The proof system $\Pi_\mathsf{range-proof}$ for the language $\Lang_{[0,2^n-1],\vecb{u}'_1,\vecb{u}'_2}$. The matrices $\matr{M},\matr{N}$ are defined as $\matr{M}:=\vecb{u}'_1(2^0,2^1,\ldots,2^{m-q}),\matr{N}:=\vecb{u}'_2$, where $\vecb{u}'_1,\vecb{u}'_2$ are the GS commitment keys from $\crs_\GS$ ({\color{red} Parece q, estupidamente, hay q usar otras claves de commitment.}). The equation $\eq$ is defined as $[\vecb{c}]_1-\sum_{i\in[\ell]}\vvar{y}_id^{i-1}[\vecb{u}_2]_1=\vvar{z}[\vecb{u}_2]_1$.
\label{fig:rp}}
\end{figure}


    \subsection{Security Proofs}
        \subsubsection{Completeness.}
If $[\vecb{c}]_1=\GS.\Com(y;r)$ and $y\in[0,2^n-1]$, then there exists $y_1,\ldots,y_\ell\in[0,d-1]$  such that $y=\sum_{i\in[\ell]}y_id^{i-1}$. Therefore $\vecb{y}=(y_1,\ldots,y_\ell)^\top$ and $r$ are solutions to $\eq$ and $[\matr{X}]_1=\GS.\Com(\vecb{y}^\top)\in\Lang_{\vecb{u}'_1,\vecb{u}'_2,d}^\ell$.

\subsubsection{Soundness.}
Given an adversary $\advA$ against the soundness of the proof system from Fig. \ref{fig:rp}, we construct an adversary $\advB$ against the soundness of $\Pi_\sfbin$. If $y\notin[0,2^n-1]$, then the perfect soundness of GS proofs implies that there is some $y_i\notin[0,d-1]$. Therefore $[\matr{X}]_1=\GS.\Com(\vecb{y}^\top)\notin\Lang_{\vecb{u}'_1,\vecb{u}'_2,d}^\ell$ and $\adv_{\Pi_\mathsf{range-proof}}(\advA)\leq\adv_{\Pi_\sfbin}(\advB)$.

\subsubsection{Zero-Knowledge.} Follows directly from Zero-Knowledge of GS proofs, the fact that $\eq$ allows simulation, and the fact that $[\matr{X}]_1=\GS.\Com(\matr{0}_{1\times n})\in\Lang_{d,\vecb{u}'_1,\vecb{u}'_2}^\ell$.

\end{document}


