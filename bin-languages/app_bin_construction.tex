%
%
%The following definition will be useful for our construction
%\begin{definition}
%Given matrices $\matr{M}\in\Z_q^{k\times \lb},\matr{N}\in\Z_q^{k\times m},\matr{\Lambda}\in\Z_q^{t\times \lb}$ and $\matr{G}=(\vecb{g}_1||\cdots||\vecb{g}_{n+1})\in\Z_q^{k\times (n+1)}$, we define
%$$
%\matr{\Xi}(\matr{M},\matr{N},\matr{\Lambda},\matr{G}):=
%\left(\begin{array}{c|c|c}
%\matr{I}_{n\times n}\otimes\matr{M}         & \matr{I}_{n\times n}\otimes\matr{N} & \matr{0}_{kn\times \lb}                          \\
%\hline
%\matr{I}_{n\times n}\otimes\matr{\Lambda}   & \matr{0}_{t n\times mn}         & \matr{0}_{t n\times \lb}                     \\
%\hline
%\begin{array}{ccc}
%    \matr{I}_{\lb\times \lb}\otimes\vecb{g}_1&
%    \cdots &
%    \matr{I}_{\lb\times \lb}\otimes\vecb{g}_n
%\end{array}                                 & \matr{0}_{k \lb \times mn}         & \matr{I}_{\lb\times \lb}\otimes\vecb{g}_{n+1}
%\end{array}\right).
%$$
%Or more explicitly
%$$
%\matr{\Xi}(\matr{M},\matr{N},\matr{\Lambda},\matr{G}):=
%\left(\begin{array}{ccc|ccc|ccc}
%\matr{M}               & \cdots & \matr{0}       & \matr{N} & \cdots & \matr{0} & \matr{0} & \cdots & \matr{0} \\
%\vdots                 & \ddots & \vdots         & \vdots   & \ddots & \vdots   & \vdots   & \ddots & \vdots   \\
%\matr{0}               & \cdots & \matr{M}       & \matr{0} & \cdots & \matr{N} & \matr{0} & \cdots & \matr{0} \\
%\hline
%\matr{\Lambda}         & \cdots & \matr{0}       & \matr{0} & \cdots & \matr{0} & \vecb{0} & \cdots & \matr{0} \\
%\vdots                 & \ddots & \vdots         & \vdots   & \ddots & \vdots   & \vdots   & \ddots & \vdots   \\
%\matr{0}               & \cdots & \matr{\Lambda} & \matr{0} & \cdots & \matr{0} & \matr{0} & \cdots & \matr{0} \\
%\hline
%    \begin{array}{ccc}
%        \vecb{g}_1 & \cdots & \vecb{0}
%    \end{array} &
%    &
%    \begin{array}{ccc}
%        \vecb{g}_n & \cdots & \vecb{0}    
%    \end{array} &
%    \matr{0} & \cdots & \matr{0} &
%    \vecb{g}_{n+1} & \cdots & \vecb{0} \\
%    \begin{array}{ccc}
%        \vdots & \ddots & \vdots
%    \end{array} &
%    \cdots &
%    \begin{array}{ccc}
%        \vdots & \ddots & \vdots   
%    \end{array} &
%    \vdots & \ddots & \vdots &
%    \vdots & \ddots & \vdots \\
%    \begin{array}{ccc}
%        \vecb{0} & \cdots & \vecb{g}_1
%    \end{array} &
%    &
%    \begin{array}{ccc}
%        \vecb{0} & \cdots & \vecb{g}_n   
%    \end{array} &
%    \matr{0} & \cdots & \matr{0} &
%    \vecb{0} & \cdots & \vecb{g}_{n+1}
%\end{array}\right)
%$$
%\end{definition}
%
%The following lemma will be useful in the security proofs.
%\begin{lemma}
%Let $\matr{B}=(\vecb{b}_1||\cdots||\vecb{b}_n)\in\Z_q^{\lb\times n},\matr{W}\in\Z_q^{m\times n},\vecb{r}\in\Z_q^\lb$, then
%$$
%\pmatri{\matr{X}\\\grkb{\alpha}||\cdots||\grkb{\alpha}} =
%\begin{pmatrix}
%\matr{M}       &  \matr{N}\\
%\matr{\Lambda} & \matr{0}_{t\times m}
%\end{pmatrix}
%\pmatri{\matr{B}\\\matr{W}}
%\text{ and }
%[\matr{C}]_1=\MP.\Com_{[\matr{G}]_1}(\matr{B}^\top;\vecb{r}).
%$$
%if and only if
%\begin{equation*}
%\pmatri{\tvecb{x}\\\tgrkb{\alpha}\\\tvecb{c}} = \matr{\Xi}(\matr{M},\matr{N},\matr{\Lambda},\matr{G})\pmatri{\tvecb{b}\\\tvecb{w}\\\vecb{r}},
%\end{equation*}
%where $\tvecb{x}:=\uvecb{X}$, $\tvecb{c}:=\uvecb{C}$, $\tvecb{b}:=\uvecb{B}$, $\tvecb{w}:=\uvecb{W}$, and $\tgrkb{\alpha}:=\vect(\grkb{\alpha}||\cdots||\grkb{\alpha})$.
%\end{lemma}\label{fact:popo}
%\begin{proof}
%$$
%\pmatri
%{
%\matr{X}\\
%\grkb{\alpha}||\cdots||\grkb{\alpha}
%}
%=
%\begin{pmatrix}
%\matr{M}       & \matr{N}\\
%\matr{\Lambda} & \matr{0}_{t\times m}
%\end{pmatrix}
%\pmatri
%{
%    \matr{B}\\
%    \matr{W}
%}
% \Longleftrightarrow 
%\pmatri{\tvecb{x}\\\tgrkb{\alpha}} =
%\left(\begin{array}{c|c|c}
%\matr{I}_{n\times n}\otimes\matr{M}       & \matr{I}_{n\times n}\otimes\matr{N}
%& \matr{0}_{kn\times\lb}\\
%\hline
%\matr{I}_{n\times n}\otimes\matr{\Lambda} & \matr{0}_{t n\times mn}
%& \matr{0}_{t n\times\lb}
%\end{array}\right)
%\pmatri{\tvecb{b}\\\tvecb{w}\\\vecb{r}}.
%$$
%and
%\begin{eqnarray*}
%[\matr{C}]_1=\MP.\Com_{[\matr{G}]_1}(\matr{B}^\top;\vecb{r})
%& \Longleftrightarrow &
%\matr{C} = \sum_{i\in[n]}\vecb{g}_i\vecb{b}_i^\top + \vecb{g}_{n+1}\vecb{r}^\top\\
%& \Longleftrightarrow &
%\tvecb{c} = \sum_{i\in[n]}(\matr{I}_{\lb\times\lb}\otimes\vecb{g}_i)\vecb{b}_i+(\matr{I}_{\lb\times\lb}\otimes\vecb{g}_{n+1})\vecb{r}\\
%& \Longleftrightarrow &
%\tvecb{c} = (\matr{I}_{\lb\times\lb}\otimes\vecb{g}_1||\cdots||\matr{I}_{\lb\times\lb}\otimes\vecb{g}_n||\matr{0}_{k\lb\times mn}||\matr{I}_{\lb\times\lb}\matr{g}_{n+1})\pmatri{\tvecb{b}\\\tvecb{w}\\\vecb{r}}
%\end{eqnarray*}
%which allows us to conclude.
%\end{proof}


In Section~\ref{sec:bin-lan-constr} we gave a description of the basic steps of the proof
%which were: 
%\begin{enumerate}
%\item Construct a MP commitment $[\vecb{c}_i]_1$ to $\vecb{b}^*_i$, for all $i \in [\lb]$,
%\item Prove that $[\vecb{c}_i]_1$ can be opened to some vector $\vecb{b}^*_i \in \{0,1\}^{n}$ for all $i \in [\lb]$, 
%\item Prove that $[\vecb{c}]_1:=[\vecb{c}_1]_1 \oplus \ldots \oplus [\vecb{c}_n]_1$ and $[\vecb{x}_1]_1 \oplus \ldots \oplus [\vecb{x}_n]_1$ open to the same value,
%\item Prove that $[\widetilde{\vecb{x}}]_1 \in \mathbf{Im}([\matr{\Sigma^n}]_1)$, 
%\end{enumerate}
%In fact,
but now, for efficiency, we fusionate Steps 3,4 in a single one. More specifically, the basic steps of the proof are:
\begin{enumerate}
\item[a)] Construct a MP commitment $[\vecb{c}_i]_1$ to $\vecb{b}^*_i$, for all $i \in [\lb]$,
\item[b)] Prove that  $[\vecb{c}_i]_1$  can be opened to some $\vecb{b}^*_i \in \{0,1\}^{n}$ for all $i \in [\lb]$, 
\item[c)] Prove that $\begin{pmatrix}[\widetilde{\vecb{x}}]_1 \\ [\vecb{c}]_1 \end{pmatrix} \in \mathbf{Im}([\matr{\Xi}]_1)$, 
\end{enumerate}
where the matrix $\matr{\Xi}$ is defined so that Step c) implies Steps 3,4, namely.
%Let $\matr{B}:=(\vecb{b}_1||\cdots||\vecb{b}_n)$, $\matr{X}:=(\vecb{x}_1||\cdots||\vecb{x}_n)$, $\matr{C}:=(\vecb{c}_1||\cdots||\vecb{c}_\lb)$, and $\matr{W}:=(\vecb{w}_1||\cdots||\vecb{w}_n)$. Note that they satisfy the equations $\matr{X}=\matr{\Sigma}\smallpmatrix{\matr{B}\\\matr{W}}$ and $\matr{C}=\matr{G}\smallpmatrix{\matr{B}^\top\\\vecb{r}^\top}$, which can be written as the linear systems $\tvecb{x}=\matr{\Sigma}^n\tvecb{b}$ and $\vecb{c}=\matr{G}^n \matr{T}(\vecb{b})$, where $\matr{T}$ is the matrix defining the transpose when matrices in $\Z_q^{\la\times\lb}$ are viewed as vectors in $\Z_q^{\la\lb}$. 

$$\matr{\Xi}(\matr{M},\matr{N},\matr{\Lambda},\matr{G}):=
\left(\begin{array}{ccc|ccc}
\matr{\Sigma}             & \cdots & \matr{0}                  & \matr{0} \\
\vdots                    & \ddots & \vdots                    & \vdots   \\
\matr{0}                  & \cdots & \matr{\Sigma}             & \matr{0} \\
\hline
\overline{\vecb{g}}_1^\lb & \cdots & \overline{\vecb{g}}_n^\lb & \vecb{g}_{n+1}^\lb
\end{array}\right),
$$ where $\overline{\vecb{g}}_i^\lb:=\vecb{g}_i^\lb||\vecb{0}$, $i\in[n]$. To see why this is the right definition of $\matr{\Xi}$, we argue for simplicity in the case where $n=2$. In this case, the matrix looks like:   
$$\matr{\Xi}(\matr{M},\matr{N},\matr{\Lambda},\matr{G})=
\left(\begin{array}{cccc}
    \matr{\Sigma}             & \matr{0}                  & \matr{0}           \\
    \matr{0}                  & \matr{\Sigma}             & \matr{0}           \\
    \overline{\vecb{g}}_1^\lb & \overline{\vecb{g}}_2^\lb & \vecb{g}_{3}^\lb
\end{array}\right)
=
\left(\begin{array}{ccccc}
    \matr{M}       & \matr{N}  & \matr{0}       & \matr{0} & \matr{0}         \\
    \matr{\Lambda} & \matr{0}  & \matr{0}       & \matr{0} & \matr{0}         \\
    \matr{0}       & \matr{0}  & \matr{M}       & \matr{N} & \matr{0}         \\
    \matr{0}       & \matr{0}  & \matr{\Lambda} & \matr{0} & \matr{0}         \\
    \vecb{g}_1^\lb & \vecb{0}  &\vecb{g}_2^\lb  & \vecb{0} & \vecb{g}_{3}^\lb
\end{array}\right).
$$ 

The statement $\begin{pmatrix}[\widetilde{\vecb{x}}]_1 \\ [\vecb{c}]_1 \end{pmatrix} \in \mathbf{Im}([\matr{\Xi}]_1)$ means that one proves that
$[\tilde{\vecb{x}}]_1$ and $[\vecb{c}]_1$ share a common opening, but it also proves that each $[\vecb{c}_i]_1$ is computed from only certain parts of the opening. For example, if $\vecb{x}_i =\matr{\Sigma}\smallpmatrix{\vecb{b}_i\\\vecb{w}_i}$, then $\vecb{c}_1=b_{1,1}\vecb{g}_1+b_{2,1}\vecb{g}_2+r_1\vecb{g}_3$ (recall that $\vecb{g}_i^\lb=\smallpmatrix{\vecb{g}_i&\cdots&\vecb{0}\\\vdots&\ddots&\vdots\\\vecb{0}&\cdots&\vecb{g}_i}$). Therefore, we are also proving that $[\vecb{c}_1]_1$ is a commitment to $\vecb{b}_1^*$, the first row of $\matr{B}=(\vecb{b}_1||\vecb{b}_2)$.
 
%  $$\begin{pmatrix}[\vecb{x}_1]_1 \\ [\vecb{\alpha}]_1 \\ [\vecb{x}_2]_1 \\ [\vecb{\alpha}]_1 \\ [\vecb{c}_1]_1 \\ [\vecb{c}_2]_1  \end{pmatrix} = \left(\begin{array}{cccccc}
%    \matr{M}       & \matr{N}  & \matr{0} & \matr{0} & \matr{0} & \matr{0}\\
%    \matr{\Lambda} & \matr{0}  & \matr{0} & \matr{0} & \matr{0} & \matr{0} \\
% \matr{0} & 
%\matr{0} &  \matr{M}       & \matr{N} &  \matr{0} & \matr{0}\\
% \matr{0} & \matr{0}  & \matr{\Lambda} & \matr{0} & \matr{0} & \matr{0} \\
%\matr{G}_1  & \vecb{0}&\vecb{0} & \vecb{0} & 
%\matr{G}_2 & \matr{0} \\
%\matr{0}  & \matr{0} & \vecb{G}_1 & \matr{0} &  \matr{0} & \matr{G}_2
%\end{array}\right) \begin{pmatrix} \vecb{b}_1 \\ \vecb{w}_1 \\ \vecb{b}_2 \\
%  \vecb{w}_2 \\ \vecb{r}_1 \\
%  \vecb{r}_2 \end{pmatrix}
%$$ 
%  for some $\vecb{b}_1,\vecb{b}_2 \in \Z_q^{\lb},\vecb{w}_1,\vecb{w}_2 \in \Z_q^{\lc},\vecb{r}_1,\vecb{r}_2 \in \Z_q^{\la}$.
%  In particular, it is obvious from this write-up that this implies both that $[\vecb{x}_j]_1 \in \Lang_{[\matr{M}]_1,[\matr{N}]_1,\matr{\Lambda},\grkb{\alpha}}$ for all $j \in [2]$ (which is equivalent to 
%  $[\tilde{\vecb{x}}]_1 \in \mathbf{Im}(\matr{\Sigma}^2)$
%  and that $[\vecb{x}_j]_1$ and $[\vecb{c}_j]_1$, for all
%   $j \in [2]$ share a common opening. 
%   
% In the following, we extend the message space of the MP commitment to matrices  $\matr{M}\in\Z_q^{m\times n}$ by defining $\MP.\Com_{[\matr{G}]_\gamma}(\matr{M};\vecb{W}):=[\matr{G}]_\gamma \pmatri{\matr{M}\\ \matr{W}}\in\GG_\gamma^{(k+1)\times n}$, for $\matr{W}\in\Z_q^{k \times n}$. The columns of the resulting commitment are the commitments to the columns of $\matr{M}$.   
%    For simplicity, we take 
% $k=1$ in the definition of the MP commitment. 
 
 
   
\begin{figure} 
$$
\begin{array}{ll}
\begin{array}{l}
\algK_1(\gk,[\matr{M}]_1,[\matr{N}]_1,n)
\quad (\mathsf{S}_1(\gk,[\matr{M}]_1,[\matr{N}]_1,n))
\\
\hline
[\matr{G}]_1 \gets \MP.\algK(1^\lambda,n)\\
{[\matr{\Xi}]_1 := [\matr{\Xi}(\matr{M},\matr{N},\matr{\Lambda},\matr{G})]_1}\\
\crs_\sflin\gets\Pi_\sflin.\algK_1(\gk,[\matr{\Xi}]_1)\\
\crs_\sfbits\gets\Pi_\sfbits.\algK_1(\gk,[\matr{G}]_1,\lb)\\
\text{Return } \ \crs:=(\crs_\sflin,\crs_\sfbits).\\
(\tau_\sflin\gets\Pi_\sflin.\algS_1(\gk,[\matr{\Xi}]_1)\\
\tau_\sfbits\gets\Pi_\sfbits.\algS_1(\gk,[\matr{G}]_1,\lb).\\
\tau := (\tau_\sflin,\tau_\sfbits)).\\
\\
\end{array}
&
\begin{array}{l}
{\algP(\mathsf{crs}, \{[\vecb{x}_i]_1,\langle \vecb{b}_i,\vecb{w}_i\rangle:i\in[n]\})}\\
\hline
{[\vecb{c}_j]_1} := \MP.\Com_{[\matr{G}]_1}(\vecb{b}_j^*;r_j),\\
r_j \gets\Z_q, \forall j\in[\lb]\\
\pi_\sflin \gets 
    \Pi_\sflin.\algP
    (
        \crs_\sflin,\\
\qquad
        
            [\tvecb{x}]_1\oplus
            [\vecb{c}]_1,
            \tilde{\vecb{b}}\oplus\vecb{r}
    )\\
\pi_\sfbits \gets
    \Pi_\sfbits.\algP
    (
        \crs_\sfbits,
        \{[\matr{c}_j]_1,\\
\qquad
        \langle\matr{b}^*_j,r_j\rangle:j\in[\lb]\}
    )\\
\text{Return } \  ([\vecb{c}]_1,\pi_\sflin,\pi_\sfbits). \\
\\
\\
\end{array}\\
\begin{array}{l}
{\algV(\mathsf{crs},\{[\vecb{x}_i]_1:i\in[n]\},([\vecb{c}]_1,\pi_\sflin,\pi_\sfbits))}\\
\hline
\mathsf{ans}_1 \gets
    \Pi_\sflin.\algV
    (
        \crs_\sflin,
            [\tvecb{x}]_1\oplus
            [\vecb{c}]_1,
        \pi_\sflin
    )\\
\mathsf{ans}_2 \gets \Pi_\sfbits.\algV(\crs_\sfbits,\{[\vecb{c}_j]_1:j\in[\lb]\},\\
\qquad\quad\pi_\sfbits)\\
\text{Return } \ \mathsf{ans}_1\wedge\mathsf{ans}_2.
\\
\\
\\
\end{array}
&
\begin{array}{l}
{\mathsf{S}_2(\crs,[\vecb{x}]_1,\tau)}\\
\hline
{[\vecb{c}_j]_1} := \MP.\Com_{[\matr{G}]_1}(\matr{0}_{n\times 1};\tilde{{r}}_j)\\
\tilde{{r}}_j\gets\Z_q^\lb, \forall j\in[\lb]\\
\pi_\sflin \gets 
    \Pi_\sflin.\algS
    (
        \crs_\sflin,
            [\tvecb{x}]_1\oplus
            [\vecb{c}]_1,
       \tau_\sflin
    )\\
\pi_\sfbits \gets
    \Pi_\sfbits.\algS
    (
        \crs_\sfbits,\\
\qquad\quad  \{[\vecb{c}_j]_1:j\in[\lb]\},
        \tau_\sfbits
    )\\
\text{Return } \  ([\vecb{c}]_1,\pi_\sflin,\pi_\sfbits). \\
\end{array}
\end{array}$$
\caption{Proof system for the language $\Lang_{\matr{M},\matr{N},\matr{\Lambda},\grkb{\alpha}}^n$, where $\Pi_\sfbits$ is the proof system for $\Lang_{ck,\sfbits}^n$ from Sect. \ref{sec:bits-n}, $\vecb{c}:=\vecb{c}_1\oplus\ldots\oplus\vecb{c}_\lb$, $\vecb{b}:=\vecb{b}_1\oplus\ldots\oplus\vecb{b}_n$, $\tvecb{x}:=\vecb{x}_1\oplus\grkb{\alpha}\oplus\ldots\oplus\vecb{x}_n\oplus\grkb{\alpha}$, and $\tilde{\vecb{b}}:=\vecb{b}_1\oplus\vecb{w}_1\oplus\ldots\vecb{b}_n\oplus\vecb{w}_n$. The proof size is $(2\lb+11)|\GG_1|+10|\GG_2|$.\label{fig:bin-leng-nizk}}
\end{figure}
%
% and where $\tvecb{x}:=\uvecb{X}$, $\tvecb{c}:=\uvecb{C}$, $\tvecb{b}:=\uvecb{B}$, $\tvecb{w}:=\uvecb{W}$, and $\tgrkb{\alpha}:=\vect(\grkb{\alpha}||\cdots||\grkb{\alpha})$.
