As announced in Section~\ref{sec:bin-lan-constr}, our construction proves two different statements. For the second statement, conditions 2''), 3'') are written as a single system of equations with a single matrix $\matr{\Xi}$ and use $\Pi_\sflin$ to prove that certain vector of $\GG_1$ is in the span of $\matr{\Xi}$.

This matrix is defined as:
$$\matr{\Xi}(\matr{M},\matr{N},\matr{\Lambda},\matr{G}):=
\left(\begin{array}{ccc|ccc}
\matr{\Sigma}             & \cdots & \matr{0}                  & \matr{0} \\
\vdots                    & \ddots & \vdots                    & \vdots   \\
\matr{0}                  & \cdots & \matr{\Sigma}             & \matr{0} \\
\hline
\overline{\vecb{g}}_1^\lb & \cdots & \overline{\vecb{g}}_n^\lb & \vecb{g}_{n+1}^\lb
\end{array}\right),
$$ where $\matr{\Sigma}:=\begin{pmatrix}
    \matr{M}       & \matr{N}\\
    \matr{\Lambda} & \matr{0}
\end{pmatrix}$ and
 $\overline{\vecb{g}}_i^\lb:=\vecb{g}_i^\lb||\vecb{0}$, $i\in[n]$. 
 
Define $\vecb{y}:=\vecb{c}_1\oplus \grkb{\alpha} \oplus
\oplus \ldots \oplus \vecb{c}_n\oplus \grkb{\alpha} 
\oplus \vecb{d}_1 \oplus  \ldots \oplus  \vecb{d}_m$
and 
 $\vecb{v}:=\vecb{b}_1\oplus \vecb{w}_1 \oplus \ldots \oplus \vecb{b}_n\oplus \vecb{w}_n \oplus r_1 \oplus \ldots \oplus r_n.$ The statement we want to prove is that $[\vecb{y}]_1 \in \mathbf{Im}([\matr{\Xi}]_1)$, and the witness is $\vecb{v}$. The upper left block of the matrix guarantees condition 2''), while the two lower blocks guarantee 
condition 3'').  
%  
%    To see why this is the right definition of $\matr{\Xi}$, we argue for simplicity in the case where $n=2$. In this case, the matrix looks like:   
%$$\matr{\Xi}(\matr{M},\matr{N},\matr{\Lambda},\matr{G})=
%\left(\begin{array}{cccc}
%    \matr{\Sigma}             & \matr{0}                  & \matr{0}           \\
%    \matr{0}                  & \matr{\Sigma}             & \matr{0}           \\
%    \overline{\vecb{g}}_1^\lb & \overline{\vecb{g}}_2^\lb & \vecb{g}_{3}^\lb
%\end{array}\right)
%=
%\left(\begin{array}{ccccc}
%    \matr{M}       & \matr{N}  & \matr{0}       & \matr{0} & \matr{0}         \\
%    \matr{\Lambda} & \matr{0}  & \matr{0}       & \matr{0} & \matr{0}         \\
%    \matr{0}       & \matr{0}  & \matr{M}       & \matr{N} & \matr{0}         \\
%    \matr{0}       & \matr{0}  & \matr{\Lambda} & \matr{0} & \matr{0}         \\
%    \vecb{g}_1^\lb & \vecb{0}  &\vecb{g}_2^\lb  & \vecb{0} & \vecb{g}_{3}^\lb
%\end{array}\right).
%$$ 
%
%
%The statement $\begin{pmatrix}[\widetilde{\vecb{c}}]_1 \\ [\vecb{d}]_1 \end{pmatrix} \in \mathbf{Im}([\matr{\Xi}]_1)$ means that one proves that
%$[\tilde{\vecb{c}}]_1$ and $[\vecb{d}]_1$ share a common opening, but it also proves that each $[\vecb{d}_i]_1$ is computed from only certain parts of the opening. For example, if $\vecb{c}_i =\matr{\Sigma}\smallpmatrix{\vecb{b}_i\\\vecb{w}_i}$, then $\vecb{d}_1=b_{1,1}\vecb{g}_1+b_{2,1}\vecb{g}_2+r_1\vecb{g}_3$ (recall that $\vecb{g}_i^\lb=\smallpmatrix{\vecb{g}_i&\cdots&\vecb{0}\\\vdots&\ddots&\vdots\\\vecb{0}&\cdots&\vecb{g}_i}$). Therefore, we are also proving that $[\vecb{d}_1]_1$ is a commitment to $\vecb{b}_1^*$, the first row of $\matr{B}=(\vecb{b}_1||\vecb{b}_2)$.
% 

 
   
\begin{figure}
\begin{\algSize}
$$
\begin{array}{ll}
\begin{array}{l}
\algK_1(\gk,[\matr{M}]_1,[\matr{N}]_1,n)
\quad (\mathsf{S}_1(\gk,[\matr{M}]_1,[\matr{N}]_1,n))
\\
\hline
[\matr{G}]_1 \gets \MP.\algK(1^\lambda,n)\\
{[\matr{\Xi}]_1 := [\matr{\Xi}(\matr{M},\matr{N},\matr{\Lambda},\matr{G})]_1}\\
\crs_\sflin\gets\Pi_\sflin.\algK_1(\gk,[\matr{\Xi}]_1)\\
\crs_\sfbits\gets\Pi_\sfbits.\algK_1(\gk,[\matr{G}]_1,\lb)\\
\text{Return } \ \crs:=(\crs_\sflin,\crs_\sfbits).\\
(\tau_\sflin\gets\Pi_\sflin.\algS_1(\gk,[\matr{\Xi}]_1)\\
\tau_\sfbits\gets\Pi_\sfbits.\algS_1(\gk,[\matr{G}]_1,\lb).\\
\tau := (\tau_\sflin,\tau_\sfbits)).\\
\\
\end{array}
&
\begin{array}{l}
{\algP(\mathsf{crs}, \{[\vecb{c}_j]_1,\langle \vecb{b}_j,\vecb{w}_j\rangle:j\in[n]\})}\\
\hline
{[\vecb{d}_i]_1} := \MP.\Com_{[\matr{G}]_1}(\vecb{b}_i^*;r_i),\\
r_i \gets\Z_q, \forall i\in[\lb]\\
\pi_\sflin \gets 
    \Pi_\sflin.\algP
    (
        \crs_\sflin,
            [\vecb{y}]_1,
            \vecb{v}
    )\\
\pi_\sfbits \gets
    \Pi_\sfbits.\algP
    (
        \crs_\sfbits,
        \{[\matr{d}_i]_1,\\
\qquad
        \langle\matr{b}^*_i,r_i\rangle:i \in[\lb]\}
    )\\
\text{Return } \  ([\vecb{d}]_1,\pi_\sflin,\pi_\sfbits). \\
\\
\\
\\
\end{array}\\
\begin{array}{l}
{\algV(\mathsf{crs},\{[\vecb{c}_j]_1:j\in[n]\},([\vecb{d}]_1,\pi_\sflin,\pi_\sfbits))}\\
\hline
\mathsf{ans}_1 \gets
    \Pi_\sflin.\algV
    (
        \crs_\sflin,
            [\vecb{y}]_1,
        \pi_\sflin
    )\\
\mathsf{ans}_2 \gets \Pi_\sfbits.\algV(\crs_\sfbits,\{[\vecb{d}_i]_1:i\in[\lb]\},\pi_\sfbits)\\
\text{Return } \ \mathsf{ans}_1\wedge\mathsf{ans}_2.
\\
\\
\\
\\
\end{array}
&
\begin{array}{l}
{\mathsf{S}_2(\crs,[\vecb{c}]_1,\tau)}\\
\hline
{[\vecb{d}_i]_1} := \MP.\Com_{[\matr{G}]_1}(\matr{0}_{n\times 1};\tilde{{r}}_i)\\
\tilde{{r}}_i\gets\Z_q, \forall i\in[\lb]\\
\pi_\sflin \gets 
    \Pi_\sflin.\algS
    (
        \crs_\sflin,
            [\vecb{y}]_1,
       \tau_\sflin
    )\\
\pi_\sfbits \gets
    \Pi_\sfbits.\algS
    (
        \crs_\sfbits,\\
\qquad\quad  \{[\vecb{d}_i]_1:i\in[\lb]\},
        \tau_\sfbits
    )\\
\text{Return } \  ([\vecb{d}]_1,\pi_\sflin,\pi_\sfbits). \\
\end{array}
\end{array}$$
\end{\algSize}
\caption{Proof system for the language $\Lang_{\matr{M},\matr{N},\matr{\Lambda},\grkb{\alpha}}^n$, where $\Pi_\sfbits$ is the proof system for $\Lang_{ck,\sfbits}^m$ from Sect. \ref{sec:bits-n}, $\vecb{d}:=\vecb{d}_1\oplus\ldots\oplus\vecb{d}_\lb$, and $\vecb{c}:=\vecb{c}_1\oplus\ldots\oplus\vecb{c}_n$. The proof size is $(2\lb+11)|\GG_1|+10|\GG_2|$.\label{fig:bin-leng-nizk}}
\end{figure}
%
% and where $\tvecb{c}:=\uvecb{X}$, $\tvecb{c}:=\uvecb{C}$, $\tvecb{b}:=\uvecb{B}$, $\tvecb{w}:=\uvecb{W}$, and $\tgrkb{\alpha}:=\vect(\grkb{\alpha}||\cdots||\grkb{\alpha})$.
