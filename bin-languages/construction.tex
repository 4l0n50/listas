The main result of this Section is a proof, of roughly the same size as in the last Section ($2\lb|\GG_1|+\Theta(1)$), that  $([\vecb{c}_1]_1,\ldots,[\vecb{c}_n]_1)$ is in $\Lang_{[\matr{M}]_1,[\matr{N}]_1,\matr{\Lambda},\grkb{\alpha}}^n$.

For all $j \in [n]$, let $(\vecb{b}_j,\vecb{w}_j) \in \{0,1\}^{\lb} \times \Z_q^{\lc}$ be the witness of $\vecb{c}_j \in \Lang_{[\matr{M}]_1,[\matr{N}]_1,\matr{\Lambda},\grkb{\alpha}}.$ Let $\matr{B}=(\vecb{b}_1|| \ldots ||\vecb{b}_n)$ and let $\vecb{b}^*_{i}$, $i \in [m]$ the ith row of $\matr{B}$. To get a proof of size independent of $n$ we commit to $\matr{B}$ ``compressing the rows'', that is, the proof includes MP commitments $[\vecb{d}_i]_1$, $i \in [n]$ to $\vecb{b}_i^*$.\footnote{To get a constant-size proof, it would be tempting to compress the commitments to all of $\matr{B}$, but we do not know how to prove soundness in this case.} Further, as announced in Section~\ref{sec:techniques}, the proof consists of two independent statements:
\begin{itemize}
\item $\exists \vecb{r} \in \Z_q^m, \matr{B} \in \Z_q^{m \times n}$, (whose rows are denoted as $\vecb{b}_i^*$, $i \in [m]$), such that  
$1'') \matr{B} \in \{0,1\}^{m \times n}$ and $3'') \forall i \in [m]: \vecb{d}_i=\matr{G}\begin{pmatrix}
\vecb{b}_i^*  \\ r_i  
\end{pmatrix}$,
\item $\exists \vecb{r} \in \Z_q^m, \{\vecb{w}_j \in\Z_q^\lc: j \in [n]\}, \widetilde{\matr{B}} \in \Z_q^{m \times n}$, (whose rows are denoted as $\widetilde{\vecb{b}}_i^*$, $i \in [m]$ and the columns $\widetilde{\vecb{b}_j}$, $j \in [n]$), such that  
   $2'') \pmatri
{
    \vecb{c}_j\\
    \grkb{\alpha}
}
=
\begin{pmatrix}
    \matr{M}       & \matr{N}\\
    \matr{\Lambda} & \matr{0}
\end{pmatrix}
\pmatri
{
    \widetilde{\vecb{b}}_j\\
    \vecb{w}_j
}$ and $3'') \forall i \in [m]: \vecb{d}_i=\begin{pmatrix} \vecb{u}_1 &   \vecb{u}_2 \end{pmatrix}   \begin{pmatrix}
\widetilde{\vecb{b}}^*_i  \\ r_i  
\end{pmatrix}$.
\end{itemize} 
For the first statement we use the constant-size Argument for $\Lang_{ck,\sfbits}^m$ of Section~\ref{sec:bits}. For the second statement, we write conditions 2''), 3'') as a single system of equations and use $\Pi_\sflin$ to prove that it can satisfied (which is equivalent to proving that the solution is in the span of the coefficient's matrix).  

The soundness argument follows from the arguments exposed in Section~\ref{sec:techniques}. The full description of the argument together with the proof of the following theorem are in Appendix~\ref{app:bin-lang}.

\begin{theorem} \label{theo:aggset} There exists a QA-NIZK argument $\Pi_\sfset$ for membership in the language $\Lang_{[\matr{M}]_1,[\matr{N}]_1,\matr{\Lambda},\grkb{\alpha}}^n$ with proof size  $(2\lb+11)|\GG_1|+10|\GG_2|$, perfect completeness, perfect-zero knowledge and computational soundness. 
\end{theorem}

