In this section we prove Theorem~\ref{theo:mp}. 

\begin{proof}
{\bf Computationally binding:} (This follows a proof due to Villar). Let $[a]_\gamma\in\GG_{\gamma}$ be the discrete logarithm challenge. To sample the commitment key according to $\distink$, choose $\matr{G}_{2} \gets \dist_k$, and define the last $k$ columns of $[\matr{G}]_\gamma$ as $[\matr{G}_{2}]_{\gamma}$. For the rest of the columns of $[\matr{G}]_\gamma$, independently  for each $j \in [n]$, $i \neq j$, sample a pair $\vecb{\alpha}_j,\vecb{\beta}_j$ and define $[\vecb{g}_{j}]_{\gamma}=[\matr{G}_{2} (a \vecb{\alpha}_j+\beta_j)]_\gamma$,
which can be computed as $[a]_\gamma \matr{G}_{2}\vecb{\alpha}_j+[ \matr{G}_{2} \beta_j]_\gamma$. If $i \neq 0$, set $\vecb{g}_{i} \gets \Z_q^{k+1}$. In this case, with overwhelming probability,  $\vecb{g}_{i}$ is linearly independent of the rest of the columns and we will assume so in the following. 
The commitment key is then given to the adversary against the binding property of the scheme, and it outputs a commitment $[\vecb{c}]_{\gamma}$, together with two valid openings 
$(\vecb{m},\vecb{r}), (\vecb{m}',\vecb{r}')$ such that $\vecb{m}\neq \vecb{m}'$. It follows that $[\vecb{c}]_{\gamma}=[\matr{G}]_{\gamma} \smallpmatrix{\vecb{m} \\ \vecb{r}} = \bmatr{G}_{\gamma}\smallpmatrix{\vecb{m}' \\ \vecb{r}'}$, which implies that $[\vecb{0}]_{\gamma}=[\matr{G}]_{\gamma} \smallpmatrix{\vecb{m}-\vecb{m}' \\ \vecb{r}-\vecb{r}'}$. Further, because $\vecb{g}_i$ is linearly independent of the rest of the columns, it holds that:
\begin{equation}\label{recovera}
a \left(\matr{G}_{2} (\sum_{j \neq i} (m'_j-m_j)  \boldsymbol{\alpha}_j) \right) = \left(  \matr{G}_{2}  (\vecb{r}-\vecb{r}' + \sum_{j \neq i} \boldsymbol{\beta}_j  (m_j-m_j'))\right).
\end{equation}
W.l.o.g we can assume that $\matr{G}_{2}$ has full rank (it can be shown that if $\dist_k$-$\mddh$ is a generically hard assumptions in $k$-linear groups, then matrices sampled from $\dist_k$ have full rank with overwhelming probability).  Then, we can recover $a \in \Z_q$ from equation \ref{recovera} except if $\sum_{j \neq i} (m'_j-m_j)  \boldsymbol{\alpha}_j = \vecb{0}$. But since, for all $j$, $\boldsymbol{\alpha}_j$ is information theoretically hidden from the adversary, the probability of this event is at most 
$1/q^k$. 

{\bf Perfectly binding at coordinate $i$:} With overwhelming probability, $\vecb{g}_i$ is linearly independent of the rest of the columns of $\matr{G}$. Therefore, given any $[\vecb{c}]_{\gamma} \in \GG_{\gamma}^{k+1}$, if $\vecb{m}\in\Z_q^n, \vecb{r} \in\Z_q^k$ are such that $\vecb{c}=\matr{G}\smallpmatrix{\vecb{m}\\ \vecb{r}}$, there exists a unique $\tilde{m}_i \in \Z_q$ such 
that $m_i=\tilde{m}_i$. 

{\bf Perfectly hiding at coordinate $j$, $j \neq i$:} This follows immediately from the fact that $\vecb{g}_j$ is in the image 
of $\matr{G}_{2}$. 
\end{proof}


