In this section we describe the languages for which there exist constant-size QA-NIZK arguments of membership which will be used as building blocks in our constructions.  These languages are (i) linear subspaces of $\GG_s^m$, $s \in \{1,2\}$ \cite{C:JutRoy14,EC:KilWee15,EC:LPJY14}, (ii) linear subspaces of $\GG_1^m\times\GG_2^n$ \cite{AC:GonHevRaf15}, (iii) equal commitment opening \cite{AC:GonHevRaf15}, and (iv) \emph{sum in subspace} \cite{AC:GonHevRaf15}. The proof systems follow the syntax from Sect. \ref{sec:zkdef} and the languages are defined as follows, for $\gamma,\nu \in \{1,2\}$, 
\begin{align}
&\Lang_{[\matr{M}]_\gamma}:=\{ [\vecb{x}]_{\gamma} \in \GG_{\gamma}^{n}:  \exists \vecb{w} \in \Z_q^{t}, \  \vecb{x}=\matr{M}\vecb{w} \}, \tag{i}\\
&\Lang_{[\matr{M}]_1,[\matr{N}]_2}:=\{ ([\vecb{x}]_1,[\vecb{y}]_2) \in \GG_1^{m}\times\GG_2^n:  \exists \vecb{w} \in \Z_q^{t}, \  \vecb{x}=\matr{M}\vecb{w}, \   \vecb{y}=\matr{N}\vecb{w} \}, \tag{ii}\\
&\Lang_{ck,ck',\sfcom}:=\left\{([\vecb{c}]_\gamma,[\vecb{d}]_\nu):\exists (\vecb{w}, \vecb{r},\vecb{s}) 
\in \Z_q^n \times \Z_q^{t_1} \times \Z_q^{t_2},  
\begin{array}{l}
    {[\vecb{c}]}_\gamma=\Com_{ck}(\vecb{w};\vecb{r}),\\
    {[\vecb{d}]}_\nu=\Com_{ck'}(\vecb{w};\vecb{s})
\end{array}
\right\}, \tag{iii}\\
&\Lang_{[\matr{M}]_1,[\matr{N}]_2,\sfsum}:=\{([\vecb{x}]_1,[\vecb{y}]_2):\exists\vecb{w}, \vecb{x}+\vecb{y}=(\matr{M}+\matr{N})\vecb{w}\}. \tag{iv}
\end{align}
In the above definitions, $\matr{M} \in \Z_q^{m \times t_1}$,
  $\matr{N} \in \Z_q^{n \times t_2}$ and $ck$ (resp. $ck$') define some commitments of vectors of $\Z_q^{n}$ where the randomness space is $\Z_q^{t_1}$ (resp. $\Z_q^{t_2}$). In (iv), $t_1=t_2$. The commitment scheme $\Com$ is assumed to be of the form 
 $\Com_{ck}(\vecb{w};\vecb{r})=[\matr{A}]_\gamma \vecb{w}+[\matr{B}]_\gamma \vecb{r}$, 
 for some matrices $[\matr{A}]_\gamma,[\matr{B}]_\gamma$ defined in 
  $ck$. 

%We emphasize that when using the proof systems  for the above languages as a building block to prove other statements,
We denote indistinctly by $\Pi_\sflin$ the proof systems for (i) and (ii), by $\Pi_\sfcom$ the proof system for (iii), and by $\Pi_\sfsum$ the proof system for (iv).

 To compute the proof sizes of our constructions, we will use the most efficient instantiations for each of these languages, which are described in Table \ref{table:lin-sp}. We note that the argument of \cite{AC:GonHevRaf15} for $\Lang_{ck,ck',\sfcom}$ is for the case $i=1,j=2$. It is not hard to see that when $i=j$, membership in $\Lang_{ck,ck',\sfcom}$ (for commitments of the form we specified) amounts to prove membership in some linear space in $\GG_\gamma$, which explains the second row of the table.  


\begin{table}[h]
\begin{center}
\begin{minipage}{\textwidth}
\begin{center}
\begin{tabular}{|l|lll|lll|}
\hline
Proof system                                & Language                            & Proof Size & Assumption \\
\hline\hline
\cite{EC:KilWee15}, \cite{C:JutRoy14}  & $\Lang_{[\matr{M}]_s}$ & $|\GG_{\gamma}|$          & $\SP_{\GG_{3-\gamma}}$ \\
  & $\Lang_{ck,ck',\com}, \gamma =\nu$ & $|\GG_{\gamma}|$          & $\SP_{\GG_{3-\gamma}}$ \\
\hline
                                            & $\Lang_{[\matr{M}]_1,[\matr{N}]_2}$      & (2,2)      & $\SSDP$ \\
\cite{AC:GonHevRaf15}                       & $\Lang_{ck,ck',\com}, \gamma \neq \nu$                    & (2,2)      & $\SSDP$ \\
                                            & $\Lang_{[\matr{M}]_1,[\matr{N}]_2,\sfsum}$ & (2,2)      & $\SSDP$\\
\hline
\end{tabular}
\end{center}
\caption{QA-NIZK arguments for linear subspaces used in this work. When the proof size is given by $(a,b)$ it means $a$ elements of $\GG_1$ and $b$ elements of $\GG_2$, otherwise $|\GG_{\gamma}|$ means one element of $\GG_{\gamma}$.  \label{table:lin-sp}} 
\end{minipage}
\vspace{-0.54cm}

\end{center}
\end{table}





%The proof system for (i) is instantiated under the $\SP$ assumption and the proof consists of one element of $\GG_1$. Proof systems for (ii), (iii), and (iv), are instantiated under the $\SSDP$ assumption and the proof consists of 2 elements of each $\GG_1$ and $\GG_2$.
%
%While (iii) is written for any commitment scheme, the proof system can be build only if $\Com_{ck}(\vecb{w};r)=[\matr{M}]_i\vecb{w}+[\matr{N}]_i\vecb{r}$, for some $[\matr{M}]_i\in\GG_i^{m\times n}$, $[\matr{N}]_i\in\GG_i^{m\times t}$, $i\in\{1,2\}$.

% Although the proof system for (iii) from \cite{AC:GonHevRaf15} only considers the case $i\neq j$, it is direct to extend the proof system to the case $i=j$ and in fact, when $i=j$, the proof consists of only one element of $\GG_i$ when security is proven under the $\SP$ assumption (and not the $\SSDP$ assumption). 
%
%In Appendix XXX we prove that this proof systems satisfy the stronger notion of soundness defined in Sect. \ref{sec:zkdef}. 
