\begin{definition} The Groth-Sahai commitment scheme in the SXDH instantiation in the group $\GG_\gamma$, $\gamma\in\{1,2\}$, is  specified by the following three algorithms 
	$(\mathsf{GS}.\algK,\mathsf{GS}.\Com,\mathsf{GS}.\algVrfy)$ such that:
	\begin{itemize} 
		\item  $\mathsf{GS}.\algK$ is a randomized algorithm, which on input the group key $gk$ and the (optional and if not given assumed to be $\mathsf{true}$) flag $\mathsf{binding}$, outputs a commitment key $ck:=[\matr{U}]_\gamma=[(\vecb{u}_1||\vecb{u}_2)]_\gamma \in\GG_\gamma^{2\times 2}$. It samples $\vecb{u}_2\gets\distlin_{1}$ and $\mu\gets\Z_q$, and, 
if $\mathsf{binding}=\mathsf{true}$, $[\matr{U}]_\gamma$ is the {perfectly binding key} and $\vecb{u}_1:=\mu\vecb{u}_2$ and else it is the {perfectly hiding key} and $\vecb{u}_1:=\mu\vecb{u}_2-\vecb{e}_2$.
		\item $\mathsf{GS}.\Com$ is a randomized algorithm which, on input a commitment key $ck=\bmatr{U}_\gamma$, and a message 
		$\mathsf{m}$ in the message space $\mathcal{M}_{ck}=\Z_q\cup\GG_\gamma$, it proceeds as follows. If $\mathsf{m}=m\in\Z_q$, it samples $r \gets \Z_q$ and outputs a commitment $\bvecb{c}_\gamma := m[\vecb{e}_2+\vecb{u}_1]_\gamma+r[\vecb{u}_2]_\gamma$ in the commitment space $\mathcal{C}_{ck}=\GG_\gamma^2$ and an opening $Op=r$. If $\mathsf{m}=[m]_\gamma\in\GG_\gamma$, it samples $\vecb{r} \gets \Z_q^2$ and outputs a commitment $\bvecb{c}_\gamma := [m]_\gamma\vecb{e}_2+[\matr{U}]_\gamma\vecb{r}$ in the commitment space $\mathcal{C}_{ck}=\GG_\gamma^2$ and an opening $Op=\vecb{r}$.
		\item $\mathsf{GS}.\algVrfy$ is a deterministic algorithm which, on input the commitment key $ck=\bmatr{U}_\gamma$, a commitment $\bvecb{c}_\gamma$,  a message 
		$m \in \mathcal{M}_{ck}$ and an opening $Op$, outputs $1$ if $\bvecb{c}_\gamma=\GS.\Com_{ck}(m;Op)$
		and $0$ otherwise.
	\end{itemize}
\end{definition}

\begin{theorem}[\cite{SIAM:GroSah12}] If $ck\gets\algK(gk)$ (resp. $ck\gets\algK(gk,\mathsf{false})$) the Groth-Sahai commitment scheme is perfectly binding (resp. computationally binding) and computationally hiding (resp. perfectly hiding).
\end{theorem}

