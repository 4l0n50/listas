Let $\ggen_a$ be some probabilistic polynomial time algorithm which on input $1^{\lambda}$, where $\lambda$ is the security parameter, returns the \emph{group key} which is the description of an asymmetric bilinear group $gk:=(q,\GG_1,\GG_2,\GG_T,e,\mathcal{P}_1,\mathcal{P}_2)$, where $\GG_1,\GG_2$
and $\GG_T$ are groups of prime order $q$, the elements $\mathcal{P}_1, \mathcal{P}_2$ are generators of 
$\GG_1,\GG_2$ respectively, and $e:\GG_1\times\GG_2\to\GG_T$ is an efficiently
computable, non-degenerate bilinear map. 

Elements in $\GG_s$, are denoted implicitly as $[a]_s:=a \Pt_s$, where $s \in \{1,2,T\}$ and $\Pt_T:=e(\Pt_1,\Pt_2)$. 
The pairing operation will be written as a product $\cdot$, that is $[a]_1 \cdot [b]_2=[a]_1 [b]_2=e([a]_1,[b]_2)=[ab]_T$. Vectors and matrices are denoted in boldface. Given a matrix $\matr{T}=(t_{i,j})$, $[\matr{T}]_s$ is
the natural embedding of $\matr{T}$ in $\GG_s$, that is, the matrix whose $(i,j)$th entry
is $t_{i,j}\mathcal{P}_s$. We denote by $|\GG_s|$ the bit-size of the elements of $\GG_s$.

$\matr{I}_{n\times n}$ refers to the identity matrix in $\Z_q^{n\times n}$, $\matr{0}_{m\times n}$ and $\matr{1}_{m\times n}$ the all-zero and all-one matrices in $\Z_q^{m\times n}$, respectively, and $\vecb{e}^{n}_i$ the $i$th element 
of the canonical basis of $\Z_q^{n}$ (simply $\matr{I}$, $\matr{0}$, $\matr{1}$, and $\vecb{e}_i$, respectively, if $m$ and $n$ are clear from the context). 
Given some matrices $\matr{A}\in\Z_q^{m\times t},\matr{A}_1\in\Z_q^{m_1\times t},\ldots,\matr{A}_n\in\Z_q^{m_n\times n}$, we define the operations
\iffull
 $$\vecb{A}_1 \oplus \ldots \oplus \vecb{A}_n:=\smallpmatrix{ \vecb{A}_1 \\ \vdots \\  \vecb{A}_n} \qquad 
\matr{A}^n:=\smallpmatrix{ \matr{A} &  & \matr{0} \\   & \ddots &   \\ \matr{0} &  & \matr{A}
}.$$
\else
$\vecb{A}_1 \oplus \ldots \oplus \vecb{A}_n:=\smallpmatrix{ \vecb{A}_1 \\ \vdots \\  \vecb{A}_n}$ and $\matr{A}^n:=\smallpmatrix{ \matr{A} &  & \matr{0} \\   & \ddots &   \\ \matr{0} &  & \matr{A}}.$
\fi
%
%We make extensive use of the set $[m+k]\times[m+k]\setminus\{(i,i):i\in[n]\}$ and for brevity we denote it by $\indexSet{m}{k}$.
%
%We write $\uvecb{T}\in\Z_q^{mn}$ for the \emph{vectorization} of $\matr{T}\in\Z_q^{m\times n}$, that is
%$
%\uvecb{T}:= (t_{1,1},\ldots,t_{m,1},$ $t_{1,2},\ldots,t_{m,2},\ldots,t_{1,n},\ldots,t_{m,n})^\top
%$. We will use the following fact about vectorizations
%\begin{fact}
%For any $\matr{A}\in\Z_q^{\ell\times n},\matr{B}\in\Z_q^{\ell\times m},\matr{C}\in\Z_q^{m\times n}$
%$$
%\matr{A}=\matr{B}\matr{C}
%\Longleftrightarrow
%\uvecb{A} = (\matr{I}_{n\times n}\otimes\matr{B})\uvecb{C} = 
%\begin{pmatrix}
%\matr{B}                & \matr{0}_{\ell\times m} & \cdots & \matr{0}_{\ell\times m}\\
%\matr{0}_{\ell\times m} & \matr{B}                & \cdots & \matr{0}_{\ell\times m}\\
%\vdots                  & \vdots                  & \ddots & \vdots                 \\
%\matr{0}_{\ell\times m} & \matr{0}_{\ell\times m} & \cdots & \matr{B}
%\end{pmatrix}
%\uvecb{C},
%$$
%where $\otimes$ denotes the Kronecker product.
%\end{fact}
%
%
%
