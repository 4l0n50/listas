
\begin{definition}   \label{def:matrixdef}
Let $\ell,k \in \N$.
We call $\dist_{\ell,k}$ a matrix distribution if it outputs (in poly time, with overwhelming probability) matrices in $\Z_q^{\ell \times k}$. We define $\dist_k := \dist_{k+1,k}$. 
\end{definition}

For the following decisional assumption to hold, it is a necessary condition that $\ell>k$. However, in other contexts, we might need $\dist_{\ell,k}$ distributions where 
$\ell \geq k$. 

\begin{definition}[Matrix Diffie-Hellman Assumption in $\GG_{\gamma}$, $\gamma \in \{1,2\}$ \cite{C:EHKRV13}]\label{def:mdh}
Let $\dist_{\ell,k}$ be a matrix distribution  and $\gk\gets \ggen_a(1^\lambda)$. We say that the $\dist_{\ell,k}$-Matrix Diffie-Hellman ($\dist_{\ell,k}$-$\mddh_{\GG_\gamma}$)
Assumption holds relative to $\ggen_a$ if for all PPT adversaries $\advD$,
\begin{eqnarray*}
\adv_{\dist_{\ell,k},\ggen_a}(\advD) & := &
    \left|
        \Pr[\advD(\group,[\matr{A}]_\gamma,[\matr{A}\vecb{w}]_\gamma)=1]-
        \Pr[\advD(\group,[\matr{A}]_\gamma, [\vecb{z}]_\gamma) =1]
    \right|
\end{eqnarray*}
is negligible in $k$,
where the probability is taken over $\gk \gets \ggen_a(1^\lambda)$, $\matr{A} \gets \dist_{\ell,k}, \vecb{w} \gets \Z_q^k, [\vecb{z}]_\gamma  \gets \GG_\gamma^{\ell}$ and the coin tosses of adversary $\advD$.
\end{definition}
 

In this paper we will refer to the following matrix distributions: 
\[
\distlin_{k}:\matrA = \left( \begin{smallmatrix}
    a_1 & 0 &  \ldots & 0 \\
    0 &  a_2 &  \ldots & 0\\
    \tiny{\vdots} &  \tiny{\vdots}  &  \tiny{\ddots} & \tiny{\vdots} \\
    0 & 0 &  \ldots  & a_{k}\\
    1 & 1 & \ldots & 1
\end{smallmatrix} \right),
\ 
\mathcal{U}_{\ell,k}: \matrA = \left( \begin{smallmatrix}
    a_{1,1} &  \ldots & a_{1,k}  \\
    \tiny{\vdots} & \tiny{\ddots}  & \tiny{\vdots} \\
    a_{\ell,1} &  \ldots &  a_{\ell,k} 
\end{smallmatrix} \right),
\]
where $a_i,a_{i,j}\leftarrow \Z_q$, for each $i,j\in[k]$.  The $\distlin_{k}$-$\mddh$ Assumption is the $k$-linear family of Decisional Assumptions
\cite{C:HofKil07,cryptoeprint:2007:074} and corresponds to 
 the Decisional Diffie-Hellman (DDH)
Assumption in $\GG_\gamma$ when $k=1$. The SXDH Assumption states that DDH holds in $\GG_\gamma$ for all $\gamma \in \{1,2\}$. The $\mathcal{U}_{\ell,k}$ Assumption is the \textit{Uniform} Assumption and is the weakest of all assumptions of size $\ell \times k$. 

Further, given any matrix distribution $\dist_{k}$, $m \in \mathbb{N}$ and any $i \in [m]$, we will repeatedly make reference to the distribution $\distink$, which is defined as follows: 
\[ \distzeronk: \matrA = \left(\begin{smallmatrix} \matr{B}\vecb{w}_1 & \ldots &  
  \matr{B}\vecb{w}_{m} & \matr{B}  \end{smallmatrix} \right)  \qquad
\distink:\matrA = \left(\begin{smallmatrix} \matr{B}\vecb{w}_1 & \ldots & \matr{B}\vecb{w}_{i-1} 
& \vecb{z} &  \matr{B}\vecb{w}_{i+1} & \ldots &  
  \matr{B}\vecb{w}_{m} & \matr{B}  \end{smallmatrix} \right) 
\]
where $\matr{B} \leftarrow \dist_{k}$, $\vecb{w}_i \leftarrow \Z_q^k$ and $\vecb{z} \leftarrow \GG^{k+1}$. The following are two trivial properties of the $\distink$ distribution. 

\begin{lemma} Under the $\dist_{k}$-$\mddh$ Assumption in $\GG_\gamma$ , for any $0 < i \leq n$, the distribution of  $[\matrA]_\gamma$ when $\matrA \leftarrow \distzeronk$ and when $\matrA \leftarrow \distink$ are computationally indistinguishable. Further, if $\ell>k$, for any $i>0$, if $\matr{A} \leftarrow \distink$, then with overwhelming probability its ith column is linearly independent of the rest. \label{lemma:dist-i}
\end{lemma}
