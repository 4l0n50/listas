The GS proof system allows to prove satisfiability of a set of quadratic equations in a bilinear group. The admissible equation types must be in the following form:
\begin{equation}\label{gseq}
\sum_{j=1}^{m_y} f(\alpha_j, \vary_j)+\sum_{i=1}^{m_x} f(\varx_i, \beta_i)+\sum_{i=1}^{m_x} \sum_{j=1}^{m_y}  f(\varx_i,\escQE_{i,j} \vary_j)=t,
\end{equation}
 where $\boldsymbol \alpha  \in \Am_1^{m_y}$, $\boldsymbol \beta  \in \Am_2^{m_x}$, $\matr{\EscQE}=(\escQE_{i,j}) \in \Z_q^{m_x\times m_y}$, $t \in \Am_T$, and $\Am_1,\Am_2,\Am_T$ are $\Z_q$-vector spaces equipped with some bilinear map $f:\Am_1\times \Am_2 \rightarrow \Am_T$. When $t=f(t_1,1)$ or $t=f(1,t_2)$, for some $t_1\in\Am_1$ or $t_2\in\Am_2$, we say that the equation allows simulation (see \cite{SIAM:GroSah12}). The modules and the map $f$ can be defined in different ways as: (a) in pairing-product equations (PPEs), $\Am_1=\GG_1$, $\Am_2=\GG_2$, $\Am_T=\GG_T$, $f([x]_1,[y]_2)=e([x]_1,[y]_2) \in \GG_T$, in which case $t=0_{T}$, (b1) in multi-scalar multiplication equations in $\GG_1$ (MMEs), $\Am_1=\GG_1$, $\Am_2=\Z_q$, $\Am_T=\GG_1$, $f([x]_1,y)=y [x]_1 \in \GG_1$,  (b2) MMEs in $\GG_2$ (MMEs),  $\Am_1=\Z_q$, $\Am_2=\GG_2$, $\Am_T=\GG_2$, $f(x,[y]_2)=x [y]_2 \in \GG_2$, and (c) in quadratic equations in $\Z_q$ (QEs), $\Am_1=\Am_2=\Am_T=\Z_q$, $f(x,y)=xy \in \Z_q$. 
 %An equation is linear if $\matr{\EscQE}=\vecb{0}$, 
 %it is \textit{two-sided linear} if both $\boldsymbol \alpha \neq \vecb{0}$ and $\boldsymbol \beta \neq \vecb{0}$, and \textit{one-sided} otherwise.  

The GS proof system consists of five algorithms $\GS:=(\algK,\algP,\algV,\algS_1,\algS_2)$. Below we describe the properties which are relevant to this work.
\begin{description}
\item[$\algK$:] The CRS generator, on input a group key $\gk$ outputs the \emph{perfectly binding CRS}.  In the SXDH instantiation the CRS is $\crs_\GS:=(\gk,ck_1,ck_2)$, 
$ck_1=([\vecb{u}_1]_1,[\vecb{u}_2]_1)$,
 $ck_2=([\vecb{v}_1]_2,[\vecb{v}_2]_2)$, 
 where $\vecb{u}_2,\vecb{v}_2 \gets \distlin_1$, $\vecb{u}_1:=\mu \vecb{u}_2 $, $\vecb{v}_1:=\epsilon \vecb{v}_2$, $\mu,\epsilon\gets\Z_q$.
\item[$\algP$:] The prover, on input an equation $\eq$ of the form (\ref{gseq}) and solution $\varx_i,\vary_j$, $i\in[m_x],j\in[m_y]$, outputs a proof $\pi=(\{\GS.\Com_{ck_1}(\varx_i):i\in[m_x]\},\{\GS.\Com_{ck_2}(\vary_j):j\in[m_y]\},\pi')$.
\item[$\algV$:] The verifier, on input an equation $\eq$ of the form (\ref{gseq}) and a proof $\pi$, accepts or rejects the proof for $\eq$.
\item[$\algS_1$:]  The CRS simulator, on input a group description $\gk$ outputs the \emph{perfectly hiding CRS}  and a trapdoor. In the SXDH instantiation, the CRS is identical to the one output by $\algK$, except that $\vecb{u}_1:=\mu \vecb{u}_2-\vecb{e}_1^\top$ and $\vecb{v}_1:=\epsilon \vecb{v}_2-\vecb{e}_1^\top$, and the simulation trapdoor is $(\mu,\epsilon)$.
\item[$\algS_2$:] The prover simulator, on input an equation $\eq$ of the form (\ref{gseq}) which allows simulation, and a trapdoor $\tau$, outputs a simulated proof $\pi$.
\end{description}
