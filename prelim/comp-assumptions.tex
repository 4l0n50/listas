Additionally, we will be using the following family  computational assumptions:
\begin{definition}[Kernel Diffie-Hellman Assumption in $\GG_{\gamma}$ \cite{EPRINT:MorRafVil15}]Let  $\gk 
\hspace*{-1pt}
\gets
\hspace*{-1pt}
\ggen_a(1^\lambda)$.
The Kernel Diffie-Hellman Assumption in $\GG_\gamma$  ($\dist_{\ell,k}\mbox{-}\kermdh_{\GG_\gamma}$) says that every PPT Algorithm has negligible advantage in the following  game: given $[\matr{A}]_\gamma$, where $\matrA \gets \dist_{\ell,k}$, find $[\vecb{x}]_{3-\gamma} \in \GG_{3-\gamma}^{\ell}$, $\vecb{x} \neq \vecb{0}$, such that 
$[\vecb{x}]_{3-\gamma}^{\top}[\matr{A}]_{\gamma}=[\vecb{0}]_T$. 
\end{definition}

The  Simultaneous Pairing Assumption in $\GG_\gamma$  (\SP$_{\GG_{\gamma}}$) is the $ \mathcal{U}_1\mbox{-}\kermdh_{\GG_{\gamma}}$ Assumption. The Kernel Diffie-Hellman assumption is a generalization and abstraction of this assumption to other matrix distributions. 
The $\dist_{\ell,k}\mbox{-}\kermdh_{\GG_{\gamma}}$ Assumption is weaker than the $\dist_{\ell,k}\mbox{-}\mddh_{\GG_{\gamma}}$ Assumption, since a solution allows to decide membership in $\rmIm([\matr{A}]_{\gamma})$.

In asymmetric bilinear groups, there is a natural variant of this assumption which was introduced in \cite{AC:GonHevRaf15}.  
\begin{definition}[Split Kernel Diffie-Hellman Assumption]
Let  $\gk \hspace*{-3pt} \gets
\hspace*{-3pt}
\ggen_a(1^\lambda)$.
The Split Kernel Diffie-Hellman Assumption in $\GG_1,\GG_2$  ($\dist_{\ell,k}\mbox{-}\skermdh$) says that every PPT Algorithm has negligible advantage in the following  game: given $([\matr{A}]_1,[\matr{A}]_2)$, $\matr{A} \leftarrow \dist_{\ell,k}$, find a pair of vectors $([\vecb{r}]_1,[\vecb{s}]_2) \in \GG_1^{\ell} \times \GG_2^{\ell}$, $\vecb{r} \neq \vecb{s}$, such that 
$[\vecb{r}]_1^{\top}[\matr{A}]_2=[\vecb{s}]_2^{\top}[\matr{A}]_1$. 
\end{definition}

While the Kernel Diffie-Hellman Assumption says one cannot find a non-zero vector in one of the groups which is in the co-kernel of $\matr{A}$, the split assumption says one cannot find a pair of vectors in $\GG_1^{\ell} \times \GG_2^{\ell}$ such that the difference of the vector of their discrete logarithms is in the co-kernel of $\matr{A}$. 
As a particular case we consider the \emph{Split Simultaneous Double Pairing Assumption in} $\GG_1,\GG_2$ ($\SSDP$) which is the $\distrlin_{2}\mbox{-}\skermdh$ Assumption, where 
$\distrlin_{2}$ is the distribution which results of sampling a matrix from $\distlin_{2}$ and replacing the last row by random elements. 

