We recall the definition of Quasi Adaptive NIZK (QA-NIZK) Arguments of Jutla et al.\ \cite{AC:JutRoy13}. A QA-NIZK proof system enables
to prove membership in a language defined by a relation $\R_\rho$, which in turn is completely determined by some parameter
$\rho$ sampled from a distribution $\dist_\gk$.
We say that $\dist_\gk$ is \emph{witness samplable} if there exists an efficient
algorithm that samples $(\rho,\omega)$ from a distribution $\dist_\gk^{\mathsf{par}}$ such that $\rho$ is distributed according to $\dist_\gk$, and membership of $\rho$
in the \emph{parameter language} $\Lang_\mathsf{par}$ can be efficiently verified with $\omega$.
While the Common Reference String can be set based on $\rho$, the zero-knowledge simulator is required to be a single probabilistic polynomial time
algorithm that works for the whole collection of relations $\R_\gk$. 

The details of the QA-NIZK definition can be found in Appendix \ref{app:qanizk}.
