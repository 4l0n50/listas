We recall the definition of Quasi Adaptive NIZK (QA-NIZK) Arguments of Jutla et al.\ \cite{AC:JutRoy13}. A QA-NIZK proof system enables
to prove membership in a language defined by a relation $\R_\rho$, which in turn is completely determined by some parameter
$\rho$ sampled from a distribution $\dist_\gk$.
We say that $\dist_\gk$ is \emph{witness samplable} if there exists an efficient
algorithm that samples $(\rho,\omega)$ from a distribution $\dist_\gk^{\mathsf{par}}$ such that $\rho$ is distributed according to $\dist_\gk$, and membership of $\rho$
in the \emph{parameter language} $\Lang_\mathsf{par}$ can be efficiently verified with $\omega$.
While the Common Reference String can be set based on $\rho$, the zero-knowledge simulator is required to be a single probabilistic polynomial time
algorithm that works for the whole collection of relations $\R_\gk$. 

A tuple of algorithms $(\algK_0,\algK_1,\algP,\algV)$ is called a QA-NIZK proof system for witness-relations
$\R_\gk = \{\R_\rho\}_{\rho \in \mathrm{sup}(\dist_\gk)}$
with parameters sampled from a distribution $\dist_\gk$ over associated parameter language
$\Lang_\mathsf{par}$, if there exists a probabilistic polynomial time simulator $(\algS_1, \algS_2)$,
such that for all non-uniform PPT adversaries $\advA_1$, $\advA_2$, $\advA_3$ we have:

\begin{description}
\item[Quasi-Adaptive Completeness:]
$$\Pr\begin{bmatrix*}[l]
    \gk \gets \algK_0(1^\lambda);
    \rho \gets \dist_\gk;
    \psi \gets \algK_1(\gk, \rho);
    (x, w) \gets \advA_1(\gk, \psi);\\
    \pi \gets \algP(\psi, x, w) :
        \algV(\psi, x, \pi) = 1 \text{ if } \R_\rho(x, w)
\end{bmatrix*} = 1.$$
\item[Computational Quasi-Adaptive Soundness:]
$$\Pr\begin{bmatrix*}[l]
    \gk \gets \algK_0(1^\lambda);
    \rho \gets \dist_\gk;
    \psi \gets \algK_1(\gk, \rho); \\
    (x, \pi) \gets \advA_2(\gk, \psi) :
        \algV(\psi, x, \pi) = 1 \text{ and } \neg (\exists w : \R_\rho(x, w))
\end{bmatrix*} \approx 0.$$ 
\item[Perfect Quasi-Adaptive Zero-Knowledge:]
\begin{eqnarray*}
\Pr[
    \gk \gets \algK_0(1^\lambda);
    \rho \gets \dist_\gk;
    \psi \gets \algK_1(\gk, \rho):
        \advA_3^{\algP(\psi,\cdot, \cdot)}(\gk,\psi) = 1]
=\;\;\;\\
\Pr[
    \gk \gets \algK_0(1^\lambda);
    \rho \gets \dist_\gk;
    (\psi,\tau) \gets \algS_1(\gk, \rho):
        \advA_3^{\algS(\psi,\tau,\cdot,\cdot)}(\gk,\psi)=1]
\end{eqnarray*}
where
\begin{itemize}
\item $\algP(\psi, \cdot, \cdot)$ emulates the actual prover. It takes input $(x,w)$ and outputs a 
proof $\pi$ if $(x,w)\in\R_\rho$. Otherwise, it outputs $\perp$.
\item $\algS(\psi,\tau, \cdot, \cdot)$ is an oracle that takes input $(x,w)$. It outputs a simulated proof
$\algS_2(\psi,\tau, x)$ if $(x, w) \in \R_\rho$ and $\perp$ if $(x, w) \notin \R_\rho$.
\end{itemize}
\end{description}
Note that $\psi$ is the CRS in the above definitions.
We assume that $\psi$ contains an encoding of $\rho$, which is thus available to $\algV$.

For witness samplable distributions, a stronger notion of soundness, where the adversary has also access to a witness of the parameter $\rho$, is defined in \cite{EPRINT:GonHevRaf15}.
\begin{description}
\item[Computational Quasi-Adaptive Strong Soundness:]
$$\Pr\begin{bmatrix*}[l]
    \gk \gets \algK_0(1^\lambda);
    (\rho,\omega) \gets \dist_\gk^\mathsf{par};
    \psi \gets \algK_1(\gk, \rho); \\
    (x, \pi) \gets \advA_2(\gk, \omega,\psi) :
        \algV(\psi, x, \pi) = 1 \text{ and } \neg (\exists w : \R_\rho(x, w))
\end{bmatrix*} \approx 0.$$ 

\end{description}
