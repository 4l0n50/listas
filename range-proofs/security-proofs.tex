\subsubsection{Completeness.}
If $[\vecb{c}]_1=\GS.\Com(y;r)$ and $y\in[0,2^n-1]$, then there exists $y_1,\ldots,y_\ell\in[0,d-1]$  such that $y=\sum_{i\in[\ell]}y_id^{i-1}$. Therefore $\vecb{y}=(y_1,\ldots,y_\ell)^\top$ and $r$ are solutions to $\eq$ and $[\matr{X}]_1=\GS.\Com(\vecb{y}^\top)\in\Lang_{\vecb{u}'_1,\vecb{u}'_2,d}^\ell$.

\subsubsection{Soundness.}
Given an adversary $\advA$ against the soundness of the proof system from Fig. \ref{fig:rp}, we construct an adversary $\advB$ against the soundness of $\Pi_\sfbin$. If $y\notin[0,2^n-1]$, then the perfect soundness of GS proofs implies that there is some $y_i\notin[0,d-1]$. Therefore $[\matr{X}]_1=\GS.\Com(\vecb{y}^\top)\notin\Lang_{\vecb{u}'_1,\vecb{u}'_2,d}^\ell$ and $\adv_{\Pi_\mathsf{range-proof}}(\advA)\leq\adv_{\Pi_\sfbin}(\advB)$.

\subsubsection{Zero-Knowledge.} Follows directly from Zero-Knowledge of GS proofs, the fact that $\eq$ allows simulation, and the fact that $[\matr{X}]_1=\GS.\Com(\matr{0}_{1\times n})\in\Lang_{d,\vecb{u}'_1,\vecb{u}'_2}^\ell$.
