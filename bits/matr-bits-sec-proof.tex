\subsection{The Scheme}
\vspace*{-1cm}
%   If $\matr{G}\gets\dist_{2,m+1}^0$ the proof that $[\vecb{c}]_1$ and $[\ovecb{c}]_1$ open to the same value implies that $\ovecb{c}_j\in\Span(\matr{G})$ for all $j\in[n]$ and, by the perfect hiding property, $([\vecb{c}_1]_1,\ldots,[\vecb{c}_n]_1)$ can be opened to any $\matr{B}\in\bits^{m\times n}$ thus $([\vecb{c}_1]_1,\ldots,[\vecb{c}_n]_1)\in\Lang_{ck,\sfbits}^n$. If $\matr{G}\gets\dist_{2,m+1}^{i^*}$ and $([\vecb{c}_1]_1,\ldots,[\vecb{c}_n]_1)\notin\Lang_{ck,\sfbits}^n$, then the $i^*$ th row of $\matr{B}$ is not in $\bits^{1\times n}$ which implies that there is some $j\in[n]$ such that $[\vecb{c}_j]_1\notin\Lang_{ck,\sfbits}$. If we pick $j^*\gets[n]$ and $\overline{\matr{G}}\gets\dist_{2,mn+1}^{m(i^*-1)+j^*}$, soundness of the proof that $[\ovecb{c}]_1\in\Lang_{\overline{ck},\sfbits}$ is violated with probability at least $1/n$.
\begin{figure} 
\begin{\algSize}
$$
\begin{array}{ll}
\begin{array}{l}
\algK_1(\gk,[\matr{G}]_1,n)\quad (\mathsf{S}_1(\gk,[\matr{G}]_1,n))\\
\hline\\[-.45cm]
[\overline{\matr{G}}]_1 \gets \MP.\algK(1^\lambda,mn,\distlin_1^{mn,0})\\
\crs_\sfcom\gets\Pi_\sfcom.\algK_1(\gk,[\vecb{G}^n]_1,[\ovG]_1, mn)\\
\crs_\sfbits\gets\Pi_\sfbits.\algK_1(\gk,[\overline{\matr{G}}]_1)\\
\text{Return } \ \mathsf{crs}:=(\crs_\sfcom,\crs_\sfbits).\\
(\tau_\sfcom\gets\Pi_\sfcom.\mathsf{S}_1(\gk,[\matr{G}^n]_1,[\ovG]_1,mn)\\
\tau_\sfbits\gets\Pi_\sfbits.\mathsf{S}_1(\gk,[\overline{\matr{G}}]_1).\\
\tau := (\vecb{a},\tau_\sflin,\tau_\sfbits)).\\
\\
\end{array}
&
\begin{array}{l}
{\algP(\mathsf{crs}, ([\vecb{c}]_1,\ldots,[\vecb{c}_n]_1), \langle (\vecb{b}_1,\ldots,\vecb{b}_n), \vecb{w}\rangle)}\\
\hline\\[-.45cm]
{[\ovecb{c}]_1 :=\MP.\Com_{[\ovG]_1}(\vecb{b};\overline{w})},\overline{w}\gets\Z_q\\
{\pi_\sfcom \gets \Pi_{\sfcom}.\algP(\crs_\sfcom,[\vecb{c}]_1,[\ovecb{c}]_1,}{\langle\vecb{b},\vecb{w},\overline{w}\rangle)}\\
\pi_\sfbits \gets \Pi_\sfbits.\algP(\crs_\sfbits,[\ovc]_1,\langle \vecb{b},\overline{w}\rangle)\\
\text{Return } \  ([\ovc]_1,\pi_\sfcom,\pi_\sfbits). \\
\\
\\
\\
\\
\end{array}\\
\begin{array}{l}
{\algV(\mathsf{crs},([\vecb{c}_1]_1,\ldots,[\vecb{c}_n]_1),([\ovc]_1,\pi_\sfcom,\pi_\sfbits))}\\
\hline
\mathsf{ans}_1 \gets \Pi_\sfcom.\algV(\crs_\sfcom,[\vecb{c}]_1,[\ovc]_1,\pi_\sfcom)\\
\mathsf{ans}_2 \gets \Pi_\sfbits.\algV(\crs_\sfbits,[\ovc]_1,\pi_\sfbits)\\
\text{Return } \ \mathsf{ans}_1\wedge\mathsf{ans}_2.
\\
\\
\end{array}
&
\begin{array}{l}
{\mathsf{S}_2(\crs,([\vecb{c}_1]_1,\ldots,[\vecb{c}_n]_1),[\matr{D}]_1,\tau)}\\
\hline
{[\ovc]_1 \gets \MP.\Com_{[\ovG]_1}(\vecb{0}_{mn\times 1})}\\
\pi_\sfcom\gets \Pi_\sfcom.\algS_2(\crs_\sfcom,[\vecb{c}]_1,[\ovc]_1,\tau_\sfcom)\\
\pi_\sfbits \gets \Pi_\sfbits.\algS_2(\crs_\sfbits,[\ovc]_1,\tau_\sfbits)\\
\text{Return }  ([\ovc]_1,\pi_\sfcom, \pi_\sfbits).
\end{array}
\end{array}$$
\end{\algSize}
\caption{The proof system for the language $\Lang_{[\matr{G}]_1,\sfbits}^n$. $\Pi_\sfbits$ is the proof system from Sect. \ref{sec:bits-scheme}. The matrix $\matr{G}^n$ is defined as
$\matr{G}^n:=(\matr{G}_1\oplus\ldots\oplus\matr{G}_1||\vecb{g}_{n+1}\oplus\ldots\oplus\vecb{g}_{n+1})$, and $\vecb{c}:=\vecb{c}_1\oplus\ldots\oplus\vecb{c}_n$ and $\vecb{b}:=\vecb{b}_1\oplus\ldots\oplus\vecb{b}$.
\label{fig:bitsn}
}
\end{figure}
\vspace*{-1cm}
\subsection{Security Proof}

\begin{proof}
{\bf Completeness.}
Follows from the fact that $([\vecb{c}]_1,[\ovc]_1)\in\Lang_{\sfcom,[\matr{G}^n]_1,[\ovG]_1,mn}$ and that $[\ovc]_1\in\Lang_{[\ovG]_1,\sfbits}$.

{\bf Soundness.}
When $\matr{G}\gets\distlinizeroone$ the proof follows from the proof that $[\vecb{c}]_1$ and $[\ovecb{c}]_1$ open to the same value.
When $\matr{G}\gets\distlinisnone$, the proof follows from the indistinguishability of the following games.

\begin{description}
\item[$\mathsf{Real}$:] This is the real soundness game. The adversary wins if it outputs $([\vecb{c}_1]_1,\ldots,[\vecb{c}_n]_1)\notin\Lang_{ck,\sfbits}^n$ and the corresponding proof which is accepted by the verifier.
\item[$\sfGame_0$:] This game is exactly as $\mathsf{Real}$ except that $\algK_1$ does not receive $[\matr{G}]_1$ as a input but it samples $\matr{G}$ itself according to $\distlinisnone$.
\item[$\sfGame_1$:] This game is exactly as $\sfGame_0$ except that the simulator picks a random $j^*\in[n]$ and uses $\matr{G}$ to check whether $\vecb{c}_{j^*}=b_{i^*,j^*}\vecb{g}_{i^*}+\tilde{w}\vecb{g}_{n+1}$ such that $b_{i^*,j^*}\notin\bits$. It aborts if this is not the case.
\item[$\sfGame_2$:] This game is exactly as $\sfGame_1$ except that $\overline{\matr{G}}\gets\distlin_1^{mn, m(i^*-1)+j^*}$.
\end{description}

It is obvious that the first two games are indistinguishable. 
The rest of the argument goes as follows. 

\begin{lemma} $\Pr\left[ \mathsf{Game}_1(\advA)=1\right]\geq\dfrac{1}{n}\Pr\left[\mathsf{Game}_0(\advA)=1\right].$
\end{lemma}

\begin{proof}  The probability that
 $\mathsf{Game}_1(\advA)=1$ is the probability that  a) $\mathsf{Game}_0(\advA)=1$ and
b)  $b_{i^*,j^*} \notin \bits$. The view of adversary $\advA$ is independent of $j^*$, while, if $\mathsf{Game_0}(\advA)=1$, then there is at least one index $\ell \in [n]$ such that $[\vecb{c}_\ell]_1\notin\Lang_{[\matr{G}]_1,\sfbits}\Longrightarrow b_{i^*,\ell} \notin \bits$. Thus, 
the probability that the event described in b) occurs conditioned on $\mathsf{Game_0}(\advA)=1$, is greater than or equal to $1/n$ and the lemma follows.
\end{proof}

\begin{lemma} There exists a\ $\dist_1$-$\mddh_{\GG_1}$ adversary $\advD$ such that
$|\Pr\left[\mathsf{Game}_{1}(\advA)=1\right]$ $-\Pr\left[\mathsf{Game}_{2}(\advA)=1\right]|$ $\leq
    \mathsf{Adv}_{\distlin_1,\ggen_a}(\advD).$
\end{lemma}
\begin{proof}
We construct an adversary $\advD$ that receives 
a challenge $([\matr{A}]_1,[\vecb{u}]_1)$ of the 
$\distlin_1$-$\mddh_{\GG_1}$ Assumption. From this challenge, $\advD$ just defines the matrix  $[\overline{\matr{G}}]_1\in\GG_1^{2\times(mn+1)}$ as the matrix whose last column consists of $[\matr{A}]_1$, the ith column consists of $[\vecb{u}]_1$ and the rest of the columns are random vectors in the image of $[\matrA]_1$. Then $\advD$ honestly simulates the rest of the CRS, gives it as input to $\advA$, and outputs whatever $\advA$ outputs.

Obviously, when $[\vecb{u}]_1$ is sampled from 
the image of $[\matr{A}]_1,$ $\overline{\matr{G}}$ follows the distribution $\distlinizeroone$ and $\advD$ perfectly simulates $\sfGame_1$, while if $[\vecb{u}]_1$ is a uniform element of $\GG^2_1$, $\overline{\matr{G}}$ follows the distribution $\distlinisnone$ and $\advD$ perfectly simulates $\sfGame_2$. 
%It is immediate to see that adversary $\advD$ perfectly simulates $\sfGame_1$ when $\overline{\matr{G}}\gets\dist_{2,mn+1}^0$ and $\sfGame_2$ when $\overline{\matr{G}}\gets\dist_{2,mn+1}^{m(i^*-1)+j^*}$. The rest of the proof follows from Lemma \ref{lemma:dist-i}.  
\end{proof}

\begin{lemma}
There exists adversaries $\advB_1,\advB_2$ such that $\Pr[\sfGame_2(\advA)=1]\leq\adv_{\Pi_\sfcom}(\advB_1)+\adv_{\Pi_\sfbits}(\advB_2)$.
\end{lemma}

\begin{proof}
If $\sfGame(\advA)=1$, then $b_{i^*,j^*}\notin\bits$ while all the verification equations are accepted. Given that $\ovecb{g}_{m(i^*-1)+j^*}$ is linearly independent from $\{\ovecb{g}_j:j\neq m(i^*-1)+j^*\}$, it holds that $\{\ovecb{g}_{m(i^*-1)+j^*},\ovecb{g}_{mn+1}\}$ is a basis for $\Z_q^2$ and thus we can define $\bb_{i^*,j^*},\overline{w}_{h,i}$ as the unique coefficients in $\Z_q$ such that $\ovecb{c} = \bb_{i^*,j^*}\ovecb{g}_{m(i^*-1)+j^*}+\overline{w}_{h,i}\ovecb{g}_{mn+1}$. If $b_{i^*,j^*}\neq\bb_{i^*,j^*}$, then $([\vecb{c}]_1,[\ovecb{c}]_1)$ can not open to the same value and we can construct an adversary $\advB_1$ against $\Pi_\sfcom$. Else, it must be the case that $\bb_{i^*,j^*}=b_{i^*,j^*}\notin\bits$. Therefore, if an adversary $\advB_2$ simulates $\sfGame_2$ until $\advA$ halts and outputs $([\ovecb{c}]_1,\pi_\sfbits)$, then $\advB_2$ breaks soundness of $\Pi_\sfbits$. 
\end{proof}

{\bf Zero-Knowledge.} Given that $\overline{\matr{G}}$ defines perfectly hiding commitments, $[\ovecb{c}]_1$ can be opened to any value. Therefore $[\ovecb{c}]_1$ and $[\vecb{c}]_1$ share a common opening and $[\ovecb{c}]_1\in\Lang_{\overline{ck},\sfbits}$, and thus $\pi_\sfcom$ and $\pi_\sfbits$ are correctly distributed.  
\end{proof}


