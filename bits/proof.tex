\begin{theorem} \label{theo:bits}
The proof system described on Sect. \ref{sec:bits-scheme} is a QA-NIZK proof system with Perfect Completeness, Computational Soundness, and Perfect Zero-Knowledge.
\end{theorem}	
\begin{proof}
We remark that proof of Completeness and Zero-Knowledge is the same for any distribution $\distlininone$.
\begin{description}
\item[Perfect Completeness:]
Note that,
by definition of $\matr{C}_{i,j}$ and $\matr{D}_{i,j}$, 
$[\matr{C}_{i,j}]_1[\matr{I}]_2+[\matr{I}]_1[\matr{D}_{i,j}]_2=$
$[\vecb{g}_{i}]_1[\vecb{h}_j]_2^\top$.  Since $b_i(b_i-1) = 0$ for each $i\in[m]$,
\begin{eqnarray*}
\lefteqn{
[\vecb{c}]_1( [\vecb{d}]_2 - \sum_{i\in[m]} [\vecb{h}_{i}]_2)^\top}\\
& = & 
    \sum_{i \in [m]}\left(
        b_i s[\vecb{g}_{i}]_1[\vecb{h}_{m+1}]_2^{\top}+
        r(b_i-1)[\vecb{g}_{m+1}]_1[\vecb{h}_i]_2^{\top}+
        \sum_{j \in [m]} b_i (b_j-1)[\vecb{g}_{i}]_1[\vecb{h}_{j}]_2^{\top}
    \right)
\\ & & \mbox{ }
    + rs[\vecb{g}_{m+1}]_1[\vecb{h}_{m+1}]_2^{\top}\\
& = & 
    \left(\sum_{i\in[m]}b_is[\vecb{g}_{i}]_1\vecb{h}_{m+1}^{\top}+r(b_i-1)[\vecb{g}_{m+1}]_1\vecb{h}_i^{\top}+
        \sum_{\substack{j \in [m]\\j\neq i}}b_i(b_j-1)[\vecb{g}_{i}]_1\vecb{h}_{j}^{\top}\right)[\matr{I}]_2\\
& & \mbox{ }
    +rs[\vecb{g}_{m+1}]_1\vecb{h}_{m+1}^{\top}[\matr{I}]_2
    +[\matr{R}]_1[\matr{I}]_2 + [\matr{I}]_1[-\matr{R}]_2
\\ & = &
    [\matr{\Theta}]_1[\matr{I}]_2+
    [\matr{I}]_1[\matr{\Pi}]_2.
\end{eqnarray*}
Finally, the rest of the proof follows from completeness of $\Pi_\sfsum$ and $\Pi_\sfcom$. 

\item[Soundness:] When $\matr{G}$ is sampled from $\distlinizeroone$ it suffices to prove that the commitment $[\vecb{c}]_1$ output by the adversary is in $\Span([\matr{G}]_1)$ since, by the perfect hiding property, $[\vecb{c}]_1$ can be opened to any $\vecb{b}\in\bits^m$ thus $[\vecb{c}]_1\in\Lang_{ck,\sfbits}$. If $[\vecb{c}]_1\notin\Span([\matr{G}]_1)$, then we can break the (strong) soundness of the proof that $[\vecb{c}]_1$ and $[\vecb{d}]_2$ open to the same value, since that proof implies that there exist $\vecb{x},r,s$ such that $[\vecb{c}]_1=[\matr{G}]_1\smallpmatrix{\vecb{x}\\r}$ and $[\vecb{d}]_2=[\matr{H}]_2\smallpmatrix{\vecb{x}\\s}$. Therefore, we construct an adversary $\advB$ against the strong soundness of $\Pi_\sfcom$ that simulates $\advA$ until it halts and outputs $([\vecb{c}]_1,[\vecb{d}]_2,\pi_\sfcom)$. Note that, in order to simulate the CRS $\advB$ requires $\matr{H}$, but this is not a problem since is part of the input in the strong soundness game.
 
When $\matr{G}$ is sampled from $\distlinisnone$, ${i^*}>0$, the proof follows from the indistinguishability of the following three games:
\begin{itemize}
\item[$\mathsf{Real}$:] This is the real Soundness game. The output is 1 if the adversary submits some $[\vecb{c}]_1\notin\Lang_{ck,\sfbits}$ and the corresponding proof which is accepted by the verifier.
\item[$\sfGame_0$:] This identical to $\mathsf{Real}$, except that $\algK_1$ does not receive $[\matr{G}]_1$ as a input but
it samples $\matr{G}$ itself according to $\distlinisnone$.
\item[$\sfGame_1$:] This game is identical to $\sfGame_0$ except that now $\matr{H}\gets\distlinisnone$.
\end{itemize}

It is obvious that the first two games are indistinguishable. The rest of the argument goes as follows.

\begin{lemma} There exists a\ $\distlin_1$-$\mddh_{\GG_2}$ adversary $\advD$ such that
$|\Pr\left[\mathsf{Game}_{0}(\advA)=1\right]$ $-\Pr\left[\mathsf{Game}_{1}(\advA)=1\right]|$ $\leq \mathsf{Adv}_{\distlin_1,\ggen_a}(\advD).$
\end{lemma}
\begin{proof}
We construct an adversary $\advD$ that receives 
a challenge $([\matr{A}]_2,[\vecb{u}]_2)$ of the 
$\distlin_1$-$\mddh_{\GG_2}$ Assumption. From this challenge, $\advD$ just defines the matrix  $[\matr{H}]_2\in\GG_2^{2\times(m+1)}$ as the matrix whose last column consists of $[\matr{A}]_2$, the ith column consists of $[\vecb{u}]_2$ and the rest of the columns are random vectors in the image of $[\matrA]_2$. 
Obviously, when $[\vecb{u}]_2$ is sampled from 
the image of $[\matr{A}]_2,$ $\matr{H}$ follows the distribution $\distlinizeroone$, while if $[\vecb{u}]_2$ is a uniform element of $\GG^2$, $\matr{H}$ follows the distribution $\distlinisnone$. 
 
Adversary $\advD$ samples
$\matr{G} \gets \distlinisnone$. Given that $\advD$ does not know the discrete logarithms of $[\matr{H}]_2$, it cannot compute the pairs $(\matr{C}_{i,j},\matr{D}_{i,j})$ exactly as in $\sfGame_0$. Nevertheless, for each $(i,j)\in\indexSet{m}{1}$ it can compute identically distributed pairs by picking $\matr{T}\gets\Z_q^{2\times 2}$ and defining
$$
([\matr{C}_{i,j}]_1,[\matr{D}_{i,j}]_2):=([\matr{T}]_1,\vecb{g}_i[\vecb{h}_j]_2^\top-[\matr{T}]_2).
$$
The rest of the elements of the CRS, namely $\crs_\sfcom$ and $\crs_\sfsum$, are honestly computed. When $\matr{H}\gets\distlinizeroone$, $\advD$ perfectly simulates $\sfGame_0$, and when $\matr{H}\gets\distlinisnone$, $\advD$ perfectly simulates $\sfGame_1$, which concludes the proof. 
\end{proof}

\begin{lemma}
There exist adversaries $\advB_1$, against the strong soundness of $\Pi_\sfcom$, and $\advB_2$, against the soundness of $\Pi_\sfsum$, such that $\Pr[\sfGame_1(\advA)=1]\leq 4/q+ \adv_{\Pi_\sfcom}(\advB_1)+\adv_{\Pi_\sfsum}(\advB_2)$.
\end{lemma}
\begin{proof}
With probability $1-4/q$, $\{\vecb{g}_{i^*},\vecb{g}_{m+1}\}$ and $\{\vecb{h}_{i^*},\vecb{h}_{m+1}\}$ are both bases of $\Z_q^2$,
we can define $b_{i^*},\overline{w}_g,\overline{w}_h,\overline{b}_{i^*}$ as the unique coefficients in $\Z_q$ such that $\vecb{c}=b_{i^*}\vecb{g}_{i^*} + \overline{w}_g \vecb{g}_{m+1}$ and $\vecb{d}= \bb_{i^*} \vecb{h}_{i^*} + \overline{w}_h \vecb{h}_{m+1}$.

In particular, if $\advA$ breaks soundness, this implies that $b_{i^*} \notin \{0,1\}$ (since for $i\neq i^*$, 
$\vecb{c}$ can always be opened to 
  choose $b_i=0$). Further, the verifier accepts the proof proof:
$ (
        [\vecb{d}]_2,
        ([\matr{\Theta}]_1, [\matr{\Pi}]_2), 
        \pi_\sfcom,\pi_\sfsum )$
  produced by $\advA$.
We distinguish two cases:
\begin{description}
\item[$b_{i^*} \neq \overline{b}_{i^*}$:] Given that $[\vecb{c}]_1$ and $[\vecb{d}]_2$ are perfectly binding at coordinate $i^*$, if $b_{i^*}\neq\bb_{i^*}$ it is not possible that $[\vecb{c}]_1$ and $[\vecb{d}]_2$ open to the same value. We construct an adversary $\advB_1$ against the strong soundness 
of $\Pi_\sfcom$ that simulates game $\sfGame_1$ with $\advA$ (using $\matr{H}$ to simulate the CRS) until it halts and outputs $([\vecb{c}]_1,[\vecb{d}]_2,\pi_\sfcom)$. If $b_{i^*}\neq\bb_{i^*}$, $\pi_\sfcom$ is a fake proof for $([\vecb{c}]_1,[\vecb{d}]_2)$ opening to the same value and then $\advB_1$ breaks the strong soundness of $\Pi_\sfcom$.
\item[$b_{i^*} = \overline{b}_{i^*}$, 
$b_{i^*}(\overline{b}_{i^*} -1) \neq 0$:]
If we express $\matr{\Theta}+\matr{\Pi}$
as a linear combination of $\{\vecb{g}_{i}\vecb{h}_{j}^{\top}:i,j\in[n+1]\}$, the coordinate of
$\vecb{g}_{i^*}\vecb{h}_{i^*}^\top$ is $b_{i^*}(\bb_{i^*}-1)\neq 0$ and thus $\matr{\Theta}+\matr{\Pi}\notin\Span(\{\matr{C}_{i,j}+\matr{D}_{i,j}:(i,j)\in\indexSet{m}{1}\})$. We construct an adversary $\advB_2$ against the soundness of $\Pi_\sfsum$ that simulates game $\sfGame_1$ with $\advA$ until it halts and outputs $([\matr{\Theta}]_1,[\matr{\Pi}]_2,\pi_\sfsum)$. If $b_{i^*} = \overline{b}_{i^*}$ but $b_{i^*}(\overline{b}_{i^*} -1) \neq 0$, $\pi_\sfsum$ is a fake proof for $([\matr{\Theta}]_1,[\matr{\Pi}]_2)$ and then $\advB_2$ breaks the soundness of $\Pi_\sfsum$.
\end{description}
\end{proof}

\item[Perfect Zero-Knowledge:] First, note that the vector $[\vecb{d}]_2 \in \GG_2^2$ output by the prover and the vector output by $\algS_2$ follow exactly the same distribution. This is because $\matr{H}\gets\distlinizeroone$ defines perfectly hiding commitments. In particular, although the simulator $\algS_2$ does not know $\vecb{b} \in \{0,1\}^{m}$ such that $[\vecb{c}]_1=[\matr{G}]_1\smallpmatrix{\vecb{b}\\r}$, for some $r\in\Z_q$, 
there exists $s \in \Z_q$ such that $[\vecb{d}]_2=[\matr{H}]_2\smallpmatrix{\vecb{b}\\ s}$. 

Since $\matr{R}$ is chosen uniformly at random in $\Z_q^{2 \times 2}$, the proof $([\matr{\Theta}]_1, [\matr{\Pi}]_2)$ is uniformly distributed conditioned on satisfying check 1) of algorithm $\algV$.
 Finally, the rest of the proof follows from Zero-Knowledge of $\Pi_\sfsum$ and $\Pi_\sfcom$.
\end{description}
\end{proof}
