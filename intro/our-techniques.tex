Showing that commitments $c_1,\ldots,c_n$ opens to $x_1,\ldots,x_n\in S$ is equivalent to show that $c_1,\ldots,c_n$ is in the language $\Lang_{ck,S}^n:=\Lang_{ck,S}\times\ldots\times\Lang_{ck,S}$, where
$$
\Lang_{ck,S}:=\{c:\exists x\in S, w\in\Z_q \text{ s.t. } c=\Com_{ck}(x;w)\}\text{ and }ck\gets\distk.
$$
Note that the language dependes on an external and honestly sampled parameter $ck$. Jutla and Roy introduced the notion of Quasi-Adaptive NIZK (QA-NIZK) \cite{AC:JutRoy13} which allows to define parameterized languages $\Lang_{\rho}$. For example $\rho$ can describe a matrix with coefficients in a group $\GG_1$ and $\Lang_{\rho}$ is the linear subspace generated by the columns of the matrix, or $\rho=(ck,S)$ and $\Lang_{\rho}=\Lang_{ck,S}$ as in our case.

If $\Com$ is an homomorphic commitment scheme, that is $\alpha\Com_{ck}(x;r)+\beta\Com_{ck}(y;s)=\Com_{ck}(\alpha x+\beta y;\alpha r + \beta s)$,  membserhip in $\Lang_{ck,S}$ is equivalent to show the satisifiability of the following set of linear and quadratic equations:
\begin{align}
c_j = \sum_{i\in[m]}\Com_{ck}(s_i;0)b_{i,j}+\Com_{ck}(0;1)r_j,&
& 1= \sum_{i\in[m]}b_{i,j},      \label{eq:lin}
\end{align}
and
\begin{align}
b_{i,j}(b_{i,j}-1)=0 \label{eq:quad},
\end{align}
for each $i\in[m],j\in[n]$ and where $m:=|S|$.

A sequence of recent works have constructed very efficient constant-size QA-NIZK arguments of membership in linear subspaces of $\GG_1^n$ \cite{EC:LPJY14,C:JutRoy14,EC:AbdBenPoi15,EC:KilWee15}.
Further, in a recent work \cite{AC:GonHevRaf15} it has been constructed a QA-NIZK constant-size argument that each of many commitments opens to either 0 or 1, denoted $(c_1,\ldots,c_n)^\top\in\Lang_{ck,\sfbits}$. Below, we show how to use this protocols to construct an argument for the satisifiability of (\ref{eq:lin}) and (\ref{eq:quad}).

Note that (\ref{eq:lin}) is equivalent to $(c_1,\ldots,c_n,1,\ldots,1)^\top\in\rmIm(\matr{\Xi})$, for the appropiate matrix $\matr{\Xi}$.
In order to show (\ref{eq:quad}) we additionally compute commitments $d_{1,1},\ldots,d_{m,n}$ to  variables $b_{1,1},\ldots,b_{m,n}$, show that $(d_{1,1},\ldots,d_{m,n})^\top\in\Lang_{ck',\sfbits}$, and then show that $d_{i,j}=\Com_{ck'}(1;0)b_{i,j}+\Com_{ck'}(0;1)s_{i,j}$. The latter statement can be shown jointly with (\ref{eq:lin}) with a proof that $(c_1,\ldots,c_n,1,\ldots,1,d_{1,1}\ldots,d_{m,n})^\top$ is in the image of $\matr{\Xi}$, adding the appropiate rows to $\matr{\Xi}$. Pictorially, when $|S|=3$ and $n=2$, we have 
%\begin{small}
$$
\newcommand{\cc}{\tikz{\node[draw,pattern=north west lines,rectangle,minimum width=.9cm,minimum height=.3cm] {};}}
\newcommand{\cd}{\tikz{\node[draw,pattern=north east lines,rectangle,minimum width=.3cm,minimum height=.3cm] {};}}
\newcommand{\cw}{\tikz{\node[draw,rectangle,minimum width=.9cm,minimum height=.3cm] {};}}
\newcommand{\cg}{\tikz{\node[draw,pattern=crosshatch,rectangle,minimum width=.3cm,minimum height=.3cm] {};}}
\newcommand{\cy}{\tikz{\node[draw,pattern=grid,rectangle,minimum width=.3cm,minimum height=.3cm] {};}}
\newcommand{\cl}{\tikz{\node[draw,fill=cyan,rectangle,minimum width=.15cm,minimum height=.44cm] {};}}
\newcommand{\cww}{\tikz{\node[draw,rectangle,minimum width=.15cm,minimum height=.44cm] {};}}
\arraycolsep=0pt\def\arraystretch{0}
\matr{\Xi} =
\left(\begin{array}
{c  c   c               c c  c                    c   c   c   c   c   c   c   c}
\multicolumn{3}{c}{\cc}& &   &                   &\cd&   &   &   &   &   &   &    \\
   &   &                 &\multicolumn{3}{c}{\cc}&   &\cd&   &   &   &   &   &    \\
\multicolumn{3}{c}{\cw}& &   &                   &   &   &   &   &   &   &   &    \\ 
   &   &                 &\multicolumn{3}{c}{\cw}&   &   &   &   &   &   &   &    \\
\cg&   &                 &   &   &               &   &   &\cy&   &   &   &   &    \\
   &\cg&                 &   &   &               &   &   &   &\cy&   &   &   &    \\
   &   &\cg              &   &   &               &   &   &   &   &\cy&   &   &    \\
   &   &                 &\cg&   &               &   &   &   &   &   &\cy&   &    \\
   &   &                 &   &\cg&               &   &   &   &   &   &   &\cy&    \\
   &   &                 &   &   &\cg            &   &   &   &   &   &   &   &\cy
\end{array}\right),
\text{ where }
\begin{array}{l}
\cc:=(\Com_{ck}(s_1;0),\Com_{ck}(s_2;0),\Com_{ck}(s_3;0)),\\
\cd:=\Com_{ck}(0;1),\\
\cw:=(1,1,1),\\
\cg:=\Com_{ck'}(1;0),\\
\cy:=\Com_{ck'}(0;1).
\end{array}
$$
However, this approach does not fulfill our requirements, since the size of the proof is $O(1)+\sum_{i,j}|d_{i,j}|=O(mn)$, and thus the size is not independent of $n$.

\subsubsection{Length-reducing homomorphic commitments.} Instead of using individual commitments to each $b_{i,j}$, we commit to many $b_{i,j}$s using a \emph{length-reducing}, homomorphic commitment scheme with constant-size commitments. Specifically, we compute $d_i:=\Com_{ck'}((b_{i,1}$,$\ldots$,$b_{i,n})^\top$;$s_i)=\sum_{i\in[m]}b_{i,j}\Com(\vecb{e}_i$;$0)+s_i\Com_{ck'}(0$;$1)$, where $\vecb{e}_i\in\Z_q^n$ is the $i$th canonical vector and $i\in[m]$. In the case $|S|=3$ and $n=2$, the matrix $\matr{\Xi}$ is as follows 
$$
\newcommand{\cc}{\tikz{\node[draw,pattern=north west lines,rectangle,minimum width=.9cm,minimum height=.3cm] {};}}
\newcommand{\cd}{\tikz{\node[draw,pattern=north east lines,rectangle,minimum width=.3cm,minimum height=.3cm] {};}}
\newcommand{\cw}{\tikz{\node[draw,rectangle,minimum width=.9cm,minimum height=.3cm] {};}}
\newcommand{\cg}{\tikz{\node[draw,pattern=crosshatch,rectangle,minimum width=.3cm,minimum height=.3cm] {};}}
\newcommand{\cy}{\tikz{\node[draw,pattern=grid,rectangle,minimum width=.3cm,minimum height=.3cm] {};}}
\newcommand{\cgg}{\tikz{\node[draw,pattern=dots,rectangle,minimum width=.3cm,minimum height=.3cm] {};}}
\arraycolsep=0pt\def\arraystretch{0}
\matr{\Xi} =
\left(\begin{array}
{c  c   c               c   c   c               c   c   c   c   c   c   c   c}
\multicolumn{3}{c}{\cc}&    &    &             &\cd&   &   &   &   &   &   & \\
   &   &               &\multicolumn{3}{c}{\cc}&   &\cd&   &   &   &   &   & \\
\multicolumn{3}{c}{\cw}&    &    &             &   &   &   &   &   &   &   & \\ 
   &   &               &\multicolumn{3}{c}{\cw}&   &   &   &   &   &   &   & \\
\cg&   &               &\cgg&    &             &   &   &\cy&   &   &   &   & \\
   &\cg&               &    &\cgg&             &   &   &   &\cy&   &   &   & \\
   &   &\cg            &    &    &\cgg         &   &   &   &   &\cy&   &   &
\end{array}\right),
\text{ where }
\begin{array}{l}
\cc:=(\Com_{ck}(s_1;0),\Com_{ck}(s_2;0),\Com_{ck}(s_3;0)),\\
\cd:=\Com_{ck}(0;1),\\
\cw:=(1,1,1),\\
\cg:=\Com_{ck'}((1,0)^\top;0),\cgg:=\Com_{ck'}((0,1)^\top;0), \\
\cy:=\Com_{ck'}(0;1).
\end{array}
$$
But, since a length-reducing commitment have multiple openings, is no longer the case that the proofs of membership in $\rmIm(\matr{\Xi})$ and $\Lang_{ck',\sfbits}$ implies $(c_1,\ldots,c_n)^\top\in\Lang_{ck,S}^n$. Indeed, suppose that in the example's soundness game the adversary produces $(c_1,c_2)$ and say $c_1\notin\Lang_{ck,S}$. If $(w_{1,1},w_{2,1},w_{3,1},\ldots)^\top\in\Z_q^{11}$ is a witness for the proof of membership in $\rmIm(\matr{\Xi})$, then $c_1=\Com_{ck}(w_{1,1}s_1+w_{2,1}s_2+w_{3,1}s_3;r_1)$ and necessarily $(w_{1,1},w_{2,1},w_{3,1})^\top\notin\bits^3$ and say $w_{2,1}\notin\bits$. The problem is that it could be the case that $d_2$ can be opened to $(w_{2,1},w_{2,2})^\top\notin\bits^2$ but also to $(b_{2,1},b_{2,2})^\top\in\bits^2$. Note that this does not contradict bindingness of $\Com_{ck'}$, since is not clear how to extract $w_{2,1},w_{2,1},b_{2,1},b_{2,2}$.

To solve this problem, we introduce a new length-reducing homomorphic commitment scheme. The commitment key $ck'$ is sampled from the distribution $\distk^0$, but it can also be sampled from any of the distributions $\distk^1,\ldots,\distk^n$ with the following properties:
\begin{enumerate}
\item $\distk^0\capprox\distk^1,\ldots\distk^0\capprox\distk^n$.
\item $\distk^0$ defines perfectly hiding and computationally binding commitments.
\item $\distk^i$ defines perfectly binding commitments \textbf{for the $i$ th coordinate} for all $i>0$.
\item Commitments are of constant size, for all $\distk^i$.
\end{enumerate} 
Formally, the third property means that, when $ck'\gets\distk^i$, for any commitment $c$, for any $\vecb{w}\in\Z_q^n,s\in\Z_q$, there is a unique $w\in\Z_q$ such that $c=\Com_{ck}(\vecb{w};s)\Longrightarrow x_i=x$.

We can solve the problem of our example by sampling $ck'$ from $\distk^j$, for random $j\in\{1,2\}$. Note that the adversary can not notice this change and then, with probability $1/2$, $j=1$ (or equal to whichever is the index in which the adversary is cheating). The fact that $c_1\notin\Lang_{ck,S}$ implies that $(w_{1,1},w_{2,1},w_{3,1})^\top\notin\bits^3$, and since $d_1,d_2,d_3$ have unique openings at coordinate $j=1$, $(d_1,d_2,d_3)^\top\notin\Lang_{ck',\sfbits}$.  

Now we face our last problem: the construction from \cite{AC:GonHevRaf15} explicitly require perfectly binding commitments.

\subsubsection{Overcoming the restriction from \cite{AC:GonHevRaf15}.}
We construct a QA-NIZK proof system that overcomes the limitation of perfectly binding commitments from \cite{AC:GonHevRaf15}. In \cite{AC:GonHevRaf15} the authors constructed a QA-NIZK proof system for the language of commitments that open to bit-strings
$$
\Lang_{ck,\sfbits} :=\{c:\exists \vecb{b}\in\bits^n,\vecb{w}\in\Z_q^m \text{ s.t. } c=\Com_{ck}(\vecb{b};\vecb{w})\},
$$
where the commitments keys $ck$ are sampled from some distribution $\distk$ that defines \textbf{perfectly binding} commitments. We show that (a simplified version of) a sub-protocol implicitly used in the proof system from \cite{AC:GonHevRaf15} is a QA-NIZK proof system \textbf{even when commitments are not perfectly binding}. Specifically, we show that it is QA-NIZK whenever the commitment keys are sampled from any of the distributions $\distk^0,\distk^1,\ldots,\distk^n$.
%\subsubsection{Aggregated Set Membership Argument.}
%Putting all together, our argument that $(c_1,\ldots,c_n)\in\Lang_{ck,S}^n$ is composed of:
%\begin{enumerate}[label=\alph*)]
%\item Commitments $d_i\gets\Com_{ck'}((b_{i,1},\ldots,b_{i,n})^\top)$, for each $i\in[m]$ ($\ell=m$), where $ck'\gets\distk^0$.
%\item A QA-NIZK proof that $d_1,\ldots,d_m\in\Lang_{ck',\sfbits}$.
%\item A QA-NIZK proof that $(c_1,\ldots,c_n,1,\ldots,1,d_1,\ldots,d_{m})^\top\in\rmIm(\matr{\Xi})$. \label{popin}
%\end{enumerate}
%
%For a simpler exposition of the soundness proof, consider the case $|S|=3$ and $n=2$. Given that $\ell=m=2$,  $d_1,d_2,d_3$ are a length-reducing commitments to vectors in $\Z_q^2$, and the matrix $\matr{\Xi}$ is constructed as follows
%In this case we can randomly pick $j^*\gets\{1,2\}$ and sample $ck'\gets\mathcal{K}^{j^*}$, and in then $d_1,d_2,d_3$ have unique openings $w_{1,j^*},w_{2,j^*}w_{3,j^*}$ at coordinate $j^*$. Given that there is some $j\in\{1,2\}$, say $j=1$ such that $c_j\notin\Lang_{ck,S}$, with probability $1/2$ $j=j^*$. The proof of membership in $\rmIm(\matr{\Xi})$ implies that there exists some $(\tilde{w}_{1,1},\tilde{w}_{2,1},\tilde{w}_{3,1},\tilde{w}_{1,2},\tilde{w}_{2,2},\tilde{w}_{3,2},r_1,r_2,s_1,s_2,s_3)^\top\in\Z_q^{11}$
%$$
%\newcommand{\cc}{\tikz{\node[draw,pattern=north west lines,rectangle,minimum width=.9cm,minimum height=.3cm] {};}}
%\newcommand{\cd}{\tikz{\node[draw,pattern=north east lines,rectangle,minimum width=.3cm,minimum height=.3cm] {};}}
%\newcommand{\cw}{\tikz{\node[draw,rectangle,minimum width=.9cm,minimum height=.3cm] {};}}
%\newcommand{\cg}{\tikz{\node[draw,pattern=crosshatch,rectangle,minimum width=.3cm,minimum height=.3cm] {};}}
%\newcommand{\cy}{\tikz{\node[draw,pattern=grid,rectangle,minimum width=.3cm,minimum height=.3cm] {};}}
%\newcommand{\cgg}{\tikz{\node[draw,pattern=dots,rectangle,minimum width=.3cm,minimum height=.3cm] {};}}
%\smallpmatrix{c_1\\c_2\\1\\1\\d_1\\d_2\\d_3}=
%\arraycolsep=0pt\def\arraystretch{0}
%\left(\begin{array}
%{c  c   c               c   c   c               c   c   c   c   c   c   c   c}
%\multicolumn{3}{c}{\cc}&    &    &             &\cd&   &   &   &   &   &   & \\
%   &   &               &\multicolumn{3}{c}{\cc}&   &\cd&   &   &   &   &   & \\
%\multicolumn{3}{c}{\cw}&    &    &             &   &   &   &   &   &   &   & \\ 
%   &   &               &\multicolumn{3}{c}{\cw}&   &   &   &   &   &   &   & \\
%\cg&   &               &\cgg&    &             &   &   &\cy&   &   &   &   & \\
%   &\cg&               &    &\cgg&             &   &   &   &\cy&   &   &   & \\
%   &   &\cg            &    &    &\cgg         &   &   &   &   &\cy&   &   &
%\end{array}\right)
%\smallpmatrix{w_{1,1}\\w_{2,1}\\w_{3,1}\\w_{1,2}\\w_{2,2}\\w_{3,2}\\r_1\\r_2\\s_1\\s_2\\s_3}
%.$$
%Then $c_1\notin\Lang_{ck,S}\Rightarrow w_{i^*,1}\notin\bits$ for some $i^*\in\{1,2,3\}$. Therefore $d_{i^*}=\tilde{w}_{i^*,1}\Com((1,0)^\top;0) + \tilde{w}_{i^*,2}\Com((0,1)^\top;0)+s_{i^*}\Com(0;1)=\Com((\tilde{w}_{i^*,1},\tilde{w}_{i^*,2})^\top;s_{i^*})$ but given the uniqueness of $w_{i^*,1}$, we conclude that $\tilde{w}_{i^*,1}=w_{i^*,1}$ and thus $d_{i^*}\notin\Lang_{ck,\sfbits}$.
%
%The matrix $\matr{\Xi}$ is chosen in a way that the statement from \ref{popin} is equivalent to show that there exist $b_{1,1},\ldots,b_{m,n}$ which are solutions to (\ref{eq:lin}) and $d_i=\Com_{ck'}((b_{i,1},\ldots,b_{i,n})^\top)$ for each $i\in[m]$.

%At a high level, soundness follows from the following argument. If there is some $j\in[n]$ such that $x_j\notin S$, given c) and the way $\matr{\Xi}$ is chosen, there is necessarily some $i\in[m]$ such that $b_{i,j}\notin\bits$. The trick is now to sample $ck'$ from $\distk^{j^*}$, for random $j^*\in[n]$, and thus $d_1,\ldots,d_m$ are perfectly binding at coordinate $j^*$ unequivocally defining $b_{1,j^*},\ldots,b_{m,j^*}$. Given that $\distk^0 \capprox \distk^{j^*}$, the adversary should have approximately the same advantage of breaking soundness whether $ck'\gets\distk^0$ or $ck'\gets\distk^{j^*}$. In the latter case, with probability at least $1/n$, $j=j^*$ and thus $d_i\notin\Lang_{ck,\sfbits}$, which breaks soundness of the proof from b). 


