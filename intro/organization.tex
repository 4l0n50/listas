The remainder of this paper is organized as follows. In Sect. \ref{secc:prelim} we introduce our notation; in Sect. \ref{sec:comp-assump} we introduce the computational assumptions used through the paper; in Sect. \ref{sec:comm} we introduce GS commitments and the two-dimensional Multi-Pedersen commitments; in Sect. \ref{sec:gs} we quickly introduce GS proofs; in Sect. \ref{sec:zkdef} we recall the formal definition of QA-NIZK proofs; and in Sect. \ref{Sect:QANIZKlinspace} we recall QA-NIZK proof systems which we use as building blocks.

In Sect. \ref{sec:bits} we show how to construct a QA-NIZK argument that a two-dimensional Multi-Pedersen commitment open to a bit-string, and in Sect. \ref{sec:bits-n} we show how to aggregate many of this proofs (this is necessary for obtaining more efficient applications).
In Sect. \ref{sec:bin-lang} we present the aggregated NIZK set membership argument. Finally, in Sects. \ref{sec:shuffle} and \ref{sec:range-proof} we present our Shuffle and Range Argument, respectively.
