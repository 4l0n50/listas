The remainder of this paper is organized as follows. In Sect. \ref{secc:prelim} we introduce the notation, assumptions, and cryptographic primitives used in this work. In Sect. \ref{sec:mp-comm} we introduce the generalized Multi-Pedersen commitment scheme.
In Sect. \ref{sec:bits} we construct a QA-NIZK argument that a two-dimensional Multi-Pedersen commitment open to a bit-string, and in Sect. \ref{sec:bits-n} we show how to aggregate many of this proofs (this is necessary for obtaining more efficient applications).
In Sect. \ref{sec:bin-lang} we present the aggregated NIZK set membership argument. Finally, in Sects. \ref{sec:shuffle} and \ref{sec:range-proof} we present our Shuffle and Range Argument, respectively.
