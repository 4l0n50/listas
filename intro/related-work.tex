\subsubsection{Zero Knowledge Set Membership Arguments.}
Camenisch et al. constructed $\Theta(1)$ interactive Zero-Knowledge set membership arguments using Boneh-Boyen Signatures, and they prove them secure under the $q$-SDH assumption \cite{AC:CamChaShe08}. Bayer and Groth constructed $\Theta(\log |S|)$ interactive Zero-Knowledge arguments for polynomial evaluation, which can be used to construct set membership arguments, relying only on the discrete logarithm assumption \cite{EC:BayGro13}.
%There is also a vast literature on \emph{Accumulators} \cite{EC:BenDeM93,C:CamLys02,AC:TsuXu03,RSA:LNguyen05,PKC:CamKohSor09,ICICS:WanWanPie07}, which can be used to construct $\Theta(1)$ proofs {\color{red}(creo q con accumulators el statement no es (no puede ser?) un commitment)}.
However, none of the previous constructions have addressed the problem of aggregating many proofs, and a direct use of them will end up with a proof of size $\Omega(n)$.
%Further, none of them seems to consider the case when $S$ is an arbitrarily chosen subset of $\GG_1$.

\iffull
\subsubsection{Vector Commitments.}
The two dimensional Multi-Pedersen commitments introduced in this work bears some similarities with \emph{Vector Commitments} introduced by Catalano and Fiore \cite{PKC:CatFio13}. In fact, one can commit to a vector $\vecb{m}\in\Z_q^n$ as $[\vecb{c}]_1:=\MP.\Com_{ck}(\vecb{m};0)$ (see more details in Sect. \ref{sec:mp-comm}) and show the so called \emph{Position Binding} property, that is, show that it opens to $m_i$ at coordinate $i$. Indeed, we can compute a proof that $([\vecb{c}]_1-\MP.\Com_{ck}(m_i\vecb{e}_i;0))\in\Span(\{\MP.\Com_{ck}(\vecb{e}_j;0):j\neq i\}))$, where $\vecb{e}_i\in\Z_q^n$ is the $i$ th canonical vector. However, we do not elaborate more on this application since Catalano and Fiore's construction is (by a constant factor) more efficient in terms of CRS size and commitment size, and also relies on weaker assumptions.
\fi

\subsubsection{NIZK Shuffle and Range Arguments.}
The most efficient NIZK Shuffle argument under falsifiable assumptions is the one from Groth and Lu \cite{AC:GroLu07}, which works for BBS ciphertexts. The proof size is linear in the number of ciphertexts, specifically $15n + 120$ group elements in Type I groups. The security of their construction relies on two assumptions: the \emph{Paring Product Assumption} and the \emph{Permutation Pairing Assumption}. The first assumption is a $\dist_{n,2}\mbox{-}\kermdh$ Assumption, when $\matr{M}\gets\dist_{n,2}$ is of the form $\matr{M}^\top:=\pmatri{x_1,\ldots,x_n\\x_1^2,\ldots,x_n^2}$ for $x_i\gets\Z_q$, $i\in[n]$. The second assumption is proven generically secure in \cite{AC:GroLu07} but it seems to be unrelated with any other assumption.

Using non-falsifiable sssumptions (i.e. Knowledge of Exponent type of assumptions), Lipmaa and Zhang \cite{SCN:LipZha12} constructed a shuffle argument with communication $6n\sG+11\sH$, and recently Fauzi and Lipmaa constructed a shuffle argument with communication $(5n+2)\sG+2n\sH$ \cite{EPRINT:FauLip15}.

Rial, Kohlweiss, and Preneel constructed a range argument in $[0,2^n-1]$ with communication $\Theta(\frac{n}{\log n -\log\log n})$ and prove it secure under the $q$-HSDH assumption \cite{PAIRING:RiaKohPre09}. One might get rid of the $q$-HSDH assumption replacing the \emph{P-signature} with any \emph{Structure Preserving Signature}, but, since the proof requires $\frac{n}{\log n-\log \log n}$ Groth-Sahai proofs of satisfiability of the signature's verification equation and the signature's size is at least 7 group elements \cite{C:KilPanWee15}, the resulting protocol is far less efficient.
Using non-falsifiable assumptions, Chaabouni, Lipmaa, and Zhang constructed a range argument with constant communication \cite{FC:ChaLipZha12}. 

A detailed comparison of our Shuffle and Range arguments with the most efficient constructions under falsifiable assumptions is depicted in Table \ref{table:eff}.


\begin{table}[h]
\begin{center}
\begin{minipage}{\textwidth}
\begin{center}
\begin{scriptsize}
\begin{tabular}{|l|ll|ll|}
\hline
                                                   & \multicolumn{2}{c|}{Shuffle Argument} & \multicolumn{2}{c|}{Range Argument} \\
                                                   & \cite{AC:GroLu07}          
%& \cite{EPRINT:FauLip15}
 & $\Pi_\mathsf{shuffle}$
                                                   & \cite{PAIRING:RiaKohPre09} & $\Pi_{\mathsf{range}\mbox{-}\mathsf{proof}}$ 
\\ \hline\hline
\rule{0pt}{2.5ex}CRS size                          & $2n + 8$                   
%& $8n + 17$
              & $(n^2+24n+36,23n+37)$                
                                                   & $\Theta(\frac{n}{\log n-\log\log n})$ & $(6n^2,6n^2)$ \\
% (6n^2+13n+n+\frac{n}{klogn}+2klogn,6n^2+13n+n+\frac{n}{klogn}+34)
\rule{0pt}{2.5ex}Proof size                        & $15n + 120$                
%& $(5n+2,2n)$
            & $(4n+17,14)$
                                                   & $\Theta(\frac{n}{\log n-\log\log n})$ & $(\frac{2n}{k\log n},10)$ \\
%$(\frac{2n}{k\log n}+2k\log n+11,10)$
\rule{0pt}{2.5ex}$\algP$'s comp.                   & $51n + 246$               
%& $22n + 11$
             & $11n+17$
                                                   & $\Theta(\frac{n}{\log n-\log\log n})$ & $2n$ \\
%2n+\frac{3n}{klogn}+3k\log n+2
\rule{0pt}{2.5ex}$\algV$'s comp.                   & $75n + 282$               
%& $18n + 6$
              & $13n+55$
                                                   & $\Theta(\frac{n}{\log n-\log\log n})$ & $\frac{4n}{k\log n}$ \\
%\frac{n}{k\log n}+6k\log n+62
\rule{0pt}{2.5ex}Assumption                        & PP                        
%& KE
                    & SXDH+SSDP
                                                   & $q$-HSDH                   & SXDH+SSDP \\\hline 
\end{tabular}
\end{scriptsize}
\end{center}
\caption{Comparison of our Shuffle, $\Pi_\mathsf{shuffle}$, and Range, $\Pi_{\mathsf{range}\mbox{-}\mathsf{proof}}$, arguments with the literature. To increase readability, for $\Pi_{\mathsf{range}\mbox{-}\mathsf{proof}}$ we include only the leading part of the sizes, that is, we write $f(n)$ and we mean $f(n)+o(f(n))$. Notation $(x,y)$ means $x$ elements of $\GG_1$ and $y$ elements of $\GG_2$. ``PP'' stands for the Permutation Pairing assumption.
%and ``KE'' for Knowledge of Exponent assumption ({\color{red} ver cual}).
The prover's computation is measured by the number of exponentiations (i.e. $z[x]_i$) and the verifier's computation is measured by the number of pairings.\label{table:eff}  } 
\end{minipage}
\vspace{-0.54cm}

\end{center}
\end{table}


%\subsubsection{Culpable Soundness.} 
%%The protocols in this work are proven Quasi-Adaptive NIZK (QA-NIZK) \cite{AC:JutRoy13}.
%%basically means that the language can be set based on the constants sampled in the CRS and vice versa. More specifically, and
%%In the scope of this work, it means that the CRS defines \emph{commitment keys} (resp. \emph{public keys}) and we consider the languages of commitments (resp. ciphertexts) computed under those commitment keys (resp. public keys), where the committed value (resp. plaintext) satisfy an additional property. From a technical point of view, 
%%In order to not trivialize the zero-knowledge definition, the simulator must not have access to the discrete log of the commitment-keys (resp. the secret key).
%In the case of shuffle arguments, Groth and Lu \cite{AC:GroLu07}, Lipmaa and Zhang \cite{SCN:LipZha12}, and also Fauzi and Lipmaa \cite{EPRINT:FauLip15}, constructed proof systems that satisfy the weaker notion of \emph{Culpable Soundness}.  This means that the soundness adversary must provide a decryption key for the underlying encryption scheme in order to break soundness. In fact, they noted that it is necessary to prove a weaker notion of soundness since otherwise known impossibility results imply that their shuffle arguments can not be NIZK \cite{TCC:AbeFeh07}.
%
%We also prove a weaker form of soundness since we construct a Quasi-Adaptive NIZK (QA-NIZK) \cite{AC:JutRoy13} proof system. This notion allows the language to depend on values defined in the CRS and, therefore, this values are sampled from an honest distribution. Specifically, this allows to define the public key of the underlying encryption scheme in the CRS and, in the soundness proof where one can also sample trapdoors associated to the CRS, amounts to know the corresponding secret key. In practice, this has the same advantage as Culpable Soundness: access to the encryption scheme's secret key to extract the plaintexts.
%
% considered languages which only depend on the group description and thus no trapdoors that trivialize the zero-knowledge definition exists. However, known impossibility results imply that their shuffle arguments can not be NIZK \cite{TCC:AbeFeh07}. They by-passed the impossibility result proving the weaker notion of \emph{Culpable Soundness}, which means that the soundness adversary must provide a decryption key for the ciphertexts. This bears some similarity with the QA-NIZK case: in the soundness game the decryption key is also known since is sampled by the reduction itself. From a practical point of view, both notions model the fact that some trusted party knows the secret-key.

%In the case of shuffle arguments, Groth and Lu \cite{AC:GroLu07}, Lipmaa and Zhang \cite{SCN:LipZha12}, and also Fauzi and Lipmaa \cite{EPRINT:FauLip15}, considered languages which only depend on the group description and thus no trapdoors that trivialize the zero-knowledge definition exists. However, known impossibility results imply that their shuffle arguments can not be NIZK \cite{TCC:AbeFeh07}. They by-passed the impossibility result proving the weaker notion of \emph{Culpable Soundness}, which means that the soundness adversary must provide a decryption key for the ciphertexts. This bears some similarity with the QA-NIZK case: in the soundness game the decryption key is also known since is sampled by the reduction itself. From a practical point of view, both notions model the fact that some trusted party knows the secret-key.

%In the case of interactive set membership arguments \cite{AC:CamChaShe08,EC:BayGro13}, since no trapdoor is needed in the simulation, they are able to prove \emph{honest-verifier zero-knowledge}. In the non-interactive case \cite{PAIRING:RiaKohPre09} or when removing interaction via the \emph{Fiat-Shamir heuristic}, the simulator do have acces to trapdoors. However, since all the previously cited works consider Pedersen commitments, which are perfectly hiding,  
