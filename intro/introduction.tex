Zero-knowledge proofs are proofs which yield nothing beyond the validity of a certain statement. Although one can prove every NP statement in zero-knowledge (going through a proof of circuit satisfiability, for instance),  the literature has extensively explored 
more efficient alternatives for concrete statements which appear often in practice. Among them, some of the most important are: proofs of membership in linear spaces \cite{AC:JutRoy13,EC:LPJY14,C:JutRoy14,EC:AbdBenPoi15,EC:KilWee15}, range proofs \cite{AC:CamChaShe08,PAIRING:RiaKohPre09,FC:ChaLipZha12}, membership in a set \cite{PAIRING:RiaKohPre09,EC:BayGro13}, or correctness of a shuffle \cite{Chaum81,AC:GroLu07,SCN:LipZha12,EPRINT:FauLip15}. 

These problems have been studied following a variety of approaches and techniques. For instance, they have been studied both in the interactive \cite{EC:BayGro12,AC:CamChaShe08,AC:Groth11} and the non-interactive setting \cite{PAIRING:RiaKohPre09,FC:ChaLipZha12,AC:GroLu07,SCN:LipZha12,EPRINT:FauLip15}, and in the latter setting, both under falsifiable (but not always standard) \cite{PAIRING:RiaKohPre09,AC:GroLu07} and non-falsifiable assumptions \cite{FC:ChaLipZha12,SCN:LipZha12,EPRINT:FauLip15} (like knowledge of exponent type of assumptions). 

Generally speaking, non-interactive zero-knowledge proofs under falsifiable assumptions remain more inefficient than other approaches for the same problem (one notable exception being the recent QA-NIZK arguments of membership in linear spaces of  \cite{EC:LPJY14,C:JutRoy14,EC:KilWee15}). However, this is the most desirable alternative from a cryptographic point of view. Indeed, interaction is not  so convenient in practice and further, there is the additional problem of non-transferablility (a proof might not convince a third party who cannot check if the challenges were computed correctly \cite{C:Pass03}).  On the other hand, non-falsifiable assumptions are very strong assumptions whose use is, at the very least, controversial \cite{EPRINT:BCPR13a,AC:BoyPas15}. Although it might still be interesting to use these assumptions in practice, from a theoretical viewpoint it is definitely worth to explore how to improve efficiency based only on standard assumptions.  

This paper focuses on obtaining efficiency improvements for non-interactive arguments based on falsifiable assumptions for two of the interesting examples discussed above, namely, range proofs and proofs of correctness of a shuffle.  

An argument of Correctness of a Shuffle is an essential tool in the construction of \emph{Mix-nets} \cite{Chaum81}. A Mix-net consists of a series of \emph{mixers}, each of which receives as input a set of $n$ ciphertexts and  outputs a \emph{shuffle} of the input ciphertexts. That is, a \emph{rerandomization} of the set of ciphertexts obtained after applying a \emph{random permutation} to the input set of ciphertexts. To enforce the honest behavior of mixers they are required to produce a zero-knowledge argument that the shuffle was correctly computed.  

A Range argument is a tool often required in e-voting and e-cash scenarios, with the purpose of showing that the opening $y$ of some commitment $c$ is an integer in some interval $[A,B]$. For simplicity, the range considered is usually $[0,2^n-1]$ since a proof in any interval can be reduced to a proof in this interval.

To derive efficiency improvements for these two languages we develop specific techniques that we can apply to both problems. Our resulting proofs are more efficient in terms of proof size and are based on more standard assumptions, but they have a rather large common reference string. They build on the recent arguments for membership in linear spaces of \cite{EC:LPJY14,C:JutRoy14,EC:KilWee15} and the argument for proving that some commitment to a vector of integers in $\Z_q^{n}$ opens to $\{0,1\}^n$ due to \cite{AC:GonHevRaf15}. 
         
