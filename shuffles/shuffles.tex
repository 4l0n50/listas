In a NIZK Shuffle argument one wants to prove that two lists of ciphertexts open to the same values when second list is permuted under some hidden permutation.
We represent each list of ciphertexts as a matrix in $\GG_2^{2\times n}$ where each column is an El-Gamal ciphertext under public key $pk:=[\vecb{v}]_2\in\GG^2_2$ and we write $\enc_{pk}([\vecb{m}^\top]_2;\vecb{r}^\top):=(\enc_{pk}([m_1]_2;r_1)||\cdots||\enc_{pk}([m_n]_2;r_n))$, where $[\vecb{m}]_2\in\GG_2^n$, $\vecb{r}\in\Z_q^n$, and $\enc_{pk}([m]_2;r):=[m]_2\vecb{e}_2+r[\vecb{v}]_2$. Similarly, through this section we will sometimes write $\GS.\Com_{ck}([\vecb{x}^\top]_\gamma;\matr{R}):=$\-$(\GS.\Com_{ck}([x_1]_\gamma;\vecb{r}_1)||\cdots||$$\GS.\Com_{ck}($$[x_n]_\gamma;$$\vecb{r}_n))$, where $\matr{R}=(\vecb{r}_1||\cdots||\vecb{r}_n)\in\Z_q^{2\times n}$.

The language of correct shuffles under public key $[\vecb{v}]_2\in\GG^2_2$ can can be defined as 
\begin{align*}
\Lang_{[\vecb{v}]_2,n,\mathsf{shuffle}}:=\{([\matr{C}]_2,&[\matr{D}]_2)\in\GG_2^{2\times n}\times \GG_2^{2\times n} :\\
                                                         &\exists \matr{P}\in\mathcal{S}_n,\grkb{\delta}\in\Z_q^n \text{ s.t. } {[\matr{C}]_2\matr{P}-[\matr{D}]_2 = \Enc_{pk}([\vecb{0}_{1\times n}]_2;\grkb{\delta}^\top)}\},
\end{align*}
where $\mathcal{S}_n$ is the set of permutation matrices of size $n\times n$. This definition can be generalized for any ``El-Gamal like'' encryption scheme as, for example, the BBS encryption scheme from \cite{C:BonBoySha04}.

We construct a NIZK Shuffle argument with linear proof size, specifically $4n+17$ elements of $\GG_1$, $14$ elements of $\GG_2$, and 1 element of $\Z_q$ in type III groups, and $6n+34$ group elements in Type I groups. The security is based on assumptions which are weaker than \sxdh~plus \SSDP~Assumption in  Type III groups. In Type I groups our construction can be based on assumptions which are all weaker than \lin{2}.\footnote{In the symmetric case, the proof system from Sect. \ref{sec:bits} can be based on assumptions which are all weaker than $\lin{2}$, similarly as done in \cite[Appendix C]{EPRINT:GonHevRaf15}.}
 
\subsection{Our construction}

Our proof system builds on a proof that a set of GS commitments open to elements in the set $S=\{[s_1]_1,\ldots,[s_n]_1\}$, where $\vecb{s}:=(s_1,\ldots,s_n)^\top\gets\dist_{n,1}$ and the $\dist_{n,1}\mbox{-}\kermdh$ Assumption holds in $\GG_1$. Given $[\matr{F}]_1\in\GG_1^{2\times n}$, where the $i$ th column is $[\vecb{f}_i]_1\gets\GS.\Com([x_i]_1)$, let $\vecb{x}:=({x}_1,\ldots,{x}_n)^\top=\matr{P}\vecb{s}$, for some permutation matrix $\matr{P}$, and given a commitment to $[y]_1:=[\vecb{s}^\top]_1\grkb{\delta}$, we prove that $([\matr{C}]_2,[\matr{D}]_2)\in\Lang_{[\vecb{v}]_2,n,\mathsf{shuffle}}$ as follows:
\begin{enumerate}[label=\alph*)]
\item Show that $[\matr{F}]_1\in\Lang_{ck,S}^n$, where $ck\gets\GS.\algK(\gk)$.\label{shuffle:a}
\item Give a GS proof for the satisfiability of $\sum_{i\in[n]}[s_i]_1-\sum_{j\in[n]}[{x}_j]_1=[0]_1$.\label{shuffle:b}
\item Give a GS proof for the satisfiability of
$
[\vecb{x}^\top]_1[\matr{C}^\top]_2-[\vecb{s}^\top]_1[\matr{D}^\top]_2=[y]_1[\vecb{v}^\top]_2.\label{shuffle:c}
$\footnote{While using slightly different assumptions, in \cite{AC:GroLu07} this step is also done.}
\end{enumerate}

A detailed description of our construction can be found in Fig. \ref{fig:shuffles}.

\subsubsection{Soundness Intuition.} Conditions \ref{shuffle:a} and \ref{shuffle:b} implies that $\vecb{x}$ is a permutation of $\vecb{s}$ or equivalently, $\matr{x}=\matr{P}\vecb{s}$ and $\matr{P}$ is a permutation matrix. Note that $\matr{P}$ is a permutation matrix iff $\matr{P}$ is a binary matrix and for each row and column there is at most one 1. Let's see in more detail why $\vecb{x}$ is a permutation of $\vecb{s}$. Condition \ref{shuffle:a} implies that each $x_i$ is an element from $\{s_1,\ldots,s_n\}$, which can be written as $\vecb{x}=\matr{P}\vecb{s}$, $\matr{P}\in\bits^{n\times n}$, where each row of $\matr{P}$ has a more one 1. But, given that there might be repeated elements, there might be also more than one 1 in some column of $\matr{P}$. For example, if $S=\{s_1,s_2,s_3\}$, it may be that $\vecb{x}=\smallpmatrix{s_2\\s_3\\s_1}=\smallpmatrix{0&1&0\\0&0&1\\1&0&0}\smallpmatrix{s_1\\s_2\\s_3}$ but also $\vecb{x}=\smallpmatrix{s_2\\s_3\\s_3}\smallpmatrix{0&1&0\\0&0&1\\0&0&1}\smallpmatrix{s_1\\s_2\\s_3}$. Condition \ref{shuffle:b} implies that there are no repeated $x_i$s unless one can break the $\dist_{n,1}\mbox{-}\kermdh$ assumption. Indeed, there are repeated $x_i$s iff  $(1,\ldots,1)\matr{P}$ (the row vector of ``frequencies" of $\vecb{x}$, which in the second example is $(0,1,2)$) is not equal to $ (1,\ldots,1)$. Given that \ref{shuffle:b} is equivalent to $((1,\ldots,1)-(1,\ldots,1)\matr{P})[\vecb{s}]_1=[0]_1$, then $((1,\ldots,1)-(1,\ldots,1)\matr{P})^\top$ is solution to the $\dist_{n,1}\mbox{-}\kermdh$ problem. We conclude that $\matr{P}$ is a permutation matrix and thus $\vecb{x}$ is a permutation of $\vecb{s}$.

The rest of the proof follows essentially the proof from \cite{AC:GroLu07}. Suppose that $[\matr{C}]_2=\Enc_{[\vecb{v}]_2}([\matr{m}^\top]_2)$ and $[\matr{C}]_2=\Enc_{[\vecb{v}]_2}([\matr{n}^\top]_2)$. Let $\vecb{k}=(-v_2/v_1,1)^\top$ the ``decryption key'' (i.e. $\vecb{v}^\top\vecb{k}=0$ and $(0,1)\vecb{k}=1$)\footnote{The availability of the decryption key $\vecb{k}$ in the soundness reduction is possible since the reduction samples by itself the language parameter $\vecb{v}$. Correspondingly Groth and Lu \cite{AC:GroLu07} proved \emph{Culpable Soundness} (also called co-soundness), which essentially requires the soundness adversary to produce the decryption key.}, we multiply by $\matr{k}$, on the right, the equation from condition $\ref{shuffle:c}$ to ``decrypt'' $[\matr{C}]_2$ and $[\matr{D}]_2$. We get that
$[\vecb{s}^\top]_1\matr{P}^\top[\matr{m}]_2-[\vecb{s}^\top]_1[\matr{n}]_2=[0]_T$, which implies that $\matr{P}^\top[\matr{m}]_2=[\matr{n}]_2$ unless $\matr{P}^\top[\matr{m}]_2-[\matr{n}]_2$ is a solution to the $\dist_{n,1}\mbox{-}\kermdh$. Finally this implies that $[\matr{C}]_2\matr{P}-[\matr{D}]_2$ is an encryption of $[\matr{0}_{n\times 1}]_2$ and thus $([\matr{C}]_2,[\matr{D}]_2)\in\Lang_{[\vecb{v}]_2,n,\mathsf{shuffle}}$. 

%\vspace*{-1cm}
\begin{figure}
\begin{\algSize}
$$
\begin{array}{ll}
\begin{array}{l}
\algK_1(\gk,[\vecb{v}]_2,n)\quad (\mathsf{S}_1(\gk,[\vecb{v}]_2,n))\\
\hline
S:=\{[s_1]_1,\ldots,[s_n]_1\}, \vecb{s}^\top\gets\distlin_{1,n}\\
\crs_\GS\gets\GS.\algK_1(\gk)\\
{\crs_\sfset\gets\Pi_\sfset.\algK_1(\gk,[\matr{M}]_1,[\matr{N}]_1,[\matr{\Lambda}]_1,n)}\\
\text{Return } \ \crs:=([\vecb{v}]_2,\crs_\GS,\crs_\sfset).\\
(\tau_\GS\gets\GS.\algS_1(\gk)\\
\tau_\sfset\gets\Pi_\sfset.\algS_1(\gk,L,[\matr{M}]_1,[\matr{N}]_1,[\matr{\Lambda}]_1,n).\\
\tau := (\vecb{s},\tau_\GS,\tau_\sfset)).\\
\\
\end{array}
&
\begin{array}{l}
{\algP(\crs, [\matr{C}]_2, [\matr{D}]_2,\langle \matr{P}, \grkb{\delta}\rangle)}\\
\hline
[\vecb{x}]_1 := \matr{P}[\vecb{s}]_1,
[y]_1 := [\vecb{s}^\top]_1\grkb{\delta}\\
\pi_\GS\gets\GS.\algP(\crs_\GS,\{\eq_1,\eq_2\},\langle[\vecb{x}]_1,[y]_1\rangle)\\
\pi_\sfset \gets \Pi_\sfset.\algP(\crs_\sfset, [\matr{F}]_1,\langle\matr{P}^\top,\matr{R}\rangle)\\
//{[\matr{F}]_1} = \GS.\Com_{\crs_\GS}([\vecb{x}^\top]_1;\matr{R})\\
\text{Return } \  (\pi_\GS,\pi_\sfset). \\
\\
\\
\\
\end{array}\\
\begin{array}{l}
{\algV(\crs,[\matr{C}]_2,[\matr{D}]_2,(\pi_\GS,\pi_\sfset))}\\
\hline
\mathsf{ans}_1 \gets \GS.\algV(\crs_\GS,\{\eq_1,\eq_2\},\pi_\GS)\\
\mathsf{ans}_2 \gets \Pi_\sfset.\algV(\crs_\sfset,[\matr{F}]_1,\pi_\sfset)\\
\text{Return } \ \mathsf{ans}_1\wedge\mathsf{ans}_2.
\end{array}
&
\begin{array}{l}
{\mathsf{S}_2(\crs,[\matr{C}]_1,[\matr{D}]_1,\tau)}\\
\hline
\pi_\GS\gets \GS.\algS_2(\crs_\GS,\{\eq_1,\eq'_2\},\tau_\GS)\\
\pi_\sfset\gets \Pi_\sfset.\algS_2(\crs_\sfset,[\matr{F}]_1,\tau_\sfset)\\
\text{Return }  (\pi_\GS,\pi_\sfset).
\end{array}
\end{array}
$$
\end{\algSize}
\caption{The proof system $\Pi_\mathsf{shuffle}$ for the language $\Lang_{[\vecb{v}]_2,n,\mathsf{shuffle}}$. $\Pi_\sfset$ is the proof system from Sect. \ref{sec:bin-lan-constr}. The matrices $\matr{M},\matr{N},\matr{\Lambda}$ are defined as
$\matr{M}:=\binom{\vecb{s}^\top}{\matr{0}_{1\times n}}$,
$\matr{N}:=(\vecb{u}_1||\vecb{u}_2),\matr{\Lambda}:= (1,\ldots,1)$, where $\vecb{u}_1,\vecb{u}_2$ are the GS commitment keys from $\crs_\GS$, and the equations are defined as $\eq_1:= \sum_{i\in[n]}[s_i]_1-\sum_{j\in[n]}[{x}_i]_1 = [0]_1$, $\eq_2:= [\vecb{x}^\top]_1[\matr{C}^\top]_2-[\vecb{s}^\top]_1[\matr{D}^\top]_2=[y]_1[\vecb{v}^\top]_2$, and $\eq'_2:= [\vecb{x}^\top]_1[\matr{C}^\top]_2-[1]_1[\vecb{s}^\top\matr{D}^\top]_2=[y]_1[\vecb{v}^\top]_2$   
. The proof size is $(4n+17)|\GG_1|+14|\GG_2|+1|\Z_q|$
%$|\pi_\GS|+|\pi_\sfset|=(2n+6)|\GG_1|+4|\GG_2|+1|\Z_q|+(2n+11)|\GG_1|+10|\GG_2|=(4n+17)|\GG_1|+14|\GG_2|+1|\Z_q|$.
\label{fig:shuffles}}
\end{figure}


